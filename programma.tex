\textbf{INHOUDSOPGAVE}

\begin{enumerate}
\def\labelenumi{\arabic{enumi}.}
\item
  \begin{quote}
  Antiracisme en dekolonisatie
  \end{quote}
\item
  \begin{quote}
  Werk, inkomen en participatie
  \end{quote}
\item
  \begin{quote}
  Economie
  \end{quote}
\item
  \begin{quote}
  Zorg
  \end{quote}
\item
  \begin{quote}
  Wonen en bouwen
  \end{quote}
\item
  \begin{quote}
  Onderwijs en wetenschap
  \end{quote}
\item
  \begin{quote}
  Kunst, cultuur en media
  \end{quote}
\item
  \begin{quote}
  Toegankelijkheid
  \end{quote}
\item
  \begin{quote}
  Recht op zelfb eschikking
  \end{quote}
\item
  \begin{quote}
  Klimaatrechtvaardigheid
  \end{quote}
\item
  \begin{quote}
  Dierenrechten- en welzijn
  \end{quote}
\item
  \begin{quote}
  Landbouw, visserij en voedsel
  \end{quote}
\item
  \begin{quote}
  Asiel en migratie
  \end{quote}
\item
  \begin{quote}
  Europa
  \end{quote}
\item
  \begin{quote}
  Internationale samenwerking
  \end{quote}
\item
  \begin{quote}
  Staatsinrichting en rechtsstaat
  \end{quote}
\item
  \begin{quote}
  Veiligheid en justitie
  \end{quote}
\item
  \begin{quote}
  BES-Eilanden en Koninkrijk
  \end{quote}
\item
  \begin{quote}
  Defensie
  \end{quote}
\item
  \begin{quote}
  Digitale rechten en technologie
  \end{quote}
\item
  \begin{quote}
  Woordenlijst
  \end{quote}
\end{enumerate}

\textbf{Legenda}

Blauw verwijst naar het woord in de begrippenlijst

\underline{Onderstreept} met een nummer erachter: verwijst naar het
betreffende achtergrondstuk in de bijlage

\textbf{ANTIRACISME EN DEKOLONISATIE}

BIJ1 staat voor een samenleving met respect en waardering voor onze
verschillen en met focus op wat ons bindt. Hierin is geen plaats is voor
welke vorm van racisme dan ook, niet voor anti-zwart racisme,
antisemitisme, anti-Aziatisch racisme, nazisme, moslimhaat of haat tegen
Roma en Sinti. Dit houdt ook in dat er erkenning moet komen voor
specifieke vormen van racisme, ook en júíst wanneer zij overlappen met
andere identiteiten die worden gemarginaliseerd.

Nederland heeft een lang verleden van mensenhandel, slavernij,
kolonialisme en uitbuiting ten behoeve van de eigen economische
welvaart. Dit gaat gepaard met een zelfbeeld van superioriteit en
onschuld. We kunnen deze valse identiteit en deze periode van onrecht
niet afsluiten als we niet bereid zijn excuses aan te bieden en actie te
ondernemen om scheve verhoudingen als gevolg hiervan recht te zetten. We
moeten \emph{met elkaar} de waarden van vrijheid en democratie
beschermen en hooghouden. Alleen op deze manier kunnen we \emph{met
elkaar} bouwen aan een eerlijke en voor iedereen veilige samenleving.
Daarom moeten we gezamenlijk de strijd aangaan tégen racisme en vóór
dekolonisatie.

\textbf{Structureel probleem}

Waar de huidige politiek er te vaak van uitgaat dat racisme in Nederland
eigenlijk niet bestaat en alleen gaat over `losse' gebeurtenissen, wil
BIJ1 dat de overheid erkent dat racisme en kolonialisme een belangrijke
en niet te onderschatten rol hebben gespeeld bij de vorming van onze
samenleving.

Die racistische en kolonialistische vorming werkt ook nu nog door.
Achterstallig rechtsherstel en de weigering racisme als een
institutioneel probleem aan te pakken, houdt de ongelijkheid op
verschillende vlakken in stand. Van een te laag advies voor de
middelbare school voor kinderen van kleur, tot racisme in de zorg. Van
wetgeving die de zelfbeschikking van moslima's schaadt, tot de manier
waarop ons rechtssysteem is ingericht. Van racisme bij de politie, tot
discriminatie op de arbeidsmarkt. Van racisme op de werkvloer, tot
algoritmes die verkeerde conclusies trekken op basis van bevooroordeelde
data.

En ook van de omgang met andere landen in internationaal verband tot het
gebrek aan urgentie voor de klimaatcrisis, die het hardst de landen
raakt die het minst bijdragen aan klimaatverandering. Het
kapitalistische systeem van (internationale) ongelijkwaardigheid,
uitbuiting en onderdrukking maakt nog steeds gebruik van racisme en een
genormaliseerd koloniaal denkbeeld. Vandaar gaat de strijd tegen racisme
en voor dekolonisatie dan ook hand in hand met de anti-imperialistische
strijd.

De bestrijding van racisme, kolonialisme en andere vormen van
ongelijkwaardigheid en onderdrukking is een kernwaarde van BIJ1 en moet
plaatsvinden op allerlei vlakken. Onderstaande punten zijn de eerste
stappen die we concreet kunnen zetten \emph{tegen} racisme en
\emph{voor} dekolonisatie.

\textbf{Racisme is geen bijzaak}

\begin{enumerate}
\def\labelenumi{\arabic{enumi}.}
\item
  \begin{quote}
  De aanpak van racisme in alle facetten van onze samenleving wordt
  topprioriteit. Er komt een Ministerie van Gelijkwaardigheid dat zich
  actief bezighoudt met de bestrijding van racisme en andere vormen van
  ongelijkwaardigheid.
  \end{quote}
\item
  \begin{quote}
  Er komt breed onafhankelijk onderzoek naar racisme binnen
  overheidsinstanties op alle niveaus: van gemeenten tot ministeries,
  van de belastingdienst tot de politie.
  \end{quote}
\item
  \begin{quote}
  Registratie van etniciteit door overheidsinstanties mag enkel gebeuren
  nadat er een grondige toetsing heeft plaatsgevonden van het nut van
  die registratie.
  \end{quote}
\item
  \begin{quote}
  We gaan artikel 1 van onze Grondwet beter naleven. Er wordt beter
  gehandhaafd op het bedenken en uitvoeren van racistisch beleid en op
  racistische uitingen.
  \end{quote}
\item
  \begin{quote}
  We verbieden en ontbinden organisaties waar racistisch ideeëngoed
  verspreid en uitgewisseld wordt.
  \end{quote}
\end{enumerate}

\textbf{Aanpak van racistisch geweld}

\begin{enumerate}
\def\labelenumi{\arabic{enumi}.}
\item
  \begin{quote}
  Er komt meer aandacht voor de aanpak van racistische terreur.
  Racistische aanvallen, waaronder aanvallen op synagogen en moskeeën,
  worden keihard bestreden door de-radicalisering en educatieve
  maatregelen. Verdere radicalisering in gevangenissen moet voorkomen
  worden.
  \end{quote}
\item
  \begin{quote}
  Aangiftes en meldingen over racistisch geweld worden serieuzer
  opgepakt. Een veilige samenleving voor iedereen wordt topprioriteit.
  Bovendien wordt nationaliteit toegevoegd als discriminatiegrond aan
  Artikel 137 van het Wetboek van Strafrecht.
  \end{quote}
\item
  \begin{quote}
  Er komen juridische definities van racisme, anti-zwart racisme,
  antisemitisme, anti-Aziatisch racisme, moslimhaat en haat tegen Roma
  en Sinti. Deze worden expliciet opgenomen in
  antidiscriminatiewetgeving en hebben betrekking op het structurele
  karakter van deze vormen van onderdrukking.
  \end{quote}
\item
  \begin{quote}
  \emph{Blackfacing}, zoals zwarte piet, wordt verboden in de openbare
  ruimte.
  \end{quote}
\item
  \begin{quote}
  Racistisch politiegeweld en etnisch profileren worden voorkomen en
  aangepakt.
  \end{quote}
\end{enumerate}

\textbf{Excuses en rechtsherstel}

\begin{enumerate}
\def\labelenumi{\arabic{enumi}.}
\item
  \begin{quote}
  De overheid maakt officiële excuses voor ons slavernijverleden en de
  koloniale bezettingen. Daarbij erkent Nederland de Indonesische
  onafhankelijkheidsdatum van 17 augustus 1945. Nederland draagt alle
  (juridische en financiële) consequenties die voortvloeien uit die
  excuses en erkenning.
  \end{quote}
\item
  \begin{quote}
  Er komt volledig rechtsherstel voor de Indische gemeenschap. Kosten
  noch moeite worden hierin gespaard.
  \end{quote}
\item
  \begin{quote}
  Er komt uitgebreid en onafhankelijk onderzoek naar de roof van
  kapitaal en eigendom uit voormalige koloniën en van Joden in de Tweede
  Wereldoorlog, de waarde van gestolen arbeid van tot slaaf gemaakten en
  contractarbeiders, en de gelden die Indonesië heeft betaald in ruil
  voor de soevereiniteitsoverdracht. Nederland zet zich onvoorwaardelijk
  in voor herstel.
  \end{quote}
\item
  \begin{quote}
  Keti Koti wordt een nationale feestdag. Het herdenken van het leed van
  de slavernij en het vieren van het einde ervan wordt onderdeel van de
  Nederlandse cultuur.
  \end{quote}
\item
  \begin{quote}
  Koloniale straatnamen krijgen duiding of worden weggehaald. Beelden
  van koloniale figuren en de Gouden Koets krijgen een nieuwe bestemming
  in musea. De vrijgekomen plaats in de openbare ruimte wordt
  bijvoorbeeld gevuld door beelden van antikoloniale verzetshelden.
  Kunstenaars en experts met wortels in de voormalige koloniën, alsmede
  omwonenden, krijgen hierbij doorslaggevende inspraak.
  \end{quote}
\end{enumerate}

\textbf{Geen plek voor nazisme en haat}

\begin{enumerate}
\def\labelenumi{\arabic{enumi}.}
\item
  \begin{quote}
  Er komen landelijke programma's om antisemitisme, anti-zwart racisme
  en moslimhaat aan te pakken. Tot deze programma's behoort een
  onafhankelijk meldpunt. Meldingen van anti-Joods, anti-zwart en
  anti-moslimgeweld worden voortaan besproken in de
  veiligheidsdriehoeken.
  \end{quote}
\item
  \begin{quote}
  We stellen een eerlijke definitie van antisemitisme op, zodat
  klassiek-nazistische denkbeelden en complottheorieën over het
  `gedegenereerde Joodse ras', `rijke Joden', een `wereldomspannende
  Joodse samenzwering' en `cultuurmarxisme' adequaat kunnen worden
  aangepakt. De IHRA-definitie is geen eerlijke definitie van
  antisemitisme.
  \end{quote}
\item
  \begin{quote}
  Het bezit van nazi-attributen moeten enkel zijn voorbehouden aan musea
  en vergunninghouders. Privé-bezit en handel wordt verboden, tenzij men
  een vergunning heeft.
  \end{quote}
\item
  \begin{quote}
  Holocaustontkenning en Hitler-aanbidding worden bij wet verboden.
  \end{quote}
\item
  \begin{quote}
  We zorgen dat gemeenten doordrongen zijn van het
  gelijkwaardigheidsprincipe in het contact met Roma, Sinti en
  woonwagenbewoners. Het uitsterfbeleid tegen Roma, Sinti en
  woonwagenbewoners mag dan van tafel zijn, het stigma in de bejegening
  van deze gemeenschappen door de overheid en gemeenten is dat niet. Er
  wordt geïnvesteerd in het behouden en stimuleren van hun culturen.
  \end{quote}
\end{enumerate}

\textbf{WERK, INKOMEN EN PARTICIPATIE}

In een kapitalistische samenleving wordt onze arbeidskracht als een waar
verkocht. Zeggenschap over ons eigen werk is ver te zoeken: op de
productie die we draaien of de diensten die we verlenen maken de bazen
hun winsten. BIJ1 vindt dat werk geen uitbuiting mag zijn en werkende
mensen recht hebben om de vruchten te plukken van hun eigen werk. We
staan voor bestaanszekerheid, zeggenschap en zelfbeschikking in je werk,
een eerlijke verdeling van werk en een inclusieve arbeidsmarkt.

\textbf{Bestaanszekerheid en eerlijke waardering}

BIJ1 wil niet dat we werken om te overleven en vindt dat
bestaanszekerheid, de zekerheid om goed te kunnen leven, voor iedereen
de basis moet zijn. Daarnaast is ook huishoudelijk werk, mantelzorg en
vrijwilligerswerk van grote waarde voor de maatschappij en dit verdient
dan ook onze waardering. Flexibilisering van de arbeidsmarkt moet worden
aangepakt: vaste contracten en daardoor meer zekerheid moeten de norm
worden. Bovendien moet er een eerlijke betaling komen van het werk van
jongeren tot dat van mensen in de zorg en het onderwijs.

\textbf{Zeggenschap en zelfbeschikking}

Als we spreken over eerlijke waardering, spreken we ook over hoe de
economie in zijn geheel is ingericht. In deze economie staat het maken
van winsten centraal. De winsten komen echter niet toe aan wie er hard
voor werkt, maar aan de bedrijfseigenaren. Dat moet anders. Werkende
mensen moet meer \underline{zeggenschap over hun werk (1)} krijgen.
Daarnaast moet ook het werk van sekswerkers worden gewaardeerd als
volwaardig werk, waarin zelfbeschikking voorop staat.

\textbf{Eerlijke verdeling van werk en een inclusieve arbeidsmarkt\\
}Werk moet toegankelijk zijn voor iedereen, maar dat is nu nog niet het
geval. Racisme, seksisme, validisme en andere vormen van onderdrukking
moeten keihard worden bestreden. Ook moet werk eerlijker worden verdeeld
en pakken we genderongelijkheid aan. Mensen met een (onzichtbare)
beperking verdienen daarnaast een eerlijke kans op een duurzame baan en
eerlijke betaling, gekeken wordt naar mogelijkheden en niet alleen naar
moeilijkheden. Arbeidsmigranten verdienen daarnaast dezelfde rechten als
Nederlandse werknemers.

\textbf{Bestaanszekerheid en eerlijke waardering van werk voorop}

\begin{enumerate}
\def\labelenumi{\arabic{enumi}.}
\item
  \begin{quote}
  Het minimumloon wordt verhoogd naar 14 euro per uur. Hiermee verhogen
  we ook de AOW en de bijstand. Het minimumjeugdloon wordt gelijk aan
  het volwassen minimumloon.
  \end{quote}
\item
  \begin{quote}
  Werk dat nu is ondergewaardeerd, zoals het werk van zorgmedewerkers,
  leraren en medewerkers van het openbaar vervoer, moet een eerlijke
  waardering krijgen. Dit vertaalt zich niet alleen in gepast loon maar
  ook in de arbeidsvoorwaarden.
  \end{quote}
\item
  \begin{quote}
  We maken flex- en uitzendwerk duurder voor werkgevers. Werknemers die
  langer dan negen maanden met een tijdelijk contract werken, krijgen
  een vast contract. We verbeteren de ontslagbescherming.
  \end{quote}
\item
  \begin{quote}
  Bij stages hoort het leren centraal te staan. Stagiairs zijn geen
  (goedkope) arbeidskracht. Voor stages komt er altijd een eerlijke
  vergoeding. Wanneer een student een stageverplichting heeft, mag het
  bemachtigen van een stageplek niet geheel afhankelijk zijn van
  sollicitatiesucces. Waar nodig of wenselijk wordt gewerkt met
  toewijzing. De Inspectie SZW krijgt meer middelen om
  stagediscriminatie en uitbuiting op te sporen en harder te bestraffen.
  \end{quote}
\item
  \begin{quote}
  We zetten zinvol onbetaald werk zoveel mogelijk om in betaalde
  arbeidsplaatsen.
  \end{quote}
\item
  \begin{quote}
  Mensen met een (onzichtbare) beperking krijgen evenveel betaald als
  vakgenoten zonder beperking met hetzelfde werk.
  \end{quote}
\end{enumerate}

\textbf{Meer zeggenschap over je eigen werk}

\begin{enumerate}
\def\labelenumi{\arabic{enumi}.}
\item
  \begin{quote}
  De zeggenschap over werk wordt uit de handen van bedrijfseigenaren
  gehaald. Werknemers krijgen zélf zeggenschap over hun eigen werk
  doordat bedrijven in handen van werknemers democratisch worden
  bestuurd.
  \end{quote}
\item
  \begin{quote}
  We verruimen het recht op deeltijdarbeid, ook in leidinggevende en
  beleidsfuncties. Daarbij moet er meer vrijheid komen voor werknemers
  om eigen werktijden te bepalen.
  \end{quote}
\item
  \begin{quote}
  We gaan schijnzelfstandigheid tegen. Voor alle schijnzelfstandigen,
  van postbezorgers tot fietskoeriers, garanderen we correcte loon- en
  arbeidsrechten. We stimuleren platformcoöperaties die door de werkende
  mensen zelf worden bestuurd.
  \end{quote}
\item
  \begin{quote}
  We zorgen dat alle bedrijfstakken een cao hebben, zodat bescherming
  van werknemers optimaal geregeld is.
  \end{quote}
\item
  \begin{quote}
  Het stakingsrecht wordt uitgebreid: het moet niet zo zijn dat je
  alleen mag staken wanneer het de autoriteiten en werkgevers uitkomt.
  \end{quote}
\item
  \begin{quote}
  Vakbonden die onvoldoende onafhankelijk zijn van werkgevers dienen
  niet de belangen van de werknemers. Deze vakbonden worden uitgesloten
  van cao-onderhandelingen.
  \end{quote}
\item
  \begin{quote}
  Uitzendbureaus die leraren, zorgpersoneel of anderen in publieke
  sectoren uitzenden, worden verboden.
  \end{quote}
\end{enumerate}

\textbf{Naar een inclusieve en toegankelijke arbeidsmarkt}

\begin{enumerate}
\def\labelenumi{\arabic{enumi}.}
\item
  \begin{quote}
  We willen een eerlijkere verdeling van werk tussen mensen van
  verschillende genderidentiteiten. Dit doen we door mogelijkheden te
  bieden om betaald en onbetaald werk te herverdelen tussen partners.
  \end{quote}
\item
  \begin{quote}
  We gaan naar een 30-urige werkweek met loonbehoud. Dit zorgt er ook
  voor dat we tijd voor gezin en hobby's vrij hebben, burn-outs worden
  bestreden en we nieuwe banen creëren.
  \end{quote}
\item
  \begin{quote}
  We maken de kinderopvang gratis voor alle ouders.
  \end{quote}
\item
  \begin{quote}
  Partnerverlof wordt gelijkgetrokken met het kraam- en
  bevallingsverlof. Voor alle ouders op verlof wordt er 100\% inkomen
  doorbetaald.
  \end{quote}
\item
  \begin{quote}
  We gaan strenger handhaven op de geldende regels voor het faciliteren
  van bijvoorbeeld kolven of bidden op de werkvloer.
  \end{quote}
\item
  \begin{quote}
  We stellen een menstruatieverlof in.
  \end{quote}
\item
  \begin{quote}
  Onbetaald werk dat nog vaak als `vrouwenwerk' wordt gezien, zoals
  mantelzorg, huishoudelijk werk of zorg voor kinderen wordt betaald
  werk. Zo werken we aan \underline{genderrechtvaardigheid (2)}.
  \end{quote}
\item
  \begin{quote}
  We werken aan toegankelijke werkplekken. Dit doen we door het recht op
  thuiswerk vast te leggen indien de aard van het werk dat toelaat. Ook
  komt er meer (onafhankelijke) ondersteuning op de werkvloer zodat
  mensen met een beperking beter naar eigen inzicht kunnen participeren.
  \end{quote}
\item
  \begin{quote}
  De mislukte Participatiewet wordt afgeschaft: deze leidt niet tot meer
  baankansen. In plaats daarvan zetten we in op Sociale
  Ontwikkelbedrijven naar idee van de FNV. Zo helpen we mensen met een
  beperking aan duurzaam, passend werk met begeleiding en fatsoenlijke
  arbeidsvoorwaarden. Deze bedrijven bieden beschut werk, begeleiding
  naar werk op de reguliere arbeidsmarkt en allerlei vormen van begeleid
  werk daartussen.
  \end{quote}
\end{enumerate}

\textbf{Geen plek voor discriminatie}

\begin{enumerate}
\def\labelenumi{\arabic{enumi}.}
\item
  \begin{quote}
  We houden discriminerende bedrijven strafrechtelijk verantwoordelijk
  en versterken de instrumenten van de Inspectie SZW. Zij gaat strenger
  ingrijpen op overtredingen van het arbeidsrecht en op
  arbeidsmarktdiscriminatie. We zetten in op namen-en-shamen, hoge
  boetes en het stopzetten van subsidies en samenwerking met overheden.
  \end{quote}
\item
  \begin{quote}
  Het Diversity Rating System, DRS, wordt operationeel gemaakt door het
  Sociaal en Cultureel Planbureau en we starten een proef in grote
  gemeenten. De overheid hanteert het Diversity Rating System ook voor
  bedrijven en organisaties waarmee het zaken doet.
  \end{quote}
\item
  \begin{quote}
  We maken vóór 2025 een einde aan loon- en inkomensongelijkheid tussen
  mannen en vrouwen. Bedrijven die mannen meer betalen dan vrouwen met
  dezelfde functie worden vervolgd.
  \end{quote}
\item
  \begin{quote}
  We schaffen de loondispensatieregeling in de Wajong af: werknemers met
  een beperking worden als volwaardig medewerker aangenomen en betaald.
  Productievermogen mag een leefbaar loon niet in de weg staan.
  \end{quote}
\item
  \begin{quote}
  Ongedocumenteerden die te maken krijgen met geweld of dwang op het
  werk, moeten dit kunnen melden zonder dat ze risico lopen op
  uitzetting.
  \end{quote}
\item
  \begin{quote}
  De rechten van arbeidsmigranten moeten beter worden beschermd. Er komt
  betere regelgeving voor arbeidsomstandigheden, vergoedingen,
  minimumloon, en maatstaven voor huisvesting. Uitzendcentra en
  werkgevers komen onder scherp toezicht te staan en arbeidsmigranten
  krijgen betere voorlichting over hun rechten.
  \end{quote}
\item
  \begin{quote}
  Arbeidsmigranten moeten dezelfde rechten op beloning en
  arbeidsvoorwaarden hebben als werkenden met een Nederlandse
  nationaliteit.
  \end{quote}
\end{enumerate}

\textbf{Sekswerk is werk}

\begin{enumerate}
\def\labelenumi{\arabic{enumi}.}
\item
  \begin{quote}
  De rechten van sekswerkers worden gelijkgetrokken met de
  arbeidsrechten van andere zelfstandigen en werknemers.
  \end{quote}
\item
  \begin{quote}
  Sekswerkers krijgen inspraak in en bepalen mede het beleid over
  sekswerk.
  \end{quote}
\item
  \begin{quote}
  De vergunningseis voor zelfstandige sekswerkers wordt opgeheven.
  Sekswerkers krijgen de mogelijkheid om thuis een `eenmanszaak' te
  starten. Daarbij stellen we hen in staat diensten bij derden in te
  kopen zonder dat die partij gecriminaliseerd wordt.
  \end{quote}
\item
  \begin{quote}
  Sekswerkers worden nooit verplicht bijzondere persoonlijke informatie
  te delen of zich te laten registreren. De Wet Regulering Sekswerk
  (WRS) gaat van tafel - sekswerk wordt gedecriminaliseerd.
  \end{quote}
\item
  \begin{quote}
  We versimpelen het vergunningenstelsel voor exploitanten. Als
  sekswerkers het willen, moet het makkelijk zijn werkplekken te creëren
  waar zij kunnen werken met gedeelde voorzieningen en in nabijheid van
  collega's.
  \end{quote}
\item
  \begin{quote}
  De leeftijdsgrens voor sekswerk blijft op 18 jaar en gaat niet naar 21
  jaar.
  \end{quote}
\end{enumerate}

\textbf{Bestaanszekerheid, ook als je niet (meer) werkt}

\begin{enumerate}
\def\labelenumi{\arabic{enumi}.}
\item
  \begin{quote}
  Persoonlijke ontwikkeling is in het belang van ons allemaal. We steken
  daarom meer geld in scholingsprogramma's voor mensen zonder werk. We
  laten werkgevers uit sectoren met personeelstekorten meebetalen aan de
  scholing van goed personeel.
  \end{quote}
\item
  \begin{quote}
  De tegenprestaties, sancties en repressieve maatregelen in de bijstand
  worden afgeschaft. Wachttijden voor het verstrekken van een uitkering
  worden ook stevig verkort.
  \end{quote}
\item
  \begin{quote}
  De kostendelersnorm wordt afgeschaft om financiële onafhankelijkheid
  te bevorderen.
  \end{quote}
\item
  \begin{quote}
  Arbeidsongeschiktheid is geen keuze en mensen die arbeidsongeschikt
  zijn kunnen goed zelf inschatten waartoe zij in staat zijn. De
  medische keuringen bij arbeidsongeschiktheid worden niet meer door de
  uitkeringsinstantie zelf gedaan, maar door onafhankelijke artsen.
  \end{quote}
\item
  \begin{quote}
  Er komt een Nationaal Pensioenfonds waarbij collectiviteit voorop
  staat en waarop ook zzp'ers aanspraak maken. Pensioengerechtigden
  krijgen zeggenschap in dit Pensioenfonds en topsalarissen en foute
  beleggingen worden uitgebannen. Daarbij indexeren we de pensioenen:
  pensioenfondsen hebben nu immers genoeg geld in kas om gegarandeerd
  pensioen uit te keren.
  \end{quote}
\item
  \begin{quote}
  We verlagen de AOW-leeftijd naar 65. Wel blijft doorwerken mogelijk en
  gaan we strenger handhaven op arbeidsmarktdiscriminatie op basis van
  leeftijd. Het zogenoemde AOW-gat, waar veel Surinaams-Nederlandse
  ouderen en ouderen met een migratieachtergrond mee te maken hebben,
  wordt gedicht.
  \end{quote}
\item
\end{enumerate}

\textbf{ECONOMIE}

De huidige verdeling van welvaart, bezit en macht komt slechts een klein
deel van onze samenleving ten goede. BIJ1 wil naar een economie die
werkt voor iedereen en niet alleen voor een klein groepje mensen. We
willen naar een democratische economie, waarin winsten en prijzen worden
beheerst. Een toekomstbestendige economie die er is voor iedereen en
vrij is van racisme, uitsluiting en uitbuiting van de planeet, mens en
dier.

Het kapitalisme is een systeem dat zoveel mogelijk wil produceren tegen
zo min mogelijk kosten. Dat gaat ten koste van mens, dier en de planeet.
Dat systeem is onhoudbaar. Economische crises volgen elkaar op,
klimaatverandering bedreigt het leven op onze planeet en de kloof tussen
arm en rijk wordt groter. De macht van grote bedrijven is een bedreiging
voor de democratie, bestaanszekerheid is er niet voor iedereen en
wereldwijd worden vele mensen uitgebuit zodat enkelen in welvaart kunnen
leven. Kwaliteit en duurzaamheid zijn in dit systeem geen prioriteit en
werkende mensen hebben geen zeggenschap over hun werk.

\textbf{Weg met de dictatuur van de `vrije' markt}

Waar de zogenaamde `vrije' markt spreekt over vrijheid, is het in feite
een dictatuur die werkende mensen uitbuit. De winsten worden immers
gemaakt over de ruggen van de werknemers. Bovendien hangt het aanbod van
producten en diensten niet af van behoefte, maar van de winsthonger van
de bedrijfseigenaren. Zo ontstaan overschotten en tekorten, worden
onnodige producten van slechte kwaliteit gemaakt en blijft een duurzame
economie ver uit zicht.

BIJ1 wil van de grond af werken aan een economie die democratisch
gepland wordt door iedereen die eraan deel heeft. We gaan de producten
en hoeveelheden produceren waar behoefte aan is en de opbrengsten komen
ten gunste van de werkende mensen. Werkende mensen krijgen
\underline{zeggenschap over hun werk (1)} en prijzen en winsten worden
beheerst.

\textbf{Geen kloof tussen arm en rijk}

We gaan de groeiende kloof tussen de mensen met lage lonen en de
allerrijksten dichten. Een rechtvaardige economie die voor ons allemaal
werkt, is een economie waarin geld op een rechtvaardige manier verdeeld
wordt. De sterkste schouders dragen dus de zwaarste lasten. In de aanpak
van armoede moeten we \underline{van armoedebestrijding naar
armoedepreventie (3)}. Ook beseffen we hoe vooroordelen en aannames over
afkomst en kleur bepalend zijn voor wie in armoede terecht komt.

\textbf{Help mensen uit de schulden, niet erin}

De consumptiemaatschappij is erop gericht dat je maar blijft kopen. De
schuldeneconomie die gecreëerd wordt door de markt is winstgevend,
waardoor schuldeisers dit graag in stand houden. Al het recht staat aan
de kant van de schuldeiser, op zowel individueel als landsniveau. BIJ1
wil dat schuld een gedeelde verantwoordelijkheid wordt van schuldeiser
en degene die de schuld heeft. Bedrijven en overheid als schuldeisers
mogen niet meer profiteren van de schulden en mensen zo verder de
armoede in drijven.

\textbf{Klimaatrechtvaardigheid}

De strijd tegen klimaatverandering betekent dat we opnieuw moeten kijken
naar de menselijke relatie tot land, arbeid, dieren, de natuur en de
economie. Het kapitalisme, imperialisme en kolonialisme hebben gezorgd
voor een globaal systeem van klimaatvervuiling. We willen
klimaatrechtvaardigheid, waarbij we de rol van westerse machtssystemen
erkennen, daarvoor verantwoordelijkheid nemen en de sterkste schouders
de zwaarste lasten laten dragen: een \emph{Decolonial Green Deal}.

Voor een rechtvaardige economie stelt BIJ1 de volgende maatregelen voor.

\textbf{Haal het geld waar het zit}

\begin{enumerate}
\def\labelenumi{\arabic{enumi}.}
\item
  \begin{quote}
  We heffen een eenmalige coronataks van 5\% op de financiële vermogens
  en het vastgoed van multimiljonairs met meer dan 3 miljoen euro.
  Hiermee dekken we (een deel van) de maatschappelijke kosten die zijn
  ontstaan door de coronacrisis.
  \end{quote}
\item
  \begin{quote}
  Private bedrijven die in Nederland hun producten verkopen gaan forse
  winstbelasting betalen en voordelige belastingconstructies worden
  gestopt. Rijke mensen gaan forse vermogens- en erfbelasting betalen.
  \end{quote}
\item
  \begin{quote}
  Nederland als belastingparadijs wordt onmogelijk gemaakt. Ook wordt
  onmogelijk gemaakt dat bedrijven in Nederland gebruik maken van
  vluchtroutes om hun belasting te ontwijken.
  \end{quote}
\item
  \begin{quote}
  Open grenzen voor mensen, niet voor kapitaal: we gaan kapitaalvlucht
  tegen. In internationaal verband investeren we in het opsporen en
  vervolgen van belastingontduikers.
  \end{quote}
\item
  \begin{quote}
  De inkomstenbelasting voor hoge inkomsten uit box 1 en 2 gaat
  proportioneel omhoog.
  \end{quote}
\end{enumerate}

\textbf{Van een markteconomie naar een economie voor iedereen}

\begin{enumerate}
\def\labelenumi{\arabic{enumi}.}
\item
  \begin{quote}
  We brengen belangrijke sectoren van de economie, zoals banken,
  pensioenfondsen, het openbaar vervoer, de zorg en andere
  basisindustrieën in publieke handen. Deze industrieën komen onder
  controle van de werknemers. De overheid faciliteert hen waar nodig.
  \end{quote}
\item
  \begin{quote}
  Productieplanning in iedere sector en bedrijf komt in handen van de
  werknemers. De overheid faciliteert hen en houdt toezicht. Hierdoor
  ontstaat een betere en eerlijkere balans tussen vraag en aanbod.
  Prijzen en winsten worden zo beheerst.
  \end{quote}
\item
  \begin{quote}
  Zeggenschap over werk betekent dat werknemers ook moeten meedelen in
  de winsten. Dit doen we door sectoren in publieke handen te brengen en
  meer wettelijke regelingen in te stellen voor winst- en
  vermogensaanwasdeling.
  \end{quote}
\item
  \begin{quote}
  Ten behoeve van de zorg, wonen en openbaar vervoer versoepelen we
  onteigeningswetgeving fors omtrent kapitaal, aandelen, bezit en grond.
  \end{quote}
\end{enumerate}

\textbf{Produceren naar vermogen, nemen naar behoefte}

\begin{enumerate}
\def\labelenumi{\arabic{enumi}.}
\item
  \begin{quote}
  We gaan produceren naar behoefte en niet meer om winsten te maken voor
  bedrijfseigenaren. We produceren zoveel mogelijk op lokale schaal.
  \end{quote}
\item
  \begin{quote}
  We laten het maatschappelijk nut van een product de prijs bepalen. Een
  benzineauto kan nu misschien goedkoop geproduceerd worden, maar is in
  verband met verduurzaming maatschappelijk ongewenst. Daarom maken we
  deze producten duurder.
  \end{quote}
\item
  \begin{quote}
  We maken optimaal gebruik van robotisering. Robotisering is echter
  geen goedkope vervanging van arbeid: we garanderen een leefbaar
  inkomen en werkgelegenheid door verkorting van de werkweek.
  \end{quote}
\end{enumerate}

\textbf{Einde aan de schuldeneconomie}

\begin{enumerate}
\def\labelenumi{\arabic{enumi}.}
\item
  \begin{quote}
  De gedeelde verantwoordelijkheid van schuldeisers voor schulden wordt
  wettelijk vastgelegd. Hierdoor dwingen we hen tot het stoppen met het
  heffen van rentes en het verkopen op afbetaling als winstmodel.
  Kwijtschelden van schulden wordt gestimuleerd, incassobureaus worden
  verboden.
  \end{quote}
\item
  \begin{quote}
  De overheid neemt schulden vaker over, zodat er maar één schuldeiser
  is en schulden niet oplopen. Ook saneert de overheid vaker schulden,
  zodat mensen een nieuwe start kunnen maken.
  \end{quote}
\item
  \begin{quote}
  Schuldhulporganisaties die geld verdienen door andermans schulden
  worden verboden: de overheid neemt zelf de schuldhulpverlening weer in
  handen. We zorgen daarbij voor een supportsysteem van budgethulp en
  een breder sociaal netwerk door te investeren in welzijn en sociaal
  werk in de wijken.
  \end{quote}
\end{enumerate}

\textbf{Een circulaire en toekomstbestendige economie}

\begin{enumerate}
\def\labelenumi{\arabic{enumi}.}
\item
  \begin{quote}
  Er komt een grootschalig onderzoek naar de impact van import- en
  exportproducten uit alle sectoren op mens en klimaat in binnen- en
  buitenland. Hierbij wordt specifiek gekeken naar hoe importproducten
  en services een klimaatonvriendelijk, onrechtvaardig en
  imperialistisch systeem mogelijk maken en in stand houden. Goederen
  afkomstig uit niet-duurzame en/of onrechtvaardige productie worden
  uitgefaseerd en uiterlijk in 2025 toegang tot de markt ontzegd.
  \end{quote}
\item
  \begin{quote}
  Bij internationale handel is rechtvaardige en klimaatvriendelijke
  handel de norm. Ook handelsverdragen (bestaande en nieuwe) worden op
  deze manier beoordeeld. We trekken ons terug uit vrijhandelsverdragen
  als CETA.
  \end{quote}
\item
  \begin{quote}
  We bieden opleidingsmogelijkheden en creëren banen voor de enorme
  duurzaamheidsslag die we moeten maken.
  \end{quote}
\end{enumerate}

\textbf{Internationale rechtvaardigheid}

\begin{enumerate}
\def\labelenumi{\arabic{enumi}.}
\item
  \begin{quote}
  De huidige imperialistische plundering van voormalige koloniën en
  andere niet-Westerse landen door Nederland en Nederlandse bedrijven
  wordt gestopt. Nederlandse bedrijven die investeren in buitenlandse
  projecten of ondernemingen, of die (grondstoffen voor) hun producten
  laten maken in het buitenland, moeten kunnen aantonen dat dit niet ten
  nadele van de lokale bevolking en klimaat gaat. Op Europees niveau
  steunt Nederland de ontwikkeling van zogenaamde `due diligence'
  wetgeving die bedrijven verplicht verantwoording af te leggen over de
  herkomst van hun producten en de impact op mens en milieu.
  \end{quote}
\item
  \begin{quote}
  Economische samenwerkingen met groepen en landen die onterecht door
  Westerse landen gemeden worden of sancties opgelegd worden, zoals Cuba
  en Venezuela, wordt onderzocht en gestimuleerd.
  \end{quote}
\item
  \begin{quote}
  We maken een einde aan Nederland als belastingparadijs voor
  wapenproducenten en de industrie van militaire technologieën. We
  verbieden de vestiging van wapenbedrijven in Nederland en heffen hoge
  belastingen voor de bestaande wapenindustrie. We vormen de
  wapenindustrie om tot civiele industrie.
  \end{quote}
\end{enumerate}

\textbf{ZORG}

BIJ1 staat voor een zorgstelsel dat \emph{van}, \emph{door} en
\emph{voor} iedereen is. We willen zorg die toegankelijk, betaalbaar,
rechtvaardig, solidair en gelijkwaardig is voor iedereen. De mens in
plaats van het geld hoort centraal te staan. Daarom moet de zorg in
handen van de overheid komen. De zorg moet bestuurd worden door de
mensen die daadwerkelijk met zorg in aanraking komen: de zorgverlener en
degene die zorg ontvangt. Zo werken we aan een zorg die door en voor
iedereen werkt.

\textbf{De zorg is geen markt}

We hebben de zorg lange tijd overgelaten aan de verwoestende vrije
markt. Kwaliteit en innovatie hebben nu te lijden onder winstbejag en
concurrentie. Kennis, ervaring en kapitaal moeten tegen elkaar opboksen.
Marktwerking heeft de zorg vervolgens alleen maar duurder gemaakt. Grote
farmaceuten kunnen door hun feitelijke monopolies de prijzen verder
opdrijven. Ziekenhuizen vallen om waardoor langere wachtlijsten ontstaan
en adequate zorg niet meer geboden kan worden.

Het recht op zorg staat daarmee onder druk. De complexiteit van de
verschillende zorgwetten zorgt er daarnaast voor dat mensen steeds
moeilijker bij de juiste zorg te komen. De decentralisaties werden
vanuit een goed principe gedaan: zorg moet dichter bij de mensen worden
georganiseerd. Zij waren echter ook een bezuinigingsmaatregel die de
zorg verder tot een doolhof heeft gemaakt.

BIJ1 wil zorgbehoefte en kwaliteit laten leiden in plaats van winsten en
kostenbesparing. Het zorgsysteem, van de jeugdzorg tot de medische zorg,
moet op de schop. Concurrentie moet plaatsmaken voor samenwerking. Door
de zorg terug te brengen in de handen van de overheid, kan de zorg
\emph{van} iedereen worden. Een Nationaal Zorgfonds maakt de zorg
betaalbaar zonder onnodige bureaucratie en geldverslindende
concurrentie. Daarbij verstevigen we het recht op zorg, moet iedereen
makkelijk zorg of ondersteuning kunnen vragen en wordt het systeem,
waarin iedereen in behandelbare hokjes moet passen, aangepakt.

\textbf{Zorgmedewerkers en ervaringsdeskundigen aan zet}

BIJ1 leert van generaties aan ervaringsdeskundigheids- en
\emph{disability} activisten en heeft als basis: \emph{niets over ons,
zonder ons}. Dit geldt ook in de zorg. Want niemand weet beter hoe het
werkt dan de mensen die zorg v

erlenen en de mensen die zorg ontvangen. Kennis van
\underline{ervaringsdeskundigen (4)} (zowel hulpverleners als
hulpontvangers) moet dan ook een gelijkwaardige positie hebben naast
wetenschappelijke kennis. De zorg moet in handen zijn van de mensen,
niet van aandeelhouders, managers en directeuren.

BIJ1 wil dat zorg draait om zeggenschap, zelfbeschikking en zekerheid.
Zeggenschap over je eigen werk als je in de zorg werkt. Zelfbeschikking
over de zorg die je krijgt. En zekerheid op een eerlijk salaris en de
vrijheid om je vak uit te voeren zonder dat je tijd wegvloeit naar
administratie, maar ook de zekerheid van kwalitatieve zorg.

\textbf{Geen discriminatie, maar inclusie}

In onze samenleving worden mensen met een zorgvraag gezien als een
individu met een probleem. Toch is vaak de inrichting van de
maatschappij de eigenlijke oorzaak van het probleem. Er zijn te veel
mensen die minder toegang hebben tot goede zorg, doordat ze
buitengesloten worden. Er is meer ruimte nodig voor kennis over en
ervaring met de grote diversiteit aan lichamen en identiteiten. Een
grotere inclusiviteit is nodig voor betere ziektepreventie en zorg op
maat.

BIJ1 staat voor een zorg die toegankelijk is, ook voor neurodivergente
en LHBTQI+ mensen. Racisme, seksisme en validisme zijn van invloed op de
kwaliteit van de zorg en moeten koste wat kost worden bestreden.~

Om de zorg toegankelijker en rechtvaardiger te maken, stelt BIJ1 de
volgende maatregelen voor.

\textbf{Zorg van iedereen: maak de zorg van ons allen}

\begin{enumerate}
\def\labelenumi{\arabic{enumi}.}
\item
  \begin{quote}
  We zetten een Nationaal Zorgfonds op dat de vele verzekeraars
  vervangt. Premies gaan we heffen via progressieve belastingen. Alle
  zorg, inclusief tandheelkunde en optometrie, wordt vanuit dit
  Zorgfonds betaald en de eigen bijdrage wordt afgeschaft.
  \end{quote}
\item
  \begin{quote}
  De zorg mag geen winstmodel zijn. We brengen ziekenhuizen en
  (para)medische zorgaanbieders in publieke handen. In plaats van geld
  te verliezen aan managers, commercie en winst, investeren we meer in
  personeel en kwalitatieve zorg. Zo blijft ook de continuïteit van zorg
  gewaarborgd.
  \end{quote}
\item
  \begin{quote}
  Niet-medische zorginstellingen en -organisaties, zoals in de jeugdzorg
  en GGZ, worden genationaliseerd en gecollectiviseerd. Dit betekent dat
  we bijvoorbeeld verschillende jeugdzorgorganisaties onderbrengen in
  één organisatie. Hiermee bestrijden we schotten tussen organisaties en
  zorgvormen, werken we op basis van samenwerking en afstemming, en
  houden organisatiebelangen en bureaucratie tussen organisaties op te
  bestaan.
  \end{quote}
\item
  \begin{quote}
  Een te groot deel van het zorgbudget gaat niet naar de zorg zelf, maar
  de organisatie daarvan. We slopen onnodige managementlagen uit de zorg
  en investeren dit geld in de zorg zelf.
  \end{quote}
\item
  \begin{quote}
  We brengen de productie van geneesmiddelen en medische apparatuur in
  publieke handen om prijzen te beheersen én laag te houden. Daarnaast
  maken we internationale afspraken en regels om de macht van grote
  farmaceutische bedrijven in te perken. We zetten in op prijsbeheersing
  en -beperking bij ingevoerde geneesmiddelen en apparatuur. Monopolies
  op geneesmiddelen worden bestreden en de lobbykracht van patenthouders
  op de zorg wordt teruggedrongen.
  \end{quote}
\end{enumerate}

\textbf{Zorg van ons allemaal: maak de zorg toegankelijk}

\begin{enumerate}
\def\labelenumi{\arabic{enumi}.}
\item
  \begin{quote}
  Huisartsenpraktijken worden zorgcentra in de wijk waar zorg samenkomt
  en je terecht kunt met iedere zorg- of ondersteuningsvraag. We
  investeren in gespecialiseerde praktijkondersteuning. Professionals
  moeten indicaties stellen; de gemeente gaat er tussenuit. De gemeente
  houdt zich bovendien niet meer bezig met controleren, maar nog enkel
  met de toegang tot zorg.
  \end{quote}
\item
  \begin{quote}
  Continuïteit van zorg wordt beter gewaarborgd. `Eén plan, één
  regisseur' wordt de basis van hulp voor mensen met een zorg- of
  ondersteuningsvraag, onder welke zorgwet die vraag ook valt.
  \end{quote}
\item
  \begin{quote}
  Er komt meer zelfbeschikking voor mensen die zorg krijgen. Iedereen
  krijgt de vrijheid te kiezen door welke hulpverlener zij geholpen
  willen worden.
  \end{quote}
\item
  \begin{quote}
  Gedwongen zorg wordt in lijn met het VN-verdrag inzake rechten van
  personen met een handicap uitgevoerd en er wordt meer zelfbeschikking
  ingeregeld voor personen die onbegrepen gedrag vertonen.
  Vrijheidsbeperkende maatregelen in instellingen (zoals de
  separeercellen) worden verboden.
  \end{quote}
\item
  \begin{quote}
  De `harde knip' moet uit de jeugdzorg: wanneer je 18 wordt moet zorg
  niet ineens stoppen. Jeugdzorg mag pas stoppen wanneer de BIG 5 van
  bestaanszekerheid voor de jongere geregeld is.
  \end{quote}
\end{enumerate}

\begin{enumerate}
\def\labelenumi{\arabic{enumi}.}
\setcounter{enumi}{5}
\item
  \begin{quote}
  Financiering van de GGZ moet worden losgekoppeld van diagnoses. Zo
  stoppen we de doorgeslagen diagnostisering en werken we aan een GGZ
  met minder wachtlijsten, waar de menselijke maat centraal staat.
  \end{quote}
\item
  \begin{quote}
  Onze ouderen verdienen respect. De kwaliteitscriteria voor
  verzorgingstehuizen moeten aangescherpt. Verzorgingstehuizen mogen
  geen dumpplek zijn. Daarom ondersteunen we het eerdere initiatief van
  zorgbuurthuizen, die sociale functies en zorg samenbrengen.
  \end{quote}
\item
  \begin{quote}
  We omarmen het FNV-plan `Drastisch versimpelen in de jeugdzorg' en
  geven hier uitvoering aan.
  \end{quote}
\item
  \begin{quote}
  We ondersteunen laaggeletterden in het verwerken en begrijpen van
  belangrijke medische en overheidsinformatie.
  \end{quote}
\end{enumerate}

\textbf{Zorg door deskundigen}

\begin{enumerate}
\def\labelenumi{\arabic{enumi}.}
\item
  \begin{quote}
  We verlagen de werk- en regeldruk in de zorg door de administratieve
  lasten te verminderen en tijdschrijven af te schaffen. Zo kan de tijd
  van zorgmedewerkers zoveel mogelijk gaan naar mensen die zorg hard
  nodig hebben: vakmanschap komt voorop te staan. Personeelstekorten,
  lange wachtlijsten, verloop en burn-outs van medewerkers gaan we
  hiermee tegen.
  \end{quote}
\item
  \begin{quote}
  Alle zorgmedewerkers, van verplegers in de medische zorg tot
  jeugdzorgmedewerkers, moeten eerlijker worden betaald. Ook het werk
  van mantelzorgers moet erkend worden en geherwaardeerd, wat zich uit
  in een eerlijke financiële waardering voor hun werk.
  \end{quote}
\item
  \begin{quote}
  Zorginstellingen worden meer coöperatief georganiseerd, zodat
  zorgmedewerkers hun eigen beroep vormgeven, zowel op individueel-,
  organisatie- als sectorniveau.
  \end{quote}
\item
  \begin{quote}
  Op organisatieniveau moet gezorgd worden voor goede (in)formele
  mogelijkheden voor mensen met een beperking of chronische ziekte om
  vanuit een gelijkwaardige samenwerking mee te denken en mee te
  beslissen.
  \end{quote}
\item
  \begin{quote}
  De inzet van ervaringskennis wordt een vast onderdeel van het
  curriculum in alle opleidingen voor hulpverleners.
  \end{quote}
\item
  \begin{quote}
  Om de expertise van de ervaringsdeskundigen te erkennen en
  gelijkwaardig te behandelen aan die van andere beroepsgroepen, wordt
  de functie `ervaringswerker' in de cao van verschillende takken van de
  zorg opgenomen. Voor de betaling van ervaringsdeskundige vrijwilligers
  worden landelijke richtlijnen opgesteld, zodat deze unieke ervaring en
  expertise gewaardeerd worden.
  \end{quote}
\item
  \begin{quote}
  Mensen met een hulpvraag hebben recht op passende zorg en
  ondersteuning. Het moet voor mensen eenvoudig en toegankelijk zijn om
  hun klachten of bezwaren in te dienen en hun recht te halen wanneer
  dat nodig is. Iedereen heeft toegang tot onafhankelijke
  cliëntondersteuning.
  \end{quote}
\end{enumerate}

\textbf{Zorg voor iedereen}

\begin{enumerate}
\def\labelenumi{\arabic{enumi}.}
\item
  \begin{quote}
  Er moet meer toezicht komen op de naleving van het VN-verdrag inzake
  rechten van personen met een handicap in het sociaal domein door het
  opstellen van een landelijk normenkader.
  \end{quote}
\item
  \begin{quote}
  De inzet van tolken en gebarentolken moet niet uit het budget van de
  zorgvrager worden betaald, maar standaard vergoed worden door de
  overheid.
  \end{quote}
\item
  \begin{quote}
  In opleidingen van GGZ-hulpverleners krijgt neurodiversiteit een plek,
  zodat er meer gehandeld wordt vanuit variatie in het menselijk brein.
  Therapievormen die de natuurlijke staat van een neurodivergent persoon
  onderdrukken, en daardoor schadelijk kunnen zijn, mogen niet toegepast
  worden.
  \end{quote}
\item
  \begin{quote}
  Uit
  \href{http://www.kis.nl/rol-etniciteit-basiszorg-jeugd-ggz}{\underline{onderzoek}}
  blijkt dat mentale gezondheidsproblematiek bij mensen van kleur minder
  goed wordt herkend. In de opleiding van artsen, hulpverleners, thuis-
  en ouderenzorg en medisch personeel moet meer aandacht zijn voor
  bestrijding van racistische stereotypen. Cultuursensitieve zorg krijgt
  hierin een belangrijke plek.~
  \end{quote}
\item
  \begin{quote}
  We zorgen voor een gelijkwaardige toegang tot zorg voor iedereen,
  ongeacht verblijfsstatus.
  \end{quote}
\item
  \begin{quote}
  Niet de witte cisgender man moet centraal staan bij medisch onderzoek.
  Er wordt meer geïnvesteerd in onderzoek naar ziektebeelden,
  geneesmiddelen en behandelmethoden bij mensen van kleur en vrouwen
  (van kleur).
  \end{quote}
\item
  \begin{quote}
  Chronische stress en trauma veroorzaakt door racisme krijgen meer
  erkenning als oorzaken van gezondheidsproblemen, zoals depressies en
  hart- en vaatziekten.~
  \end{quote}
\item
  \begin{quote}
  Suïcidaliteit en thuisloosheidonder neurodivergente en
  LHBTQI+-jongeren is schrikbarend hoog. Er komt meer aandacht voor
  passende zorg, goede begeleiding en suïcidepreventie bij LHBTQI+
  personen. Verspreid over het land worden \emph{safehouses} ingericht,
  waar zij terecht kunnen in noodsituaties.
  \end{quote}
\item
  \begin{quote}
  De \underline{zorg voor trans personen (5)} wordt gezien als reguliere
  zorg en opgenomen in het basis- curriculum van medische opleidingen.
  Bovendien wordt het aanbod van transzorg vergroot om wachtlijsten te
  verkorten: expertise wordt gedeeld tussen de vijf grootste
  ziekenhuizen. Alle behandelingen en operaties worden vergoed. Trans
  personen hoeven daarnaast geen diagnose meer te hebben om zorg te
  krijgen. En huisartsen kunnen hormoonbehandelingen voorschrijven.
  \end{quote}
\item
  \begin{quote}
  Er komen trainingen om medewerkers in de zorg bewust te maken van hun
  (onbewuste) vooroordelen, houding en verwachtingspatronen en er worden
  protocollen ontwikkeld om racisme, seksisme en discriminatie op grond
  van seksuele geaardheid te bestrijden.
  \end{quote}
\end{enumerate}

\textbf{WONEN EN BOUWEN}

BIJ1 wil dat iedereen in Nederland prettig, betaalbaar en veilig kan
wonen. Wij beschouwen wonen dan ook als een recht. Jongeren en jonge
gezinnen moeten binnen afzienbare tijd een woning kunnen vinden en ook
ouderen en mensen met een beperking moeten een geschikte woning hebben.
Voor alternatieve woonvormen en huishoudens met een niet-standaard
samenstelling moet er beter beleid komen. Kwetsbare groepen voor wie nu
vaak geen plek is, zoals bijvoorbeeld alleenstaande moeders en
uitstromers vanuit de jeugdzorg, moeten ook veilig en prettig kunnen
wonen. Dak- en thuisloze mensen gaan we helpen op een manier die bij hen
past.

\textbf{Rechtvaardig wonen}

We willen ook dat er rechtvaardigheid komt op het gebied van wonen. Het
kapitalistische systeem werkt ongelijkheid en onrechtvaardigheid in de
hand, ook op het gebied van wonen. Deze discriminatie gaan we beter
bestrijden. Wonen is voor veel mensen een enorme kostenpost en daar
wordt door anderen grof geld aan verdiend. Wij vinden dat wonen niet
over winst mag gaan. De waarde van een woning moet worden bepaald door
de daadwerkelijke woonkwaliteit en niet door de marktwaarde. We
bestrijden leegstand van woningen, speculatie met leegstaande panden en
speculatie met bouwgrond. We stellen ook betere regels op voor huurders
en gaan deze handhaven. Huurprijzen moeten te betalen zijn en de rechten
van huurders moeten worden beschermd. Huurders krijgen ook meer
zeggenschap over het verbeteren van woningen.

\textbf{Bouwen en duurzaamheid}

De energietransitie vraagt om een andere visie op het gebied van wonen
en bouwen. Zo gaan we beleid richten op het renoveren en verduurzamen
van woningen. Met oog op de mogelijke impact van klimaatverandering op
zowel korte als lange termijn, wordt klimaatadaptief bouwen de norm. Als
sloop en nieuwbouw niet te voorkomen zijn, moet het aantal sociale
huurwoningen minimaal hetzelfde blijven. Kwesties als duurzaamheid en
toegankelijkheid gaan een belangrijke rol spelen bij het afgeven van
bouwvergunningen en huurwoningen met slechte isolatie en lage
energielabels worden aangepast, zodat deze in 2030 niet meer bestaan of
verhuurd mogen worden.

BIJ1 stelt de volgende oplossingen voor om te zorgen dat iedereen
prettig, betaalbaar en veilig kan wonen.

\textbf{Een plek voor iedereen}

\begin{enumerate}
\def\labelenumi{\arabic{enumi}.}
\item
  \begin{quote}
  Het recht op woonruimte wordt wettelijk vastgelegd.
  \end{quote}
\item
  \begin{quote}
  De kostendelersnorm wordt afgeschaft.
  \end{quote}
\item
  \begin{quote}
  De verhuurderheffing wordt afgeschaft.
  \end{quote}
\item
  \begin{quote}
  De hypotheekrenteaftrek wordt afgeschaft.
  \end{quote}
\item
  \begin{quote}
  Makelaars, bemiddelaars of verhuurders die discrimineren, verliezen
  het recht om deze functies nog langer te beoefenen en worden
  strafrechtelijk vervolgd.
  \end{quote}
\item
  \begin{quote}
  De jaarlijkse huurverhoging wordt maximaal inflatievolgend.
  \end{quote}
\item
  \begin{quote}
  Het kraakverbod wordt opgeheven.
  \end{quote}
\item
  \begin{quote}
  Het bezitten maar onbenut laten van woningen en panden wordt verboden.
  Zo voeren we een woonplicht in en gaan we leegstand beboeten.
  \end{quote}
\item
  \begin{quote}
  We heffen een extra belasting op het bezit van woningen met een waarde
  boven de 500.000 euro.
  \end{quote}
\item
  \begin{quote}
  We zetten in op de bouw van meer sociale woningen, onder andere door
  grondbeleid in de vorm van actieve verwerving van private grond.
  \end{quote}
\item
  \begin{quote}
  Met actieve handhaving bestrijden we leegstand van woningen,
  speculatie met leegstaande panden en speculatie met bouwgrond.
  \end{quote}
\item
  \begin{quote}
  Bij het afgeven van bouwvergunningen worden toegankelijkheid,
  duurzaamheid en een klimaatparagraaf bepalende factoren.
  \end{quote}
\item
  \begin{quote}
  Zoveel mogelijk bestaande woningen worden geschikt gemaakt voor mensen
  met een beperking. Het VN Verdrag Handicap is hierin het uitgangspunt.
  \end{quote}
\item
  \begin{quote}
  (Lokale) overheden geven bij het beschikbaar stellen van grond
  voorrang aan sociale huurwoningen, in plaats van aan de hoogste
  bieder.
  \end{quote}
\end{enumerate}

\textbf{Rechtvaardig huren, bouwen en renoveren}

\begin{enumerate}
\def\labelenumi{\arabic{enumi}.}
\item
  \begin{quote}
  Woningcorporaties worden genationaliseerd. We vormen regionale
  wooncoöperaties door het hele land, die democratisch zijn
  georganiseerd en zich ook daadwerkelijk richten op het socialer maken
  van sociale huurwoningen en wijken.
  \end{quote}
\item
  \begin{quote}
  Het huidige puntenstelsel voor huurwoningen wordt vervangen door een
  nieuw systeem gebaseerd op de daadwerkelijke woonkwaliteit van een
  woning.
  \end{quote}
\item
  \begin{quote}
  Huurcontracten voor onbepaalde tijd worden weer de norm. Tijdelijke
  huurcontracten zijn alleen nog in uitzonderlijke situaties toegestaan.
  \end{quote}
\item
  \begin{quote}
  We voeren een verhuurvergunning met kwalificatie-eisen in om huurders
  te beschermen tegen kwaadwillende verhuurders (huisjesmelkers).
  \end{quote}
\item
  \begin{quote}
  Huurders, individueel en als collectief, krijgen meer invloed op
  woningverbeteringen via initiatiefrechten.
  \end{quote}
\item
  \begin{quote}
  De sloop van woningen die nog te renoveren zijn wordt zoveel mogelijk
  voorkomen. Als sloop toch de enige optie blijkt, moet bij de nieuwbouw
  het aantal woningen in het sociale segment minimaal hetzelfde zijn als
  voorheen.
  \end{quote}
\item
  \begin{quote}
  Er wordt geen onderscheid gemaakt bij het opknappen, opfleuren en
  onderhouden van wijken. Gebieden met sociale huurwoningen hebben net
  zo veel recht op recreatie, bereikbaarheid en groen, zonder dat dit
  ten koste gaat van de status van sociale huur.
  \end{quote}
\item
  \begin{quote}
  Er komt een schadecompensatieregeling voor woningen met
  aardbevingsschade.
  \end{quote}
\item
  \begin{quote}
  De verhuurder wordt verantwoordelijk voor het opknappen van
  energie-onzuinige woningen. Het verhuren van woningen met labels G, F
  en E wordt niet langer toegestaan. Per 2030 dienen alle huurwoningen
  naar minimaal energielabel A of B te zijn gerenoveerd en zonder
  gebruik van aardgas verwarmd te worden.
  \end{quote}
\item
  \begin{quote}
  Er worden fondsen beschikbaar gesteld voor de verduurzaming van
  sociale woningen. Ook investeren we flink in het toegankelijk maken
  van sociale woningen.
  \end{quote}
\item
  \begin{quote}
  Er komt een nationaal plan om ook bij een veranderend klimaat voor
  iedereen veilige en toegankelijke huisvesting te garanderen. Hierbij
  wordt rekening gehouden met onder andere overstromingsgevaar, extreme
  hitte, kou en droogte. Ook wordt er speciale aandacht besteed aan de
  BES-Eilanden.
  \end{quote}
\item
  \begin{quote}
  Er wordt onderzoek gedaan naar de gevolgen en gevaren van extremer
  wordende omstandigheden zoals droogte, verlaagde grondwaterstanden en
  bodeminklinking voor bestaande bebouwing, natuur en dier. Er worden
  passende maatregelen getroffen om dit tegen te gaan.
  \end{quote}
\item
  \begin{quote}
  Huurders krijgen toegang tot een platform van onafhankelijke
  deskundigen in het geval van voortdurende schimmel- of
  vochtproblematiek.
  \end{quote}
\end{enumerate}

\textbf{Kwetsbare groepen en alternatieve (woon)groepen}

\begin{enumerate}
\def\labelenumi{\arabic{enumi}.}
\item
  \begin{quote}
  Er worden per direct voldoende gratis opvangplekken voor dak- en
  thuislozen gerealiseerd, ook voor zelfredzame dak- en thuislozen.
  Hierbij is individuele privacy een topprioriteit.
  \end{quote}
\item
  \begin{quote}
  Bestaande projecten zoals \emph{Housing First,} waar dak- en
  thuislozen worden voorzien van een woning (indien nodig met
  individuele hulp), worden verder uitgebreid en tot norm gemaakt.
  \end{quote}
\item
  \begin{quote}
  Directe huisuitzettingen op basis van betalingsachterstanden worden
  verboden.
  \end{quote}
\item
  \begin{quote}
  Het recht van Roma en Sinti om in woonwagens te wonen wordt
  gerespecteerd en er wordt actief en ruimhartig meegewerkt aan het
  maken en faciliteren van woonlocaties.
  \end{quote}
\item
  \begin{quote}
  Voor de meest kwetsbaren komt er een extra categorie binnen de sociale
  huursector met huren die niet hoger zijn dan 350 euro per maand.
  \end{quote}
\item
  \begin{quote}
  Er komen meer mogelijkheden voor (samen)wonen zoals \emph{tiny houses}
  en woongroepen.
  \end{quote}
\item
  \begin{quote}
  Bij transformatie van gebouwen moeten gemeenten ruimte bieden aan
  bewonersinitiatieven of innovatieve woonvormen.
  \end{quote}
\end{enumerate}

\textbf{ONDERWIJS EN WETENSCHAP}

BIJ1 wil gelijke kansen voor alle leerlingen in het onderwijs. Het moet
voor de kansen van leerlingen niet uitmaken wat hun huidskleur is, wat
hun gender- en/of seksuele identiteit is, of zij een beperking hebben,
of wat hun geloof is. Ook het inkomen of de achtergrond van hun ouders
mag geen rol spelen. Als je op latere leeftijd een opleiding wilt doen,
moet dat mogelijk zijn, ongeacht je eigen inkomen. Verder moet het
onderwijs voor leerlingen met een beperking volledig toegankelijk zijn.
Dat geldt ook voor het reguliere onderwijs, als leerlingen met een
beperking daar naartoe willen.

Om gelijke kansen te creëren voor alle leerlingen willen we niet
bezuinigen, maar juist investeren. We gaan \underline{segregatie van
leerlingen (6)} en discriminatie actief bestrijden. Door te investeren
zorgen we voor kwaliteitsonderwijs en sterke, diverse en democratische
onderwijsinstellingen met goede instroming en doorstroming. Leraren en
wetenschappers gaan de (financiële) waardering krijgen die zij
verdienen.

\textbf{Kwaliteitsonderwijs dat past}\\
Kwaliteitsonderwijs dat past betekent onderwijs waarin leerlingen zich
kunnen ontplooien en kunnen groeien als mens. Elke soort onderwijs moet
kwaliteitsonderwijs bieden, van voorschool tot universiteit. Het
\underline{lesprogramma moet flexibel zijn (7)}, zodat alle leerlingen
onderwijs krijgen dat bij hen past en waar zij talenten en vaardigheden
zoals creativiteit, inlevingsvermogen en samenwerking kunnen
ontwikkelen. Ook de diverse perspectieven en ervaringen van identiteit,
marginalisatie en geschiedenis in Nederland moeten onderdeel zijn van
het lesprogramma. BIJ1 vindt het belangrijk om kinderen en jongeren te
betrekken bij hedendaagse maatschappelijke vraagstukken en burgerschap.
Maatschappijleer wordt een verplicht vak gedurende het hele VO. Ook
buiten het mbo wordt burgerschapsonderwijs verplicht gesteld.
\underline{Digitale middelen (8)} worden ingezet voor (verbetering van)
het lesprogramma en om kansenongelijkheid tegen te gaan. En om
laaggeletterdheid tegen te gaan - dat vaak op een jonge leeftijd
ontstaat - moet de toegang tot openbare bibliotheken voor leerlingen
gestimuleerd worden.

\textbf{Sterke onderwijsinstellingen en betere in- en doorstroming}

De onderwijsinstellingen willen we versterken om ze meer mogelijkheden
te geven voor het begeleiden van leerlingen en studenten. Het
\underline{mbo (9)} moet meer waardering krijgen. In het
\underline{hoger onderwijs (hbo en universiteiten) moet meer diversiteit
en democratie (10)} komen. We willen samenwerking tussen
onderwijsinstellingen stimuleren in plaats van concurrentie.
Onderwijsinstellingen moeten ook zorgen dat leerlingen of studenten op
het juiste niveau kunnen instromen. Dat niveau mag niet op te jonge
leeftijd bepaald worden en onderadvisering moet worden tegengegaan. Als
een leerling of student wil overstappen of doorstromen naar een andere
soort onderwijs, moet dat soepel verlopen.

\textbf{Waardering voor leraren en wetenschappers}

Er moet weer waardering komen voor wie in het onderwijs werkt. Het
\underline{lerarentekort (11)} willen we oplossen. Wetenschappers moeten
goed onderzoek kunnen doen en leraren moeten goed kunnen lesgeven. Hun
werkdruk moet omlaag en hun salaris moet omhoog.

BIJ1 heeft volgende oplossingen voor inclusief, toegankelijk en
kwalitatief hoogwaardig onderwijs.

\textbf{Gelijke kansen voor alle leerlingen}

\begin{enumerate}
\def\labelenumi{\arabic{enumi}.}
\item
  \begin{quote}
  We gaan op zoek naar alternatieven voor de momenten waarop een
  leerling geselecteerd en getoetst wordt als het gaat om het niveau van
  het voortgezet onderwijs. Scholen met brede brugklassen worden
  gestimuleerd - op voorwaarde dat er binnen die brugklassen genoeg
  mogelijkheden zijn om verschillende manieren van leren te
  ondersteunen.
  \end{quote}
\item
  \begin{quote}
  Advies voor vervolgonderwijs wordt gebaseerd op de overeenkomst tussen
  3 partijen: leraar, ouder(s)/verzorger(s) en leerling.
  \end{quote}
\item
  \begin{quote}
  Alle leraren in Nederland krijgen trainingen over het geven van een
  schooladvies op de juiste manier en het creëren van bewustwording van
  onderadvisering.
  \end{quote}
\item
  \begin{quote}
  Er komen meer middelen voor scholen om kennis te delen met andere
  scholen over het creëren van gelijke kansen.
  \end{quote}
\item
  \begin{quote}
  Scholen moeten volledig toegankelijk worden voor leerlingen met een
  beperking. Er worden initiatieven ingevoerd door het hele land waarbij
  leerlingen met een beperking deel uitmaken van het regulier onderwijs.
  Er wordt extra budget vrijgemaakt voor leraren voor ondersteuning van
  de leerlingen met een beperking.
  \end{quote}
\item
  \begin{quote}
  We zorgen voor meer toegankelijke mogelijkheden om mensen die
  laaggeletterd bij te staan en ondersteunen waar nodig.
  \end{quote}
\item
  \begin{quote}
  Brede scholengemeenschappen met verschillende schoolsoorten (vmbo,
  havo, vwo) en initiatieven om deze te starten worden gestimuleerd. Op
  de lange termijn werken we toe naar het afschaffen van een individuele
  indeling en collectieve segregatie op niveau.
  \end{quote}
\item
  \begin{quote}
  Diploma's in het voortgezet onderwijs moeten gestapeld kunnen worden
  zonder aanvullende eisen.
  \end{quote}
\item
  \begin{quote}
  De voorschool wordt gratis voor alle kinderen.
  \end{quote}
\item
  \begin{quote}
  De vrijwillige ouderbijdrage in het basisonderwijs en voortgezet
  onderwijs wordt afgeschaft.
  \end{quote}
\item
  \begin{quote}
  Toegang tot bijles moet niet afhankelijk zijn van de economische
  positie van de ouders. Er zou gratis bijles aangeboden kunnen moeten
  worden aan kinderen wiens ouders beneden een bepaalde inkomensgrens
  zitten.
  \end{quote}
\item
  \begin{quote}
  Bibliotheken krijgen meer ondersteuning bij het aanpakken van
  laaggeletterdheid onder kinderen en jongeren. Er moet makkelijke
  toegang zijn voor ieder burger tot een openbare bibliotheek.
  \end{quote}
\item
  \begin{quote}
  Bedrijven die discrimineren in hun selectie voor stagiairs worden
  beboet en uitgesloten van overheidsopdrachten en subsidies.
  \end{quote}
\item
  \begin{quote}
  De manier van inschrijven wordt zo ingericht, dat scholen voor
  iedereen toegankelijk zijn. Er komt een einde aan het postcodebeleid
  en zeer vroegtijdige inschrijving als selectiemiddel door scholen.
  \end{quote}
\end{enumerate}

\textbf{Een divers en inclusief aanbod op school}

\begin{enumerate}
\def\labelenumi{\arabic{enumi}.}
\item
  \begin{quote}
  Scholen krijgen een wettelijke taak om interculturele kennis tussen
  leerlingen te bevorderen en te organiseren.
  \end{quote}
\item
  \begin{quote}
  De overheid stimuleert een divers perspectief in het curriculum en
  lesmateriaal dat vrij is van schadelijke stereotyperingen en
  eurocentrisme.
  \end{quote}
\item
  \begin{quote}
  De koloniale geschiedenis en het migratieverleden van Nederland
  krijgen een centrale plek in het lesprogramma, met daarnaast de
  achtergrond en politiek-economische context van vluchtelingen en
  (arbeids)migranten. Ook komt er meer aandacht voor de gevolgen van
  deze geschiedenis voor de volgende generaties, die in Nederland
  geboren worden.
  \end{quote}
\item
  \begin{quote}
  Er komen trainingen voor leraren over het historisch besef van de
  koloniale geschiedenis en het migratieverleden van Nederlanders.
  \end{quote}
\item
  \begin{quote}
  Er komt ruimschoots en kwalitatief hoogstaande aandacht in het
  curriculum voor seksualiteit, consent en diversiteit op het gebied van
  gender en seksuele geaardheid.
  \end{quote}
\item
  \begin{quote}
  Er komen trainingen om leraren bewust te maken van hun (onbewuste)
  vooroordelen, houding en verwachtingspatronen en er worden protocollen
  ontwikkeld om racisme, seksisme en discriminatie op grond van seksuele
  geaardheid te bestrijden.
  \end{quote}
\item
  \begin{quote}
  Er komt een garantie dat alle leerlingen toegang hebben tot laptops en
  WiFi \emph{hotspots} voor thuis. Elke school krijgt hiervoor een
  aangewezen budget.
  \end{quote}
\item
  \begin{quote}
  De overheid gaat scholen ontmoedigen samen te werken met platforms van
  bedrijven die \emph{surveillance} als bedrijfsmodel hebben, om de
  privacy van leerlingen te beschermen.
  \end{quote}
\end{enumerate}

\textbf{Betere werkomstandigheden voor leraren}

\begin{enumerate}
\def\labelenumi{\arabic{enumi}.}
\item
  \begin{quote}
  De arbeidsvoorwaarden van leraren worden verbeterd, zowel met
  betrekking tot het loon als tot de werkdruk.
  \end{quote}
\item
  \begin{quote}
  De loonkloof tussen het basisonderwijs en voortgezet onderwijs wordt
  gedicht, zodat alle leraren hetzelfde gaan verdienen.
  \end{quote}
\item
  \begin{quote}
  De klassen worden kleiner en de administratieve lasten worden
  verlicht, om de werkdruk te verminderen en om meer voorbereidingstijd
  te creëren.
  \end{quote}
\item
  \begin{quote}
  Pabo's moeten meer aandacht besteden aan de werkdruk in het
  onderwijsveld (veroorzaakt door samenleving, schoolbestuur en ouders)
  en hoe daarmee om moet worden gegaan.
  \end{quote}
\end{enumerate}

\textbf{Toegankelijk beroeps- en hoger onderwijs en verbeterde
doorstroom}

\begin{enumerate}
\def\labelenumi{\arabic{enumi}.}
\item
  \begin{quote}
  Het collegegeld wordt afgeschaft: middelbaar beroepsonderwijs en het
  hoger onderwijs worden gratis. De basisbeurs wordt opnieuw ingevoerd
  en wordt inkomensafhankelijk, gebaseerd op het inkomen van de
  ouders/verzorgers en dat van de student zelf.
  \end{quote}
\item
  \begin{quote}
  De schakelprogramma's voor studenten tussen het mbo en hbo worden
  verbeterd.
  \end{quote}
\item
  \begin{quote}
  We werken aan toegankelijk onderwijs\hspace{0pt} door het recht op
  thuisstudie vast te leggen. Ook komt er meer (onafhankelijke)
  ondersteuning zodat mensen met een beperking beter naar eigen wensen
  kunnen studeren, bijvoorbeeld door het faciliteren van online
  colleges.
  \end{quote}
\item
  \begin{quote}
  Het mbo wordt versterkt en verbreed: leraren krijgen meer ruimte om
  door te groeien binnen hun functie, de overheid gaat samenwerken met
  het bedrijfsleven om meer stageplaatsen te bewerkstelligen, en er
  komen meer kleinschalige vakscholen.
  \end{quote}
\item
  \begin{quote}
  De toegang tot het beroepsonderwijs en het wetenschappelijk onderwijs
  wordt vergemakkelijkt, bijvoorbeeld door drempels voor werkende mensen
  weg te nemen. Opties om de opleiding in deeltijd te kunnen volgen
  worden verplicht, leeftijdsgrenzen worden verboden.
  \end{quote}
\item
\item
  \begin{quote}
  Het geldbedrag dat onderwijsinstellingen ontvangen hangt niet langer
  af van het percentage afgestudeerden.
  \end{quote}
\item
  \begin{quote}
  Op elke vorm van beroeps- en hoger onderwijs komt er een
  schoolpsycholoog waar studenten die met mentale problematiek te maken
  hebben terecht kunnen.
  \end{quote}
\end{enumerate}

\textbf{Meer waardering, democratie en diversiteit in het hoger
onderwijs}

\begin{enumerate}
\def\labelenumi{\arabic{enumi}.}
\item
  \begin{quote}
  De doelmatigheidskorting op het wetenschappelijk onderwijs wordt
  afgeschaft.
  \end{quote}
\item
  \begin{quote}
  De overheidsbijdrage aan het wetenschappelijk onderwijs krijgt een
  extra investering van 1.15 miljard euro per jaar.
  \end{quote}
\item
  \begin{quote}
  Er is een oligopolie op publicaties in de academische wereld, waarbij
  drie grote publicatiebedrijven een groot deel van tijdschriften en
  publicatiekanalen voor academische publicaties bezitten. Dit kost de
  universiteitsbibliotheken de helft van hun budget. De overheid gaat
  deze oligopolie tegen door te investeren in \emph{open access}
  publiceren, door middel van digitale infrastructuur en internationale
  lobby.
  \end{quote}
\item
  \begin{quote}
  De overheid gaat zich inzetten om de invloed van de fossiele industrie
  op het onderwijs te stoppen, met name in het curriculum.
  \end{quote}
\item
  \begin{quote}
  Elke instelling krijgt een diversiteitscommissie en er komen
  diversiteitsquota in alle onderwijs- en bestuurslagen in het hoger
  onderwijs..
  \end{quote}
\item
  \begin{quote}
  Besturen en raden van toezicht worden democratisch verkozen.
  \end{quote}
\end{enumerate}

\textbf{KUNST, CULTUUR EN MEDIA}

BIJ1 wil vechten voor een bloeiende kunst-, cultuur- en mediasector die
de Nederlandse samenleving weerspiegelt en die toegankelijk is voor
iedereen. Een sector die ons scherp houdt, ons vermaakt en onderwijst,
en die haar verantwoordelijkheid neemt voor haar aandeel in onder andere
de koloniale geschiedenis en het slavernijverleden. BIJ1 wil dat makers
gewoon weer kunnen maken, en schrijvers gewoon weer kunnen schrijven.

\textbf{Ruimte voor vernieuwing en toegankelijkheid}

Kunst, cultuur en de creatieve sector moeten op hoog niveau kunnen
werken, nieuwe ideeën ontwikkelen, inspireren en provoceren, en zo voor
de hele samenleving betekenis hebben. Kunstenaars en creatieven moeten
van hun werk kunnen leven. Onafhankelijke en diverse media zijn
onmisbaar voor een vrije en democratische samenleving. Helaas wordt de
ruimte voor kritische en creatieve denkers, schrijvers en makers steeds
een beetje kleiner gemaakt. Subsidies worden steeds kleiner en
schaarser, terwijl de verplichtingen van makers tegenover
subsidieverstrekkers groeien. Kleine instellingen wordt zo de nek
omgedraaid.

Juist nu community art, spoken word en internet art meer omarmd worden
door grote instellingen, is er minder geld voor deze vernieuwende
vormen. Uitgevers durven het bijvoorbeeld steeds minder aan om boeken
uit te geven die weinig kans hebben om \emph{bestsellers} te worden,
omdat die niet genoeg geld opbrengen. In Nederland kan slechts een
honderdtal schrijvers van het schrijven leven. Kranten en omroepen
worden steeds meer afgerekend op het aantal abonnees en kijkcijfers. Dat
betekent minder kansen voor controversiële en vernieuwende creaties die
niet direct het grote publiek trekken. Culturele vrijplaatsen staan
onder druk en makers worden gedwongen om commercieel te worden.
Toegangsprijzen voor voorstellingen en tentoonstellingen worden ook
alleen maar hoger: zo wordt kunst en cultuur nog enkel voor de elite.

\textbf{Ondersteuning van de makers}

Ondertussen wordt de (verborgen) armoede onder makers alleen maar
groter. Hoewel de Fair Practice Code (die zich richt op het verbeteren
van de inkomsten van makers) een goed instrument is om dit tegen te
gaan, wordt de naleving hiervan niet gecontroleerd en hebben
instellingen hier ook de middelen niet voor. Zo houdt het systeem van
armoede, regels en beperkingen zichzelf in stand en wordt de kunst- en
cultuursector verder uitgekleed.

\textbf{Representatie van de samenleving}

Culturele instellingen en omroepen zijn nog steeds geen afspiegeling van
de samenleving. Vooral witte mensen beslissen wat er in musea komt te
hangen of welke organisaties subsidie krijgen. De \emph{talking heads}
op televisie en radio zijn nog te vaak witte mannen. Redacties
vertegenwoordigen niet de diverse perspectieven binnen onze
maatschappij. Daarbij is het voor makers van kleur vaak moeilijker om
toegang te krijgen tot fondsen en culturele instellingen. Het gevolg
hiervan is dat door de overheid gesubsidieerde kunst en cultuur maar een
beperkt deel van de Nederlandse samenleving aanspreekt. De Code
Diversiteit en Inclusie is opgesteld om hier een einde aan te maken,
maar de uitvoering ervan blijft achter.

\textbf{Erken de makers, erken de geschiedenis}

Veel van onze museumcollecties zijn gebouwd op de Nederlandse koloniale
geschiedenis en het Nederlandse slavernijverleden. In Nederlandse musea
kun je nog steeds kunst bekijken die is gestolen uit andere landen of
die is aangekocht met geld dat is verdiend met slavernij en uitbuiting.
Hier moet een eind aan komen. Onze aanpak voor een representatieve en
gezonde kunst-, cultuur- en mediasector luidt als volgt.

\textbf{Voor de makers}

\begin{enumerate}
\def\labelenumi{\arabic{enumi}.}
\item
  \begin{quote}
  We stellen meer fondsen beschikbaar op het gebied van kunst, cultuur
  en media waarbij ook een grotere diversiteit aan instellingen
  aanspraak kan maken op deze fondsen. De bezuinigingen van Rutte-I
  worden ongedaan gemaakt en er wordt structureel meer geïnvesteerd in
  kunst, cultuur en media.
  \end{quote}
\item
  \begin{quote}
  Er wordt meer geïnvesteerd in makers bij de NPO. Deze investeringen
  worden gekoppeld aan het verplicht naleven van de \emph{Fair Practice
  Code} en de Code Culturele Diversiteit. Daarnaast worden de scheve
  financiële constructies waarmee productiemaatschappijen grote winst
  maken ongedaan gemaakt.
  \end{quote}
\item
  \begin{quote}
  De Wet Werk \& Inkomen Kunstenaars wordt weer in het leven geroepen en
  er wordt structureel geprobeerd om de opheffing van armoede in de
  kunst, cultuur en mediasector te verwezenlijken.
  \end{quote}
\item
  \begin{quote}
  Er komt strikt toezicht op de naleving van de \emph{Fair Practice
  Code,} zodat makers in hun onderhoud kunnen voorzien. Er komen
  voldoende extra middelen voor instellingen om deze code in te voeren
  en bindend te maken, zodat er gevolgen worden verbonden aan het niet
  naleven van de \emph{Fair Practice Code}.
  \end{quote}
\item
  \begin{quote}
  Er komen minder verplichtingen voor het genereren van eigen inkomsten,
  zodat makers zich kunnen toeleggen op hun werk en de toegangsprijzen
  omlaag kunnen.
  \end{quote}
\item
  \begin{quote}
  We zorgen dat alle roofkunst door Nederlandse culturele instellingen
  wordt teruggegeven en helpen het mogelijk te maken de kunst aldaar te
  conserveren en exposeren.
  \end{quote}
\item
  \begin{quote}
  (Culturele) vrijplaatsen zijn unieke plekken in onze samenleving met
  een grote sociale en culturele waarde. Bestaande vrijplaatsen worden
  erkend en behouden. Nieuwe vrijplaatsen worden praktisch en legaal
  ondersteund.
  \end{quote}
\end{enumerate}

\textbf{Diversiteit en representatie}

\begin{enumerate}
\def\labelenumi{\arabic{enumi}.}
\item
  \begin{quote}
  Er komt strikt toezicht op de naleving van de Code Diversiteit en
  Inclusie zodat de vertegenwoordiging van verschillende groepen uit de
  maatschappij versterkt wordt binnen de cultuursector. Dit doen we met
  voldoende extra middelen voor instellingen om deze code in te voeren
  en bindend te maken, en verbinden gevolgen aan het niet naleven van de
  Code.
  \end{quote}
\item
  \begin{quote}
  Er komt een representatief en inclusief wervingsbeleid voor
  besluitnemende functies binnen media- en cultuurfondsen, zodat
  besluitmakers binnen deze fondsen een betere afspiegeling van de
  samenleving vormen.
  \end{quote}
\item
  \begin{quote}
  Aandacht van culturele en media-instellingen voor Nederlandse
  koloniale geschiedenis en slavernijverleden, maar dan vanuit het
  perspectief van de voormalige koloniën, wordt verplicht.
  \end{quote}
\item
  \begin{quote}
  We willen een publieke omroep die onafhankelijk blijft en niet door de
  overheid of door de commercie wordt beïnvloed.
  \end{quote}
\item
  \begin{quote}
  We zorgen voor een goede bewaking van de representativiteit van media,
  waarbij regionale media, lokale media en doelgroepmedia worden
  beschermd.
  \end{quote}
\item
  \begin{quote}
  Er komt een toezichthouder op diversiteit binnen de omroepen, die
  zowel kijkt naar representatie binnen de organisaties als diversiteit
  en inclusie in media-uitingen.
  \end{quote}
\end{enumerate}

\textbf{Toegankelijkheid}

\begin{enumerate}
\def\labelenumi{\arabic{enumi}.}
\item
  \begin{quote}
  De toegang tot musea en culturele instellingen wordt gratis, zodat
  iedereen van kunst en cultuur kan genieten, ongeacht je inkomen.
  \end{quote}
\item
  \begin{quote}
  Er komt een cultuurbudget voor elk kind dat zich in Nederland bevindt.
  \end{quote}
\item
  \begin{quote}
  We sturen aan op gratis toegankelijke, online programma's van de
  publieke omroep.
  \end{quote}
\item
  \begin{quote}
  Kunst- en cultuureducatie wordt een verplicht onderdeel van zowel het
  basisonderwijs als het voortgezet onderwijs en het mbo. Hierbij wordt
  samenwerking gezocht tussen onderwijs, de culturele omgeving van de
  school en de kunst- en cultuursector. Ook participatie in de vrije
  tijd moet toegankelijker gemaakt worden voor ieder kind.
  \end{quote}
\end{enumerate}

\textbf{TOEGANKELIJKHEID}

In totaal zijn er in Nederland meer dan 3 miljoen mensen met een
beperking. Dit kunnen verschillende vormen zijn: denk aan auditieve
beperkingen (bijvoorbeeld doofheid of slechthorendheid), visuele
beperkingen (bijvoorbeeld blindheid of slechtziendheid), chronische
ziekten (bijvoorbeeld diabetes), psychische kwetsbaarheid (bijvoorbeeld
depressie of angststoornissen) en lichamelijke beperkingen (bijvoorbeeld
verlamming).

BIJ1 staat voor een inclusieve samenleving, waarin mensen met een
beperking structureel en volledig deel uitmaken van onze maatschappij.
Helaas is dit nu in Nederland nog niet het geval. De Nederlandse
samenleving is nog te vaak ingericht voor mensen zonder beperking.
Daardoor moeten mensen met een beperking zich aanpassen aan deze
samenleving, terwijl dit niet altijd mogelijk is. Iedereen is onderdeel
van de samenleving en de samenleving moet dus aangepast worden, zodat
die \underline{toegankelijk is voor iedereen (12)}.

\textbf{Inclusiviteit} \textbf{in plaats van segregatie}\\
Segregatie speelt een grote rol bij het benadelen van mensen met een
beperking in Nederland. Mensen met een beperking worden zo buiten de
samenleving gehouden. Het VN-verdrag inzake rechten van personen met een
handicap onderschrijft dit. Er zijn speciale woonvormen, speciale
scholen, speciaal vervoer en speciaal werk, waardoor de reguliere
samenleving niet genoodzaakt is zich aan te passen voor mensen met een
beperking. Regulier openbaar vervoer, reguliere woonvormen, scholen en
andere voorzieningen zijn op deze manier niet toegankelijk. Wij willen
dat mensen met een beperking volwaardig mee kunnen doen en bij kunnen
dragen aan de reguliere samenleving.

Nederland heeft het VN-verdrag inzake rechten van personen met een
handicap getekend, dus hebben we de plicht om dit na te leven en de
omstandigheden te verbeteren. Zonodig moeten we sancties opleggen als
dit niet gebeurt.

Om de inclusieve maatschappij voor mensen met een beperking te
waarborgen, stellen wij de volgende punten voor.

\textbf{Aanpassing wetgeving en naleving bestaande verdragen}

\begin{enumerate}
\def\labelenumi{\arabic{enumi}.}
\item
  \begin{quote}
  Nederland moet gehouden worden aan het zonder meer naleven van het
  VN-verdrag inzake rechten van personen met een handicap. Bij
  niet-naleven moeten klachten ingediend kunnen worden.
  \end{quote}
\item
  \begin{quote}
  Er worden sancties opgelegd als er geen concrete wetgeving en/of
  concreet beleid komt om de positie van mensen met een beperking te
  verbeteren, bijvoorbeeld als er ontoegankelijk wordt gebouwd.
  \end{quote}
\end{enumerate}

\textbf{Onderzoek en luisteren naar ervaringsdeskundigen voor een beter
beleid}

\begin{enumerate}
\def\labelenumi{\arabic{enumi}.}
\item
  \begin{quote}
  Bij het ontwikkelen van beleid wordt er ruimte gemaakt voor
  ervaringsdeskundigen en organisaties die bestaan uit
  ervaringsdeskundigen. Er worden toegankelijkheidsvoorzieningen
  geregeld, zoals een rolstoeltoegankelijke locatie en faciliteiten voor
  dove of slechthorende mensen.
  \end{quote}
\item
  \begin{quote}
  Er wordt onderzocht hoe gender, etniciteit en seksualiteit invloed
  hebben op het leven en welzijn van mensen met een beperking, zodat
  economisch en sociaal beleid effectiever en inclusiever gemaakt kan
  worden.
  \end{quote}
\item
  \begin{quote}
  Er komt een onderzoek naar de ervaringen van discriminatie en
  bejegening van mensen met een beperking. Er komt concreet beleid om
  deze discriminatie tegen te gaan, bijvoorbeeld door het geven van
  trainingen over de beginselen en normen van het VN-verdrag inzake
  rechten van personen met een handicap. Deze trainingen worden gegeven
  aan overheidspersoneel, rechters en advocaten, architecten,
  ontwerpers, onderwijspersoneel en anderen die te maken hebben met
  beleid en ondersteuning voor mensen met een beperking. De regionale
  meldpunten voor discriminatie worden actiever op het gebied van deze
  vorm van discriminatie.
  \end{quote}
\end{enumerate}

\textbf{Maatregelen om de samenleving toegankelijker te maken}

\begin{enumerate}
\def\labelenumi{\arabic{enumi}.}
\item
  \begin{quote}
  Alle bestaande overheidsgebouwen worden waar mogelijk volledig
  toegankelijk. Aan nieuwe overheidsgebouwen wordt de eis gesteld dat ze
  volledig toegankelijk zijn.
  \end{quote}
\item
  \begin{quote}
  Alle openbare toiletten worden toegankelijk gemaakt en zijn gratis te
  gebruiken.
  \end{quote}
\item
  \begin{quote}
  In de publieke ruimte wordt gratis drinkwater beschikbaar en
  toegankelijk aangeboden.
  \end{quote}
\item
  \begin{quote}
  Er worden doelgerichte budgetten vrijgemaakt voor gemeenten om
  toegankelijkheid in de openbare ruimte te bevorderen. Voorbeelden
  daarvan zijn geleidestroken op belangrijke en gevaarlijke
  verkeerspunten, het tegengaan van fietsen op stoepen en het
  toegankelijker maken van straten en stoepen.
  \end{quote}
\item
  \begin{quote}
  De landelijke overheid, gemeenten en vervoerders moeten gezamenlijk
  maatregelen nemen om de toegankelijkheid van het openbaar vervoer en
  doelgroepenvervoer te verbeteren. Het vernieuwde Besluit
  Toegankelijkheid Openbaar vervoer is hiervoor de basis. De overheid
  ziet toe op naleving van dit besluit. Persoonsvolgende
  vervoersvoorzieningen voor mensen met een beperking worden mogelijk
  gemaakt. Voor mensen die niet met het openbaar vervoer kunnen reizen,
  worden oplossingen op maat gemaakt.
  \end{quote}
\item
  \begin{quote}
  De oversteekduur van verkeerslichten wordt langer voor mensen met een
  lagere beweegsnelheid. Alle verkeerslichten worden gebruiksvriendelijk
  gemaakt voor mensen met een visuele beperking.
  \end{quote}
\item
  \begin{quote}
  De Nederlandse Gebarentaal wordt als officiële taal erkend, zodat
  mensen met een auditieve beperking erkenning krijgen voor hun
  identiteit en taligheid. Het aanbieden van een gebarentolk wordt
  hierdoor ook verplicht gesteld.
  \end{quote}
\item
  \begin{quote}
  Informatie in het publieke domein wordt beschikbaar gemaakt in
  braille.
  \end{quote}
\item
  \begin{quote}
  Brieven van de overheid en belangrijke brieven van instanties, moeten
  voor een ieder begrijpelijk zijn.
  \end{quote}
\item
  \begin{quote}
  Het media-aanbod (in alle varianten) moet toegankelijk zijn voor
  iedereen.
  \end{quote}
\end{enumerate}

\textbf{RECHT OP ZELFBESCHIKKING}

Het individuele recht op zelfbeschikking is een belangrijk recht dat te
allen tijde beschermd en nageleefd moet worden. Je hebt het recht goed
geïnformeerd te worden over zaken die betrekking hebben op jouw leven en
lichaam en op basis daarvan je eigen weloverwogen keuzes te maken. Dit
betekent ook dat je het recht hebt om je te kleden zoals jij wilt, het
recht om lief te hebben wie jij wilt, en het recht om jezelf uit te
drukken zoals jij wilt en te geloven waarin jij wilt. Het recht op
zelfbeschikking is een groot goed, van je geboorte tot aan je dood. Het
betekent dat je zelf mag bepalen wat je met je lichaam doet, welke
medische of psychische behandelingen je wel of niet wilt ondergaan.
Natuurlijk is het recht op zelfbeschikking wel aan grenzen gebonden,
namelijk waar jouw recht datzelfde recht van anderen schendt.

Het recht op zelfbeschikking is bij bepaalde mensen meer in gevaar dan
bij andere mensen. In Nederland is momenteel het recht op
zelfbeschikking in de volgende gevallen in gevaar of zelfs beperkt:

\begin{itemize}
\item
  \begin{quote}
  \underline{Moslimvrouwen (13)} wordt opgelegd wat zij wel of niet
  mogen dragen (d.m.v. het Gedeeltelijk Verbod Gezichtsbedekkende
  Kleding);
  \end{quote}
\end{itemize}

\begin{itemize}
\item
  \begin{quote}
  Mensen die een \underline{abortus (14)} ondergaan worden regelmatig
  bedreigd bij de ingang van de abortuskliniek;
  \end{quote}
\item
  \begin{quote}
  De identiteit van non-binaire personen wordt in Nederland niet
  officieel erkend;
  \end{quote}
\item
  \begin{quote}
  Er is weinig juridische ruimte voor niet-traditionele gezinnen,
  bijvoorbeeld voor mensen die met meer dan twee mensen een kind
  opvoeden, of voor iemand die zijn erfenis wil nalaten aan hun beste
  vriend;
  \end{quote}
\item
  \begin{quote}
  Mensen die ondraaglijk psychisch lijden hebben te weinig rechten om
  hun leven op een waardige en zelfgekozen manier te beëindigen;
  \end{quote}
\item
  \begin{quote}
  Mensen zonder baarmoeder beslissen nog steeds in grote mate over de
  rechten van mensen met baarmoeder.
  \end{quote}
\end{itemize}

BIJ1 staat achter de intersectionele feministische beweging die het
recht op zelfbeschikking voor iedereen wil waarborgen. Daarom stelt BIJ1
de volgende beleidsveranderingen voor.

\textbf{Toegankelijke anticonceptie en abortus}

\begin{enumerate}
\def\labelenumi{\arabic{enumi}.}
\item
  \begin{quote}
  Er komt een volledige vergoeding van anticonceptie.
  \end{quote}
\item
  \begin{quote}
  Het taboe rondom abortus wordt actief bestreden: abortus wordt uit het
  Wetboek van Strafrecht gehaald. Het Ministerie van Volksgezondheid,
  Welzijn en Sport gaat een campagne starten om abortus bespreekbaar te
  maken.
  \end{quote}
\item
  \begin{quote}
  De overheid stopt met het subsidiëren van organisaties die geen
  onafhankelijke hulp bieden aan ongewenst zwangere personen.
  \end{quote}
\item
  \begin{quote}
  De bedenktijd van 5 dagen voor een abortus wordt afgeschaft.
  \end{quote}
\item
  \begin{quote}
  Ervaringsdeskundigen worden actief betrokken bij het ontwikkelen van
  beleid rond abortuszorg.
  \end{quote}
\item
  \begin{quote}
  Abortuszorg gaat onder de reguliere zorg vallen; er komen
  abortusafdelingen in verloskundigenpraktijken en andere medische
  centra. Zowel verloskundigen als de huisarts moeten de abortuspil
  kunnen voorschrijven en werken hierin actief samen met abortusartsen.
  \end{quote}
\end{enumerate}

\textbf{Zwanger worden, bevallen en adoptie}

\begin{enumerate}
\def\labelenumi{\arabic{enumi}.}
\item
  \begin{quote}
  Vruchtbaarheidsbehandelingen zoals IVF moeten vergoed worden. Het
  aantal vergoede behandelingen moet uitgebreid worden op basis van
  kansen, niet kosten. Dit geldt ook voor LHBTQI+ stellen.
  \end{quote}
\item
  \begin{quote}
  Mensen die zwanger zijn en gaan bevallen hebben altijd het recht om
  zelf beslissingen te nemen over wat er met hun lichaam gebeurt. Er is
  geen afgedwongen zorg meer, en er komt respect voor autonomie en
  zeggenschap voor, tijdens en na de bevalling.
  \end{quote}
\item
  \begin{quote}
  Mensen die als kind zijn geadopteerd hebben het recht om de feiten van
  hun afstamming te weten. Bij binnenlandse adopties betekent dit, dat
  de afstammingsgegevens niet uit de archieven mogen worden gewist. Bij
  buitenlandse adopties betekent dit dat de overheid gecontroleerd moet
  hebben of de biologische ouders ingestemd hebben met de adoptie.
  Gegevens als afstammingsfeiten, geboortedatum, geboorteplaats,
  oorspronkelijke naam, namen van ouders en familie moeten op aanvraag
  beschikbaar zijn.
  \end{quote}
\end{enumerate}

\textbf{Geslachtsontwikkeling en gender (niet-standaard)}

\begin{enumerate}
\def\labelenumi{\arabic{enumi}.}
\item
  \begin{quote}
  Er komt wetgeving om een einde te maken aan medisch ingrijpen bij
  (niet-levensbedreigende) geslachtelijke ontwikkeling van intersekse
  kinderen die niet standaard is. De zorgvrager krijgt recht op advies
  en ondersteuning van ervaringsdeskundigen bij vragen rond
  geslachtsvariatie.
  \end{quote}
\item
  \begin{quote}
  Een genderwijziging mag zonder beoordeling en zo vaak als nodig.
  \end{quote}
\item
  \begin{quote}
  Uiteindelijk wordt de genderregistratie afgeschaft in de BRP (Basis
  Registratie Personen) en tot die tijd wordt het invullen van `X'
  (onbekend, irrelevant) een mogelijkheid op identiteitsdocumenten en
  reispapieren.
  \end{quote}
\item
  \begin{quote}
  Non-binaire mensen mogen hun voornaam kosteloos wijzigen in de
  gemeentelijke basisadministratie.
  \end{quote}
\end{enumerate}

\textbf{Diverse gezinsvormen}

\begin{enumerate}
\def\labelenumi{\arabic{enumi}.}
\item
  \begin{quote}
  De Nederlandse wet gaat uit van `traditionele' gezinnen. Er wordt niet
  genoeg rekening gehouden met de meer diverse gezinsvormen, zoals
  samengestelde gezinnen, mensen die met meer dan twee mensen een kind
  opvoeden, en ongehuwde maar samenwonende mensen. Dit moet veranderen.
  Er komt een wet voor meerouderschap. Het huidige trouwrecht en
  samenlevingsrecht, worden op zo'n manier aangepast dat er gelijke
  sociale en financiële rechten worden geboden aan alle gezinnen,
  samenwonenden en samenlevenden (zoals dit in het erfrecht ook het
  geval is).
  \end{quote}
\end{enumerate}

\textbf{Ondraaglijk lijden en euthanasie}

\begin{enumerate}
\def\labelenumi{\arabic{enumi}.}
\item
  \begin{quote}
  Mensen die hun leven als voltooid ervaren, moeten de vrijheid hebben
  een weloverwogen en vrijwillige keuze te maken om hun leven op een
  waardige manier te beëindigen. Euthanasie wordt uit het Wetboek van
  Strafrecht gehaald. Artsen worden hierdoor pas strafbaar als zij zich
  bij euthanasie niet aan de regels houden.
  \end{quote}
\item
  \begin{quote}
  Artsen die een verzoek tot euthanasie weigeren te beoordelen, worden
  verplicht om patiënten door te sturen naar een arts die bereid is dat
  wel te doen.
  \end{quote}
\item
  \begin{quote}
  Euthanasie wordt ook beschikbaar gesteld voor kinderen onder de twaalf
  jaar.
  \end{quote}
\item
  \begin{quote}
  Er komt meer bewustwording over ondraaglijk psychisch lijden, en het
  wordt gemakkelijker om in testamenten vast te leggen onder welke
  voorwaarden iemand niet meer wil leven, zodat ook patiënten met
  dementie beter geholpen kunnen worden met een zelfgekozen einde.
  \end{quote}
\item
  \begin{quote}
  Er wordt onderzocht of er specifieke gevallen zijn waarin hulp bij
  zelfdoding niet strafbaar is, zoals het helpen van een dierbare met
  een duurzame wens tot zelfdoding.
  \end{quote}
\end{enumerate}

\textbf{Zelfbeschikking over religieuze uitingen}

\begin{enumerate}
\def\labelenumi{\arabic{enumi}.}
\item
  \begin{quote}
  Alle vormen van religieus onderwijs, scholen en gebedshuizen worden
  volgens gelijke criteria beschermd.
  \end{quote}
\item
  \begin{quote}
  De overheid waarborgt het recht op zelfbeschikking van moslimvrouwen.
  Het gedeeltelijk verbod op gezichtsbedekkende kleding (in de volksmond
  `het niqab-verbod') moet per direct worden opgeheven. Ook het verbod
  op het dragen van een hoofddoek voor rechters en griffiers wordt
  opgeheven.
  \end{quote}
\item
  \begin{quote}
  Nederland gaat zich op Europees niveau hard maken voor wetgeving die
  ervoor zorgt dat werkgevers in Nederland niet mogen weigeren iemand in
  dienst te nemen met een hoofddoek, kippah (keppeltje) of andere
  religieuze attributen.
  \end{quote}
\item
  \begin{quote}
  Er komt meer keuzevrijheid in het opnemen van vrije dagen op basis van
  religieuze overtuiging.
  \end{quote}
\item
  \begin{quote}
  Er komen meer begraafplaatsen voor eeuwige grafrust.
  \end{quote}
\end{enumerate}

\textbf{KLIMAATRECHTVAARDIGHEID}

De klimaatcrisis is één van de grootste vraagstukken van onze tijd.
Ecosystemen en biodiversiteit worden op grote schaal vernietigd door het
systeem van marktwerking met economische groei als belangrijkste doel.
Wij als mensen maken ook deel uit van deze ecosystemen; ook onze
leefomgeving staat onder druk van extreme veranderingen als gevolg van
deze crisis. Ook de luchtkwaliteit in Nederland is erg zorgelijk vanwege
de hoge uitstoot van landbouw, industrie en verkeer.

Het is onze verantwoordelijkheid om een rechtvaardige oplossing te
vinden. Zo'n oplossing heeft hoe dan ook drastische gevolgen voor het
winst-denken van overheden en grote bedrijven. Internationale
rechtvaardigheid en solidariteit zijn absoluut noodzakelijk voor een
oplossing. Landen die economisch worden uitgebuit ondervinden als eerste
de gevolgen van deze crisis, zoals zij helaas vaker het eerste
slachtoffer zijn van internationale noodsituaties.
Klimaatrechtvaardigheid bestaat niet zonder het bestrijden van dergelijk
\underline{klimaatracisme (15)}.

De enige mogelijkheid om tot een oplossing te komen, is radicale
systeemverandering. We zullen onze manier van productie moeten herzien
om te komen tot een groene, circulaire en duurzame samenleving. Veel
tijd om het tij te keren is er niet meer. Wij moeten snel, daadkrachtig
en rechtvaardig ingrijpen. Binnen Nederland willen we maatregelen nemen
op het gebied van beleid, energie, uitstoot, landbouw, natuur en
omgeving. In internationale samenwerking willen we sturen op het gebied
van industrie, handel en lobby's. Daarbij hebben we de volgende
kernpunten voor ogen.

\textbf{Nederland: beleid, uitstoot en energie}

\begin{enumerate}
\def\labelenumi{\arabic{enumi}.}
\item
  \begin{quote}
  De Nederlandse overheid roept per direct de klimaatcrisis uit. De
  urgentie van deze crisis moet officieel erkend worden, zodat hiernaar
  gehandeld kan worden.
  \end{quote}
\item
  \begin{quote}
  Alle overheidsbeleid moet worden getoetst op duurzaamheid en impact op
  klimaat en milieu, ook met terugwerkende kracht.
  \end{quote}
\item
  \begin{quote}
  De Nederlandse uitstoot van broeikasgassen is in 2025 minstens 75\%
  lager dan in 1990 en staat in 2030 op 0. Deze doelen worden wettelijk
  vastgelegd en zijn bindend.
  \end{quote}
\item
  \begin{quote}
  We organiseren een pilot voor een (door loting samengesteld)
  burgerforum, met als doel dit uiteindelijk in te zetten bij
  besluitvorming over onder andere klimaat- en ecologische
  rechtvaardigheid.
  \end{quote}
\item
  \begin{quote}
  Er komen geen nieuwe vergunningen voor olie- en gasvelden, op land en
  op zee. Er komt een snel en daadkrachtig afbouwplan voor de huidige
  olie- en gasproductie waarbij rechtvaardige oplossingen voor de
  werknemers voorop staan.
  \end{quote}
\item
  \begin{quote}
  De gaskraan in Groningen gaat volledig dicht. De overheid erkent haar
  aandeel en alle gedupeerden worden direct volledig en eerlijk
  gecompenseerd. De overheid eist deze compensatie vervolgens zelf op
  bij de verantwoordelijke bedrijven.
  \end{quote}
\item
  \begin{quote}
  Subsidies op (de productie van) fossiele brandstoffen en financiële
  steun aan de fossiele industrie wordt zo snel mogelijk, maar uiterlijk
  per 2022 beëindigd. Subsidies voor biomassacentrales worden per direct
  stopgezet.
  \end{quote}
\item
  \begin{quote}
  Er wordt fors geïnvesteerd in (nieuwe) duurzame energie-alternatieven,
  zodat alle kolencentrales zo snel mogelijk kunnen worden gesloten.
  \end{quote}
\item
  \begin{quote}
  Het budget voor het warmtefonds wordt verhoogd. Hierbij wordt rekening
  gehouden met inkomenspositie, zodat mensen met lagere inkomens niet de
  rekening gepresenteerd krijgen voor de energietransitie.
  \end{quote}
\item
  \begin{quote}
  In 2030 is 75\% van de woningvoorraad hoogwaardig geïsoleerd.
  Gemeenten, woningcorporaties en woningverhuurders krijgen daarvoor
  middelen en bindende doelen voorgeschreven vanuit de overheid.
  \end{quote}
\end{enumerate}

\textbf{Natuur en vervoer}

\begin{enumerate}
\def\labelenumi{\arabic{enumi}.}
\item
  \begin{quote}
  Het Meerjarenprogramma Infrastructuur, Ruimte en Transport (MIRT)
  wordt een bereikbaarheidsfonds met als doel niet alleen het
  financieren van beter openbaar vervoer, maar ook (lokale)
  voorzieningen voor fietsen en deelvervoer. Er wordt een
  bereikbaarheidsnorm vastgesteld waaraan alle financieringsplannen
  worden getoetst. Deze norm bepaalt de maximale geografische afstand
  tussen burgers en openbaar vervoer.
  \end{quote}
\item
  \begin{quote}
  OV-bedrijven worden genationaliseerd en het openbaar vervoer wordt
  gratis.
  \end{quote}
\item
  \begin{quote}
  Het openbaar vervoer is in 2030 volledig elektrisch.
  \end{quote}
\item
  \begin{quote}
  We zetten ons op Europees niveau in om zo snel mogelijk op duurzame
  wijze een Europees spoornetwerk te realiseren. Vluchten binnen Europa
  kunnen op deze manier worden uitgefaseerd.
  \end{quote}
\item
  \begin{quote}
  Schiphol mag niet verder uitbreiden. Ook de plannen voor Lelystad
  Airport worden stopgezet. Vluchten binnen het Koninkrijk (en Suriname)
  moeten ook bij een eventuele prijsstijging als gevolg van maatregelen
  om luchtverkeer te reduceren betaalbaar blijven.
  \end{quote}
\item
  \begin{quote}
  We gaan BTW en accijns heffen op kerosine. Lagere belastingen voor de
  luchtvaart zijn niet van deze tijd.
  \end{quote}
\item
  \begin{quote}
  We stoppen met snelwegverbredingen en leggen alleen nog nieuwe wegen
  aan als deze de bereikbaarheidsnorm bevorderen en niet ten koste gaan
  van natuurgebieden.
  \end{quote}
\item
  \begin{quote}
  Biodiversiteit wordt prioriteit. Maaibeleid wordt aangepast om natuur
  de ruimte te geven en er wordt plaats gemaakt voor meer (kleine)
  ecosystemen in parken en natuurgebieden.
  \end{quote}
\item
  \begin{quote}
  Alle natuurgebieden komen onder verantwoordelijkheid van de provincie.
  Zij krijgen doelstellingen op (o.a.) het gebied van biodiversiteit en
  wildstand.
  \end{quote}
\item
  \begin{quote}
  Er komt een nationaal bomenplan om het aantal bomen in Nederland
  binnen afzienbare tijd fors uit te breiden en uiteindelijk te
  verdubbelen.
  \end{quote}
\item
  \begin{quote}
  Het mestbeleid van intensieve veeteelt wordt veel strenger om de
  vervuiling van omringende natuurgebieden en grondwater tegen te gaan.
  Onze standpunten op het gebied van landbouw worden nader toegelicht in
  het hoofdstuk \emph{Landbouw en Visserij}.
  \end{quote}
\item
  \begin{quote}
  Er komt een landelijk vuurwerkverbod voor particulieren. Vuurwerkshows
  of lichtshows kunnen door gemeenten worden georganiseerd op centrale
  plekken. Verder mag nergens vuurwerk worden afgestoken.
  \end{quote}
\end{enumerate}

\textbf{Internationale samenwerking: industrie, handel \& lobby's}

\begin{enumerate}
\def\labelenumi{\arabic{enumi}.}
\item
  \begin{quote}
  Nederland zet zich in voor een forse herziening van het Europees
  landbouwbeleid, waarbij alleen nog subsidies worden verstrekt aan
  kringlooplandbouw.
  \end{quote}
\item
  \begin{quote}
  Nederland stemt niet in met en trekt zich terug uit verdragen als deze
  claimrecht bevatten, tot landbouwintensivering leiden, voor toename
  van uitstoot zorgen, of ontbossing versnellen. Voorbeelden van dit
  soort verdragen zijn CETA, EU-Mercosur en TTIP.
  \end{quote}
\item
  \begin{quote}
  Nederland zet zich in voor een grootschalige samenwerking binnen de VN
  als het aankomt op de gezamenlijke bestrijding van de klimaatcrisis.
  \end{quote}
\item
  \begin{quote}
  Er komt een algeheel verbod op de import, handel en doorvoer van
  ontbossingsproducten.
  \end{quote}
\item
  \begin{quote}
  Diplomatieke steun aan projecten die te maken hebben met (het
  produceren van) fossiele brandstoffen wordt beëindigd. Bedrijven die
  grotendeels investeren in fossiele brandstoffen worden uitgesloten van
  handelsmissies.
  \end{quote}
\item
  \begin{quote}
  De financiering van ontbossing, landroof en productie van fossiele
  brandstoffen wordt voor zowel private als publieke financiële
  instellingen onmogelijk.
  \end{quote}
\item
  \begin{quote}
  Er komt meer transparantie over lobbyactiviteiten van fossiele en
  vervuilende industrieën via een openbaar gepubliceerd lobbyregister.
  Dit register biedt informatie over welke organisaties lobbyisten in
  dienst hebben, waarvoor die lobbyen, en de financiële middelen die zij
  hier jaarlijks voor inzetten.
  \end{quote}
\item
  \begin{quote}
  Bedrijven verantwoordelijk voor het aanrichten van klimaatschade en
  humanitaire rampen, worden verantwoordelijk gehouden voor misdaden
  tegen mens en natuur.
  \end{quote}
\item
  \begin{quote}
  Er komt een reclameverbod voor de fossiele industrie.
  \end{quote}
\end{enumerate}

\textbf{DIERENRECHTEN EN DIERENWELZIJN}

De grondbeginselen van BIJ1 zijn radicale gelijkwaardigheid en
economische rechtvaardigheid. Hier horen ook de rechten van dieren bij.
Elk dier is een individu, met het recht op leven en welzijn. In een
samenleving die uitgaat van gelijkwaardigheid, is er ruimte voor dieren
om natuurlijk gedrag te vertonen en niet langer als product of vermaak
gezien te worden. Om deze gelijkwaardigheid te bereiken zijn er
maatregelen nodig die dierenleed bestrijden. De visserij en de
bio-industrie moeten aan banden worden gelegd. Ook willen we
veranderingen op het gebied van dierproeven, jacht, en bonthandel. Om
dierenwelzijn te waarborgen stelt BIJ1 de volgende oplossingen voor.

\textbf{Een einde aan de bio-industrie en leed in de visserij}

\begin{enumerate}
\def\labelenumi{\arabic{enumi}.}
\item
  \begin{quote}
  Er komt per direct een einde aan de onhoudbare bio-industrie. Er wordt
  geïnvesteerd in plantaardige en duurzame landbouw. Het aantal dieren
  dat gehouden mag worden gaat fors omlaag.
  \end{quote}
\item
  \begin{quote}
  Er komt wettelijke vastlegging van het beschikken over voldoende
  ruimte, weidegang, vrije uitloop, sociaal contact, voldoende
  schuilmogelijkheden en het recht op natuurlijk gedrag van dieren.
  \end{quote}
\item
  \begin{quote}
  Er komt een verbod op alle vormen van verminking van de snavels van
  kippen en kalkoenen. Het couperen van staarten wordt ook per direct
  verboden, evenals het onthoornen van dieren.
  \end{quote}
\item
  \begin{quote}
  Diertransporten mogen niet langer dan twee uur duren. Bij extreme
  temperaturen stoppen de transporten volledig.
  \end{quote}
\item
  \begin{quote}
  Stallen en slachterijen worden voorzien van permanent cameratoezicht.
  Sprinklerinstallaties en vluchtroutes voor stallen worden verplicht.
  \end{quote}
\item
  \begin{quote}
  Slachtmethoden die extra onnodig leed veroorzaken, zoals bijvoorbeeld
  de CO\textsubscript{2}-verdoving van varkens, worden per direct
  verboden. Het slacht-tempo gaat daarnaast flink omlaag. We stimuleren
  de samenleving om zo veel mogelijk een plantaardig voedingspatroon aan
  te nemen, om zo het slachten waar mogelijk te kunnen uitfaseren.
  \end{quote}
\item
  \begin{quote}
  Er komt een verbod op de vangst- en slachtmethodes van vissen die
  onnodig en vaak langdurig leed veroorzaken. Ook het levend koken van
  dieren zoals kreeften en garnalen wordt verboden.
  \end{quote}
\item
  \begin{quote}
  Er worden steeds meer beschermde gebieden aangewezen waar visserij
  niet is toegestaan.
  \end{quote}
\item
  \begin{quote}
  De eisen voor het `Beter Leven-keurmerk' worden flink verhoogd.
  Dierlijke producten zonder dit keurmerk mogen op korte termijn niet
  meer worden verkocht.
  \end{quote}
\item
  \begin{quote}
  Het geplande verbod op `verrijkte kooien' voor kippen gaat per direct
  in en ook het gebruik van `koloniekooien' wordt niet langer
  toegestaan.
  \end{quote}
\end{enumerate}

\textbf{Dieren zijn geen vermaak of product}

\begin{enumerate}
\def\labelenumi{\arabic{enumi}.}
\item
  \begin{quote}
  De rechten van dieren worden grondwettelijk verankerd.
  \end{quote}
\item
  \begin{quote}
  De productie, import, export en verkoop van bont stopt. Uitbreiding
  van bontfokkerijen wordt niet toegestaan.
  \end{quote}
\item
  \begin{quote}
  Dieren zijn er niet voor het vermaak van mensen. Dierentuinen en
  aquaria worden omgevormd tot (tijdelijke) opvang voor dieren die zich
  (nog) niet staande kunnen houden in hun oorspronkelijke leefgebied.
  \end{quote}
\item
  \begin{quote}
  Tradities waarbij dieren worden gepest en mishandeld, zoals
  `\emph{kallemooi}' en `\emph{zwijntje-tik}' worden verboden.
  \end{quote}
\item
  \begin{quote}
  De verkoop van dieren in dierenwinkels wordt ontmoedigd.
  Doorverwijzing naar een asiel of opvangcentrum wordt het alternatief.
  Ook wordt de handel van dieren via internet en markten verboden, om
  impulsaankopen tegen te gaan.
  \end{quote}
\item
  \begin{quote}
  Er komen strengere regels voor het fokken van dieren. Zo komt er een
  vergunningplicht voor fokkers en stopt het fokken op uiterlijke
  kenmerken in verband met de gezondheidsrisico's voor dieren.
  \end{quote}
\item
  \begin{quote}
  De hobbyjacht wordt per direct verboden.
  \end{quote}
\item
  \begin{quote}
  Er wordt harder opgetreden tegen stroperij. Dieren in het wild mogen
  niet worden gedood. Alleen in extreme gevallen, zoals bijvoorbeeld
  ernstige verwondingen of direct gevaar voor het voortbestaan van
  andere soorten, wordt een uitzondering gemaakt.
  \end{quote}
\item
  \begin{quote}
  Er wordt flink geïnvesteerd in alternatieven voor medische
  dierproeven, zodat deze zo snel mogelijk volledig kunnen worden
  stopgezet. Schoonmaakmiddelen worden volledig dierproefvrij.
  Cosmeticamerken mogen geen ingrediënten gebruiken die voor andere
  producten wel op dieren worden getest. Nederland zet zich op
  internationaal niveau in voor het stoppen van dierproeven.
  \end{quote}
\end{enumerate}

\textbf{LANDBOUW, VISSERIJ EN VOEDSEL}

Het thema landbouw, visserij en voedsel hangt nauw samen met andere
thema's, zoals klimaatrechtvaardigheid en dierenrechten. Ook de positie
van de agrarische sector als beroepsgroep is belangrijk. Deze
verwevenheid maakt dit onderwerp extra uitdagend. BIJ1 neemt alle
aspecten mee. Het samenwerken met de natuur en het streven naar
duurzaamheid zijn hierbij leidend. BIJ1 wil sterk optreden tegen de
verarming van onze natuur, zonder in te boeten aan voedselzekerheid en
de bestaanszekerheid voor onze boeren en vissers. We willen
klimaatverandering bestrijden, de landbouw en visserij verduurzamen en
naar een duurzaam voedselsysteem. Boeren en vissers moeten in deze
processen worden ondersteund.

\textbf{Klimaatcrisis}

De landbouw in Nederland ondervindt nu al de gevolgen van de
klimaatcrisis en de uitbuiting van onze eigen grond. De gezondheid van
de grond wordt extreem belast door overmatig gebruik en vervuiling. De
biodiversiteit is uitgeroeid en het land heeft haar natuurlijke
veerkracht verloren. Die veerkracht is nodig om te kunnen herstellen na
moeilijke periodes, zoals droogte en extreme regenval. Daarom wil BIJ1
inzetten op circulaire landbouw.

\textbf{Landbouw van de toekomst}

\textbf{Het doel van de Nederlandse landbouw zou moeten zijn dat we
produceren wat we op kunnen eten. We willen af van de huidige situatie
waarin internationale concurrentiepositie zo belangrijk is dat voedsel
wordt vernietigd en kleine boeren in de problemen komen. De prijs van
landbouwproducten wordt met behulp van subsidies laag gehouden, omdat we
willen concurreren op de internationale markt. De subsidies zijn echter
niet toereikend waardoor het onmogelijk is voor boeren om af te stappen
van hun destructieve en vervuilende werkwijze.}

\textbf{Wij pleiten voor een omvorming van de landbouw waarin naar
lokale behoefte wordt geproduceerd en stellen programma's op die de
landbouw in 10 jaar tijd duurzaam maken. We moeten investeren in
betaalbare verduurzaming van de sector voor een toekomstbestendige,
circulaire landbouw. Ook is het van belang voor het behoud van de natuur
die we nog bezitten, dat de veestapel sterk wordt ingeperkt en dat
boeren worden geholpen mest om te zetten in minder stikstofhoudende
vormen.}

\textbf{Visserij}

Ook de Noordzee heeft het zwaar. Er wordt momenteel gewerkt aan een
Noordzee-akkoord, waarin naar voren komt dat de visserij moet
verduurzamen en inkrimpen. Om de Noordzee de kans te geven zich te
herstellen en aan te passen, is het van belang dat de vispopulaties
blijven bestaan en de kans krijgen om te groeien.

\textbf{(Gezonde) voedselzekerheid voorop}

Naast het basisrecht op zorg, wonen en onderwijs, vindt BIJ1 dat
iedereen ook het basisrecht heeft op voedsel. De voedselzekerheid staat
echter onder druk. Steeds meer mensen moeten gebruik maken van
Voedselbanken en te veel kinderen gaan `s ochtends zonder eten naar
school. In ons voedselsysteem gaat er veel fout. Gezond voedsel als
groente en fruit is ruim 40\% duurder geworden tussen 2000 en 2017,
terwijl de lonen niet zo zijn mee-gestegen. En terwijl de één honger
lijdt, verspilt Nederland jaarlijks honderden miljoenen kilo's voedsel.
Het voedselsysteem stemt vraag en aanbod dus niet op elkaar af, doet
afbreuk aan ieders basisrecht op eten en stimuleert de verkoop van
ongezonde producten. Dat moet veranderen.\\
~\\
Wij stellen de volgende beleidsveranderingen voor op het gebied van
landbouw, visserij en voedsel.\\
~\\
\textbf{De overgang naar een duurzame landbouw}

\begin{enumerate}
\def\labelenumi{\arabic{enumi}.}
\item
  \begin{quote}
  We stellen programma's op die de landbouw in tien jaar tijd duurzaam
  maken. Daartoe maken we subsidies vrij en komt er een hoge belasting
  op het gebruik van insecticiden en de uitstoot van methaan en
  CO\textsubscript{2}.
  \end{quote}
\item
  \begin{quote}
  Op dit moment produceren boeren voor meer dan 80\% voor de export.
  Door sterker te richten op lokale productie verdwijnt veel vervoer en
  daarmee vervuiling. Dit gebeurt in samenspraak met de ecologisch
  agrarische sector.
  \end{quote}
\item
  \begin{quote}
  Grote (miljonairs)boerderijen worden in gemeenschapshanden gesteld om
  de voedselproductie veilig te stellen. Zo verzekeren we dat de
  agrarische sector veel meer voor binnenlandse consumptie gaat doen en
  minder beïnvloedbaar is door kapitalistische marktwerking.
  \end{quote}
\item
  \begin{quote}
  Er komt een verbod op giftige bestrijdings- en ontsmettingsmiddelen.
  \end{quote}
\item
  \begin{quote}
  De overheid gaat de productie van biologische en plantaardige
  producten actief ondersteunen. Hiertoe gaat de BTW op (biologische)
  groenten en fruit naar 0\%.
  \end{quote}
\item
  \begin{quote}
  De overheid gaat in Europees verband actief optreden in het
  prijsbeleid van landbouwproducten.
  \end{quote}
\end{enumerate}

\textbf{Het beschermen van de vissen en het tegengaan van overvisserij}

\begin{enumerate}
\def\labelenumi{\arabic{enumi}.}
\item
  \begin{quote}
  Momenteel wordt controle op vissersboten nauwelijks tot slecht
  uitgevoerd. Hierdoor krijgen vissers een vrijbrief om boven legale
  richtlijnen te vissen. Dit is eigenlijk een noodzaak om te kunnen
  blijven concurreren. Deze illegale overvisserij moet zwaar worden
  aangepakt: er komen streng gereguleerde zones waar visserij verboden
  wordt. Er komt een betere regulatie van wat er gevangen wordt.
  \end{quote}
\item
  \begin{quote}
  Boten moeten permanent hun radar aan laten staan, zodat hun locatie
  altijd duidelijk en controleerbaar is. Controle op vissersboten wordt
  geïntensiveerd, bijvoorbeeld door een extra onafhankelijke tak binnen
  de kustwacht in het leven te roepen. Deze tak is speciaal ingericht op
  het reguleren van de visserij.
  \end{quote}
\end{enumerate}

\textbf{Een duurzaam en eerlijk voedselbeleid}

\begin{enumerate}
\def\labelenumi{\arabic{enumi}.}
\item
  \begin{quote}
  Voedselmonopolies worden verboden.
  \end{quote}
\item
  \begin{quote}
  We ontwikkelen een Nationaal Voedselzekerheidsplan, waarin de overheid
  meer controle neemt op de prijzen van voedsel en garandeert dat
  iedereen die in Nederland woont of verblijft altijd gegarandeerd is
  van voedsel. Dit plan wordt opgesteld in samenwerking met
  Voedselbanken, supermarkten en andere aanbieders van levensmiddelen.
  \end{quote}
\item
  \begin{quote}
  We gaan zoeken naar manieren om supermarktketens op te breken en te
  deprivatiseren met instelling van zelfbestuur door werknemers. Over
  basisbehoeften hoort niet een kleine groep veel winst te maken, vraag
  en aanbod moeten bovendien op elkaar worden afgestemd.
  \end{quote}
\item
  \begin{quote}
  Voedselbedrijven worden gestimuleerd om over te stappen op ecologische
  principes. Er komen wettelijke normen voor het maximale gehalte aan
  zout en suiker in producten.
  \end{quote}
\item
  \begin{quote}
  Gezond eten moet voor iedereen, ongeacht inkomen, toegankelijk zijn:
  de BTW op groente en fruit wordt verlaagd naar nul procent.
  \end{quote}
\item
  \begin{quote}
  Subsidies aan veehouderij worden in z'n geheel afgebouwd. In plaats
  daarvan worden subsidies aangewend om kleine en middelgrote boeren te
  ondersteunen in de omwenteling naar een plantaardig voedselsysteem.
  \end{quote}
\end{enumerate}

\textbf{ASIEL EN MIGRATIE}

De vrijheid om te reizen is een universeel mensenrecht. BIJ1 staat voor
een wereld waarin dit recht beschermd en gestimuleerd wordt. Wij willen
een door heel Europa gedragen asiel- en migratiebeleid, dat
mensenhandel, marteling en uitbuiting bestrijdt. Dit beleid heeft
geweldloosheid, waardigheid en solidariteit als fundament. Geen mens is
illegaal. BIJ1 vindt het hoog tijd om de reputatie en rechtspositie van
vluchtelingen en arbeidsmigranten te verbeteren. Door mensen als
gelijkwaardig te zien in plaats van als last, kan Nederland juist baat
hebben bij migratie.

\textbf{Onderdeel van het probleem, dus onderdeel van de oplossing}

Niemand vertrekt definitief uit zijn thuisland zonder daar een geldige
reden voor te hebben. De reden om te vluchten heeft vaak direct of
indirect te maken met een geschiedenis van uitbuiting en kolonisatie
door Westerse landen. Conflicten die veroorzaakt zijn door deze landen
duren ook nu nog voort en vergroten de kloof tussen de rijkere en armere
landen. Bovendien dragen deze Westerse landen op dit moment actief bij
aan deze conflicten door middel van o.a. de wapenhandel. Als gevolg
hiervan zijn de leefomstandigheden zodanig verslechterd, dat er soms
geen andere keuze is dan te vertrekken. Omdat Nederland onderdeel is van
het probleem, is het onze verantwoordelijkheid om onderdeel te zijn van
een oplossing.

Door de klimaatcrisis zullen in de komende jaren meer gebieden
onleefbaar worden, dan wel direct door overstromingen, dan wel indirect
door verhoogde voedsel- en wateronzekerheid en daardoor veroorzaakte
conflicten. Nederland bereidt zich voor op het ontvangen van mensen die
door de klimaatcrisis hun thuis moeten verlaten. Ook zet Nederland zich
binnen Europa en internationaal in om hier internationale plannen en
afspraken over te maken, waarbij solidariteit en rechtvaardigheid
leidend zijn.

\textbf{Migratiecrisis? Nee, humanitaire crisis.}

In Nederland wordt al jaren gesproken van een \emph{migratiecrisis}.
Deze migratiecrisis bestaat, maar niet zoals die op dit moment wordt
geschetst door de overheid en andere politieke partijen. Zij beweren dat
Nederland overspoeld wordt door migranten die `onze' banen en `onze'
huizen innemen. Dit is niet waar. Sinds 2015 daalt het aantal
asielaanvragen en nareizigers constant doordat asielzoekers aan de
grenzen van Europa worden vastgezet. BIJ1 ziet deze crisis als een
\emph{humanitaire crisis}, die veroorzaakt wordt door falend Europees
opvangbeleid als de Turkijedeal, met mensenrechtenschendingen als
gevolg.

\textbf{Falend Europees opvangbeleid}

Massa's mensen verdrinken op de Middellandse Zee, doordat Italië weigert
deze migranten op te vangen. De EU is gestopt met het financieren van
reddingsacties en reddingsacties door humanitaire NGO's en anderen
riskeren boetes tot 1 miljoen euro. De EU subsidieert echter wel de
Libische kustwacht, die migranten vastzet in detentiecentra waar zij
worden gemarteld. `Opvang in de regio' is zo geen oplossing. Buurlanden
van conflictgebieden vangen bovendien 86\% van alle vluchtelingen op. De
opvang in de regio is verzadigd. De vluchtelingenkampen `in de regio'
zijn gevaarlijk: verblijf in onmenselijke leefomstandigheden zonder
zicht op verbetering laat mensen geen andere keus dan op goed geluk een
levensgevaarlijke reis te ondernemen naar een `veiliger' Europa. Van de
1,2 miljoen `kwetsbare vluchtelingen' die in dat soort
vluchtelingenkampen verblijven, nodigt Nederland er momenteel 500 uit om
zich te hervestigen in Nederland. Dit is niet genoeg.

\textbf{Wanbeleid bij IND en AZC's}

Door personeelstekorten en planningsfouten binnen de IND lopen
wachttijden op, nog voor deze een asielaanvraag in behandeling neemt. De
asielprocedure, die slechts enkele weken hoort te duren, duurt op deze
manier met gemak 2 jaar. De onzekerheid die dit met zich meebrengt
vergroot het trauma van mensen die daarbovenop ook bezorgd zijn om hun
familie in het land van herkomst. Hoewel de oorzaken van dit wanbeleid
eenvoudig zijn, blijken de gevolgen rampzalig. AZC's zijn overvol. Ook
genieten LHBTQI+ personen in deze situaties niet de bescherming die zij
nodig hebben.

Gezinshereniging wordt vertraagd en trauma wordt vergroot. Dit alles
zorgt ervoor dat het moeilijker wordt om een bestaan om te bouwen in
Nederland. In sommige gevallen worden mensen zelfs onterecht
gedeporteerd naar het land van herkomst. Minderjarige asielzoekers
worden bovendien vaak door mensensmokkelaars verscheept en tot slavernij
gedwongen. In Nederland verdwijnen jaarlijks honderden kinderen uit de
opvang zonder dat de overheid hier zicht of grip op heeft.

\textbf{Oplossingen op de korte en lange termijn}

Nederland heeft gefaald om internationale verdragen om te zetten in
doeltreffend beleid. Een radicale omslag is nodig. Hiervoor draagt BIJ1
een aantal doelstellingen aan voor de korte en de lange termijn.

\textbf{Korte termijn: een einde aan geweld en een begin van
solidariteit}

\begin{enumerate}
\def\labelenumi{\arabic{enumi}.}
\item
  \begin{quote}
  Nederland pleit binnen de EU actief voor meer toegankelijke, veilige
  vluchtroutes en een betere toegang tot asielprocedures. Asielzoekers
  behouden hierbij het recht om in Europa hun asielaanvraag in te
  dienen.
  \end{quote}
\item
  \begin{quote}
  Nederland zet extra geld en middelen in ter ondersteuning van
  nationale en internationale vluchtelingenorganisaties en om
  reddingsoperaties in het Middellandse Zeegebied weer op gang te
  brengen.
  \end{quote}
\item
  \begin{quote}
  Nederland hervestigt jaarlijks een groeiend aantal kwetsbare mensen
  uit vluchtelingenkampen.
  \end{quote}
\item
  \begin{quote}
  Er komt een generaal pardon voor alle uitgeprocedeerde asielzoekers en
  ongedocumenteerden die zich in Nederland bevinden.
  \end{quote}
\item
  \begin{quote}
  Nederland verleent verblijfsrecht aan staatlozen in Nederland door hen
  een Nederlands of staatloosheidspaspoort te geven.
  \end{quote}
\item
  \begin{quote}
  Mensen die ontheemd raken als gevolg van grote klimaatveranderingen in
  hun thuisland en die naar Nederland vluchten, ook wel
  klimaatvluchtelingen genoemd, krijgen recht op bescherming.
  \end{quote}
\end{enumerate}

\textbf{Korte termijn: betere opvang en procedures}

\begin{enumerate}
\def\labelenumi{\arabic{enumi}.}
\item
  \begin{quote}
  Asielzoekers worden alleen verhuisd naar andere locaties als dit
  essentieel is voor de procedure.
  \end{quote}
\item
  \begin{quote}
  Er komen structureel meer investeringen voor de materiële en
  persoonlijke ondersteuning van vluchtelingen in de AZC's.
  \end{quote}
\item
  \begin{quote}
  In AZC's komt meer aandacht voor LHBTQI+ personen en in het bijzonder
  trans personen. Zij moeten toegang hebben tot zorg in een veilige
  omgeving.
  \end{quote}
\item
  \begin{quote}
  De IND neemt zijn verantwoordelijkheid om de asielprocedure per
  asielzoeker niet langer dan 8 weken te laten duren.
  \end{quote}
\item
  \begin{quote}
  Er wordt meer geïnvesteerd in hoor- en beslismedewerkers van de IND om
  cultuursensitief te communiceren en asielzoekers te benaderen en
  beoordelen vanuit de context en waardigheid van de asielzoeker. Ook
  worden IND-medewerkers getraind om beter te herkennen als vrouwen zijn
  gevlucht voor gendergerelateerd geweld, zoals uithuwelijking en
  verkrachting.
  \end{quote}
\item
  \begin{quote}
  De geloofwaardigheidstoets wordt aangepast. Hoor- en beslismedewerkers
  mogen alleen zaken behandelen als zij expertise hebben in
  geloofwaardigheid, psychologie en de culturele context van de
  asielzoeker.
  \end{quote}
\item
  \begin{quote}
  De sociale advocatuur blijft bestaan en onvoorwaardelijk beschikbaar
  gedurende de gehele asielprocedure.
  \end{quote}
\item
  \begin{quote}
  De inburgeringstoets wordt door de overheid afgenomen en verandert in
  een naturalisatiecursus die mensen in twee jaar tijd mogen afronden.
  \end{quote}
\item
  \begin{quote}
  Niemand komt in vreemdelingenbewaring.
  \end{quote}
\item
  \begin{quote}
  Het `buiten schuld-criterium' moet zodanig worden aangepast, dat
  ongedocumenteerden die buiten hun schuld niet kunnen terugkeren, recht
  krijgen op een verblijfsvergunning.
  \end{quote}
\item
  \begin{quote}
  Er komt een integrale strategie voor het bestrijden van (mensen)handel
  in minderjarige asielzoekers.
  \end{quote}
\item
  \begin{quote}
  Minderjarige asielzoekers die langer dan 3 jaar in Nederland wonen,
  krijgen verblijfsrecht. Ze behouden hun verblijfsrecht als ze
  meerderjarig worden.
  \end{quote}
\item
  \begin{quote}
  Asielzoekers mogen werk zoeken en een studie beginnen. Ook
  ongedocumenteerden mogen werken en studeren. De Koppelingswet wordt
  afgeschaft.
  \end{quote}
\item
  \begin{quote}
  Bij elk gesprek met de vreemdelingenpolitie en/of IND moet een
  onafhankelijke vertrouwenspersoon en/of vertaler aanwezig zijn die een
  `vreemdeling' bijstaat. Deze regels worden streng gehandhaafd. De
  `vreemdeling' heeft direct en altijd het recht om contact met iemand
  van buiten aan te vragen, onafhankelijk van of deze persoon familie is
  of niet.
  \end{quote}
\item
  \begin{quote}
  Beroep tegen besluit tot uitzetting mag afgewacht worden in Nederland.
  Mensen in afwachting worden gesteund en zo nodig opgevangen.
  \end{quote}
\end{enumerate}

\textbf{Lange termijn: hervorming van het asiel- en migratiebeleid in
Nederland en Europa}

\begin{enumerate}
\def\labelenumi{\arabic{enumi}.}
\item
  \begin{quote}
  Migranten en asielzoekers krijgen toegang tot dezelfde rechten als
  Nederlanders, ongeacht hun documenten of gebrek daaraan. Nederland
  opent de grenzen voor alle landen en maakt zich binnen de Europese
  Unie hard voor een hervorming die in de gehele EU geldt.
  \end{quote}
\item
  \begin{quote}
  Nederland schaft de verblijfsvergunningen af en verstrekt
  verblijfsrecht aan migranten ongeacht hun achtergrond.
  \end{quote}
\item
  \begin{quote}
  Het mandaat van de IND wordt omgezet in een loketfunctie om mensen op
  weg te helpen in Nederland.
  \end{quote}
\item
  \begin{quote}
  Nederland zet zich op Europees niveau in voor het afschaffen van
  FRONTEX en de Dublinverordening.
  \end{quote}
\item
  \begin{quote}
  Nederland zet zich binnen de EU actief in voor een solidair
  asielsysteem, waarbinnen een eerlijke (her)verdeling van asielzoekers
  plaatsvindt middels een verdeelsleutel, en waarbij mediterrane landen
  ontlast worden. Hierbij wordt rekening gehouden met de behoeften van
  de asielzoeker, zoals culturele achtergrond en familiebanden.
  \end{quote}
\item
  \begin{quote}
  Nederland gaat actief asielzoekers uitnodigen die vastzitten in
  vluchtelingenkampen en beweegt andere Europese landen ertoe om de
  aantallen jaarlijkse uitnodigingen op te schroeven.
  \end{quote}
\item
  \begin{quote}
  Nederland sluit alle detentiecentra voor asielzoekers en stopt met
  alle deportaties.
  \end{quote}
\end{enumerate}

\textbf{EUROPA}

BIJ1 staat voor een rechtvaardig, gelijkwaardig en solidair Europa.
Daarom pleiten we voor radicale democratisering van de Europese Unie
(EU). De inspraak van Europese burgers moet makkelijker en
toegankelijker worden gemaakt en de invloed van verwoestende
multinationals moet tot een minimum beperkt worden. Wij willen een EU
waar mensenrechten- en klimaatverdragen worden ingezet en versterkt om
groots te verduurzamen en om welvaart eerlijker te verdelen en
onrechtvaardigheid tegen te gaan. De EU mag niet dienen als liberaal en
racistisch instituut dat de belangen van het grootkapitaal dient.

\textbf{Ondoorzichtig en anti-democratisch}

Een probleem van de EU in haar huidige vorm is dat het bestuur ervan,
met name als het gaat om de rol van nationale regeringen
vertegenwoordigd binnen de Raad van Ministers, ondoorzichtig te werk
gaat en burgers het nakijken hebben. De EU is nog altijd in veel
opzichten een `club van staten', niet de `Unie van volkeren en burgers`
die het zou moeten zijn. Ook worden andere instituten binnen de EU in
zeer grote mate beïnvloed door multinationals in plaats van Europese
burgers. Deze multinationals hoeven alleen rekenschap af te leggen aan
hun aandeelhouders en hebben dus vrij spel in de politiek waar zij dat
niet zouden moeten hebben. Ze zijn immers geen politieke partij. Deze
multinationals dragen vaak bij aan vervuiling, mensenrechtenschending en
discriminatie. Op deze manier dient de EU niet het belang van haar
bewoners. De EU moet doorzichtiger en democratischer worden.

\textbf{Verspilling en gierigheid}

Een ander probleem is de enorme kostenverspilling aan de ene kant en de
tegenstrijdige gierigheid aan de andere kant. Aan de ene kant verhuist
het Europese parlement onnodig elke maand van Brussel naar Straatsburg
en andersom. Dit kost 200 miljoen euro per jaar. Aan de andere kant
stelt de EU zich gierig op als het gaat om noodfondsen voor
Zuid-Europese landen. Dit zijn de landen die juist het hardst worden
getroffen door bijvoorbeeld de coronacrisis, maar ook de financiële
crisis in 2008, of de humanitaire migratiecrisis. Het lijkt er op deze
manier verdacht veel op dat gierigheid en zelfbehoud leidend zijn. Dat
hoort niet zo te zijn. We willen dat het uitgavenbeleid gebaseerd is op
menselijkheid, waardigheid en solidariteit.

\textbf{Rechtsstaat en mensenrechten staan onder druk}

Tegelijkertijd staan mensenrechten, zelfbeschikking en vrijheid van
meningsuiting steeds meer onder druk binnen Europa. In alle EU-landen
moet de rechtsstaat worden beschermd en landen moeten elkaar aanspreken
op schendingen van mensenrechten, zelfbeschikking en vrijheid van
meningsuiting. De bestaande mensenrechteninstrumenten moeten verbeterd
worden en actief worden ingezet om misstanden binnen de EU aan te
kaarten en daar gevolgen aan te verbinden.

Kortom, er is momenteel op Europees niveau een blokkade als het gaat
over democratie, milieu en mensenrechten. Om deze blokkade op te heffen
stellen wij de volgende maatregelen voor.

\textbf{Meer democratie en betere bescherming van mensenrechten}

\begin{enumerate}
\def\labelenumi{\arabic{enumi}.}
\item
  \begin{quote}
  We streven naar radicale democratisering van de EU. De transparantie
  van de Raad van Ministers en de interinstitutionele onderhandelingen
  moet sterk verbeterd worden. Verder moeten bestaande vormen van
  Europese burgerparticipatie -- zoals het petitierecht en het Europees
  burgerinitiatief -- versterkt worden.
  \end{quote}
\item
  \begin{quote}
  Burgers moeten actief betrokken worden en medezeggenschap krijgen bij
  een algehele herziening van de prioriteiten, plannen en organisatie
  van de Europese Unie, beginnende met de reeds aangekondigde
  Conferentie over de Toekomst van Europa en eindigend met een
  fundamentele hervorming van de Europese verdragen.
  \end{quote}
\item
  \begin{quote}
  Meer middelen worden ingezet voor het invoeren en in stand houden van
  mensenrechten, dit zijn instrumenten zoals het Europees Handvest, de
  rasrichtlijn en de naleving van het VN-verdrag inzake rechten van
  personen met een handicap.
  \end{quote}
\item
  \begin{quote}
  We streven naar het aannemen en invoeren van de
  gelijkebehandelingsrichtlijn binnen heel Europa.
  \end{quote}
\item
  \begin{quote}
  Meer middelen en capaciteit worden ingezet voor het beschermen van de
  rechten van migranten, asielzoekers, staatlozen en volkeren zoals de
  Roma en Sinti.
  \end{quote}
\item
  \begin{quote}
  Wij zetten ons in om ook binnen de Europese samenwerking te werken
  naar het steunen van verzet tegen imperialistische uitbuiting en
  onderdrukking.
  \end{quote}
\end{enumerate}

\textbf{Meer geld naar mensen en klimaat}

\begin{enumerate}
\def\labelenumi{\arabic{enumi}.}
\item
  \begin{quote}
  Meer middelen worden ingezet om het maatschappelijk middenveld
  (belangengroepen, mensenrechtenactivisten, dissidenten, vrije pers) te
  ondersteunen en versterken. Hierbij is de agenda van belangengroepen
  leidend. Financieringsinstrumenten zoals subsidies worden
  toegankelijker en behapbaarder gemaakt zodat ook kleine organisaties
  en sociale bewegingen hier gebruik van kunnen maken.
  \end{quote}
\item
  \begin{quote}
  We stellen een grootschalig en wettelijk bindend klimaatverdrag op
  Europees niveau op, inclusief verstrekkende doelstellingen en
  sancties.
  \end{quote}
\end{enumerate}

\textbf{Bestrijden van ongewenste activiteiten van multinationals en
ongewenste handel}

\begin{enumerate}
\def\labelenumi{\arabic{enumi}.}
\item
  \begin{quote}
  Er komt nieuwe wetgeving, waaronder een verplicht lobbyregister, om
  lobbywerk transparanter en ethischer te maken. Ook zoeken we naar
  effectieve manieren om lobby van multinationals en grote private
  bedrijven terug te dringen.
  \end{quote}
\item
  \begin{quote}
  We streven naar het op Europees niveau verhogen van belastingen voor
  multinationals en vermogenden, en zetten breder in op het opsporen van
  kapitaalvlucht en belastingfraude en -ontduiking.
  \end{quote}
\item
  \begin{quote}
  Er komt een verbod op alle export van Europese goederen waarmee
  mensenrechten geschonden worden in andere landen, zoals wapens en
  onderdelen of technologie voor wapens en surveillance.
  \end{quote}
\end{enumerate}

\textbf{Tegengaan van militarisering en afschaffing van onnodige
uitgaven}

\begin{enumerate}
\def\labelenumi{\arabic{enumi}.}
\item
  \begin{quote}
  We maken een einde aan de militarisering van de grenzen van Europa.
  \end{quote}
\item
  \begin{quote}
  We voorkomen dat er ooit een Europees Leger komt.
  \end{quote}
\item
  \begin{quote}
  Er komt een eind aan absurde kostenposten zoals de maandelijkse
  verhuizing naar Straatsburg.
  \end{quote}
\item
  \begin{quote}
  We voorkomen dat Europese samenwerking wordt ingezet om internationaal
  mensen en/of landen uit te buiten of te onderdrukken. Europese
  bedrijven en projecten die investeren in buitenlandse projecten of
  ondernemingen moeten kunnen aantonen dat dit niet ten nadele van de
  lokale bevolking en het lokale klimaat gaat.
  \end{quote}
\end{enumerate}

\textbf{INTERNATIONALE SAMENWERKING}

BIJ1 wil dat Nederland zich proactief inzet voor het uitdragen van
internationale solidariteit en samenwerking. Dat is hard nodig, want er
zijn veel problemen in de wereld die ons allemaal aangaan en die we
alleen samen kunnen oplossen. Het beschermen van mensenrechten,
gelijkwaardigheid, klimaatrechtvaardigheid, zelfbeschikking en vrijheid
van verkeer moet centraal staan in ons buitenlandbeleid. Ook moet
Nederland de gevolgen van de eigen koloniale geschiedenis erkennen en
zich inzetten om de toegebrachte schade te herstellen.

\textbf{Problemen die we samen moeten bestrijden}

Mensenrechten, burgerlijke vrijheden, het klimaat en de welvaart staan
in veel landen steeds meer onder druk. Verworven rechten voor etnische
en religieuze minderheden, vrouwen, mensen met een beperking,
journalisten, klimaatactivisten en mensenrechtenactivisten lopen gevaar.
De inkomensongelijkheid tussen landen wordt alleen maar groter.
Klimaatcrises, humanitaire rampen, conflicten en uitbraken van
epidemieën worden niet effectief bestreden. Alleen met internationale
solidariteit kunnen we klimaatdoelen halen, conflicten beëindigen en
armoede effectief bestrijden. Dat is in het belang van iedereen, dus ook
in het belang van Nederland.

\textbf{Inlossen koloniale schuld}

Nederland heeft een koloniale schuld in te lossen. Onze rijkdom is in
het verleden verworven over de ruggen van onderdrukte mensen in de
Nederlandse koloniën en handelsposten. Ook nu nog verdient Nederland
geld aan landen die een stuk minder welvarend zijn. Die landen zien hier
weinig voor terug. Het beschermen van grote bedrijven die vaak belasting
ontwijken is steeds belangrijker geworden voor onze internationale
betrekkingen. Dat moet stoppen. In plaats daarvan moeten we in deze
landen de mensenrechten beschermen en werken aan wederopbouw en
reparaties.

We willen af van ontwikkelingssamenwerking als symptoombestrijding en we
willen toe naar een wereld waarin dit niet meer nodig is en we
samenwerken op basis van gelijkwaardigheid. Wij zien een toekomst waarin
eerlijke handel wèl vanzelfsprekend is, waarin landen van hun schulden
verlost zijn en waarin de gevolgen van kolonisatie zijn gerepareerd.

BIJ1 stelt de volgende maatregelen voor met betrekking tot
internationale samenwerking.

\textbf{Economische rechtvaardigheid}

\begin{enumerate}
\def\labelenumi{\arabic{enumi}.}
\item
  \begin{quote}
  Nederland zet zich internationaal actief in voor het kwijtschelden van
  alle schulden van landen met lage inkomens, lage middeninkomens, en
  middeninkomens.
  \end{quote}
\item
  \begin{quote}
  Het budget voor internationale armoedebestrijding wordt opgehoogd tot
  1\% van het BNP. Niets van dit budget wordt gebruikt voor het
  beheersen van migratiestromen, zoals nu het geval is. Speciale
  aandacht gaat naar het bestrijden van onderliggende oorzaken van
  armoede en ongelijkheid.
  \end{quote}
\item
  \begin{quote}
  Nederland verhoogt capaciteit en fondsen om internationale en
  multilaterale (mensenrechten)instituties zoals VN-instellingen en
  speciaal rapporteurs te ondersteunen en versterken.
  \end{quote}
\item
  \begin{quote}
  Nederland gaat hard inzetten op het bestrijden van belastingontwijking
  van multinationals in Nederland. Belastinglekken in wetgeving worden
  gedicht.
  \end{quote}
\item
  \begin{quote}
  Handel en ontwikkelingssamenwerking worden uit elkaar getrokken en
  komen niet meer terug in één ministerie. Handelsbelang mag nooit
  gekoppeld zijn aan internationale reparaties van aangedane schade.
  \end{quote}
\end{enumerate}

\textbf{Mensenrechten en verduurzaming}

\begin{enumerate}
\def\labelenumi{\arabic{enumi}.}
\item
  \begin{quote}
  De Nederlandse inzet voor mensenrechten wordt een integraal onderdeel
  van alle Nederlandse internationale betrekkingen, zoals bij
  handelsmissies en justitiële samenwerking.
  \end{quote}
\item
  \begin{quote}
  Nederland zet zich actief in voor het behalen van de Duurzame
  Ontwikkelingsdoelen met concrete \emph{targets} en combineert dit met
  mensenrechtenbeleid.
  \end{quote}
\item
  \begin{quote}
  Nederland vergroot zijn directe investering in maatschappelijke
  organisaties en mensenrechtenverdedigers in het buitenland, die zich
  inzetten voor rechtvaardigheid en gelijkwaardigheid. De agenda's en
  behoeften van deze organisaties worden hierbij leidend.
  \end{quote}
\item
  \begin{quote}
  De landenselectie voor internationale hulp wordt bepaald op basis van
  noodzaak in de landen zelf en de toegevoegde waarde van de Nederlandse
  bijdrage, niet op basis van de Nederlandse agenda op het gebied van
  handel, migratie of terrorismebestrijding.
  \end{quote}
\item
  \begin{quote}
  Nederland spreekt bedrijven die opereren in conflictgebieden aan op
  hun verantwoordelijkheid onder de OESO-richtlijnen en verbindt
  consequenties aan het niet naleven van die richtlijnen. Het niet
  naleven van de OESO-richtlijnen wordt standaard opgenomen als
  uitsluitingsgrond in het PvE (programma van eisen) voor openbare
  aanbestedingen.
  \end{quote}
\item
  \begin{quote}
  Nederland zet zich actief in voor het ontwikkelen van zowel nationale
  als internationale regelgeving, om bedrijven die zich schuldig maken
  aan mensenrechtenschendingen aansprakelijk te stellen. Slachtoffers
  worden erkend en gecompenseerd.
  \end{quote}
\item
  \begin{quote}
  Nederland stelt protocollen op voor eerlijke handel, duurzaamheid en
  mensenrechten voor bedrijven die gevestigd zijn in Nederland. Bij het
  overtreden van deze protocollen worden bedrijven beboet of Nederland
  uitgezet.
  \end{quote}
\item
  \begin{quote}
  Nederland zet zich internationaal in voor het bestrijden van
  klimaatverandering in landen met een laag inkomen of laag-midden
  inkomen, inclusief een financieringsmechanisme voor verlies en schade.
  \end{quote}
\item
  \begin{quote}
  Nederland zet zich internationaal in om de belasting op vermogen,
  winst en vervuiling te verhogen als onderdeel van een progressieve,
  alomvattende aanpak van extreme ongelijkheid en klimaatverandering.
  \end{quote}
\end{enumerate}

\textbf{Internationale rechtvaardigheid}

\begin{enumerate}
\def\labelenumi{\arabic{enumi}.}
\item
  \begin{quote}
  Er moet een einde komen aan de bezetting en kolonisatie van
  Palestijnse grond en het Palestijnse volk. Palestijnse vluchtelingen
  moeten het recht houden op terugkeer naar hun huizen en bezittingen.
  Zolang de bezetting voortduurt en de rechten van Palestijnen worden
  geschonden, schorten we het Associatieverdrag met Israel op en tevens
  alle handelsbetrekkingen met Israël. \underline{Palestina wordt direct
  erkend (16)}.
  \end{quote}
\item
  \begin{quote}
  Nederland erkent de Republik Maluku Selatan (Republiek der
  Zuid-Molukken) en steunt de onafhankelijkheidsstrijd van West-Papoea.
  \end{quote}
\item
  \begin{quote}
  Zelfbeschikking van volkeren geldt als uitgangspunt bij de erkenning
  van nieuwe staten.
  \end{quote}
\item
  \begin{quote}
  Nederland besteedt internationaal extra aandacht aan staatlozen,
  ontheemden en onderdrukte volkeren.
  \end{quote}
\item
  \begin{quote}
  Nederland zet zich actief in voor het corrigeren van koloniaal onrecht
  dat Nederland de voormalige Nederlandse koloniën heeft aangedaan.
  \end{quote}
\item
  \begin{quote}
  Nederland zet zich extra in voor het arresteren van verdachten van
  oorlogsmisdaden die aan het Internationaal Strafhof moeten worden
  overgedragen, evenals het aanklagen van politici en militairen uit de
  VS en Europa die zich schuldig hebben gemaakt aan oorlogsmisdaden
  onder de valse vlag van `terrorismebestrijding' of de omverwerping van
  anti-imperialistische regimes.
  \end{quote}
\item
  \begin{quote}
  Wij zetten ons extra in om bij het Internationaal Strafhof de
  mensenrechten van anti-imperialistische activisten te beschermen. We
  versterken dan ook de positie van het Internationaal Strafhof binnen
  de anti-imperialistische strijd.
  \end{quote}
\item
  \begin{quote}
  Nederland gaat door met het investeren in noodhulp, met een focus op
  preventie van conflicten. Nederland zet zich hierbij in voor een
  hervorming van het humanitaire systeem. In dit systeem is meer
  aandacht voor lokale humanitaire organisaties en de weerbaarheid van
  lokale gemeenschappen.
  \end{quote}
\item
  \begin{quote}
  We moedigen stages aan naar landen en gebieden die aan de voorgrond
  staan van het internationale anti-imperialistische verzet, zoals
  verzet in Palestina, de Filipijnen (tegen Duterte) en Venezuela. Dit
  met het doel om te leren, banden op te bouwen en het
  anti-imperialistische bewustzijn onder de Nederlandse bevolking te
  ontwikkelen.
  \end{quote}
\item
  \begin{quote}
  Nederland zet zich ook internationaal in voor veiligheid voor
  vrijdenkers, atheïsten en humanisten. Te veel van hen worden vermoord,
  vervolgd en verstoten. Nederland is voor hen een veilige haven.
  \end{quote}
\end{enumerate}

\textbf{STAATSINRICHTING EN RECHTSSTAAT}

BIJ1 staat pal achter democratie: een staatsvorm die gebaseerd is op het
principe van gelijkwaardigheid. Wanneer de gelijkwaardigheid van mensen
niet meer vanzelfsprekend is, wordt onze democratische rechtsstaat
aangetast. BIJ1 staat voor een eerlijke rechtsstaat voor iedereen. Een
rechtsstaat die vrij en veilig is en waarin burgers, klokkenluiders en
journalisten worden beschermd. In zo'n rechtsstaat, waarin
gelijkwaardigheid de basis is, is geen plaats voor ouderwetse symbolen
van ongelijkwaardigheid.

\textbf{Een rechtsstaat voor iedereen}

\begin{enumerate}
\def\labelenumi{\arabic{enumi}.}
\item
  \begin{quote}
  Alle mensenrechtenverdragen krijgen rechtstreekse werking in het
  Nederlandse recht. Dit betreft in ieder geval het
  Kinderrechtenverdrag, het Vrouwenrechtenverdrag en het VN-verdrag
  inzake rechten van personen met een handicap. We ratificeren het
  facultatief protocol van het VN-verdrag inzake rechten van personen
  met een handicap.
  \end{quote}
\item
  \begin{quote}
  Nederland neemt de aanbevelingen van internationale
  mensenrechtenorganisaties en -comités serieus en voert hun
  aanbevelingen uit in overeenkomst met ons programma.
  \end{quote}
\item
  \begin{quote}
  Etniciteit, afkomst, nationaliteit, beperking, seksuele gerichtheid,
  genderidentiteit en genderexpressie worden in artikel 1 van de
  Grondwet opgenomen.
  \end{quote}
\item
  \begin{quote}
  De overheid gaat haar recht (en plicht) weer gebruiken om organisaties
  die structureel discrimineren, groepen beledigen en/of haat zaaien te
  verbieden.
  \end{quote}
\item
  \begin{quote}
  De bestaande wetten tegen discriminatie worden strenger nageleefd. We
  breiden deze wetten uit door een juridische definitie van racisme,
  anti-zwart racisme en islamofobie op te stellen die recht doet aan het
  structurele karakter ervan. Discriminatiewetgeving wordt aangepast op
  deze definities.
  \end{quote}
\item
  \begin{quote}
  Groepsbelediging, digitaal pesten, discriminatie en oproepen tot
  racistisch geweld op sociale media worden hard bestreden, in eerste
  plaats door convenanten met de platforms.
  \end{quote}
\item
  \begin{quote}
  De bestaande wetgeving aangaande smaad, laster en haat zaaien wordt
  streng nageleefd, ook ter bestrijding van `fake news' en gevolgen
  hiervan.
  \end{quote}
\item
  \begin{quote}
  Belastingontduiking en fraude door de allerrijksten moet worden
  opgespoord en gestopt. We investeren in opsporingsteams en brengen de
  misdadigers voor de rechter.
  \end{quote}
\item
  \begin{quote}
  Kleding is een vrije keuze. Wetten die discriminatie bevorderen en
  zelfbeschikking van groepen mensen aantasten, zoals het gedeeltelijke
  niqaabverbod, worden geschrapt.
  \end{quote}
\item
  \begin{quote}
  Nederland steunt het principe dat een meerduidige identiteit
  meerwaarde heeft en verbreedt daarom de thans zeer beperkte
  mogelijkheden om een dubbele nationaliteit te behouden voor elke
  Nederlander die dat wenst.
  \end{quote}
\end{enumerate}

\textbf{Een vrije en veilige rechtsstaat}

\begin{enumerate}
\def\labelenumi{\arabic{enumi}.}
\item
  \begin{quote}
  Er komt een sterkere en openbare controle op het functioneren van de
  veiligheidsdiensten. Daarnaast moet duidelijk worden hoe de privacy
  van mensen gewaarborgd is in de algoritmes die de diensten gebruiken.
  Ook moet worden aangetoond dat de inzet van deze algoritmes niet leidt
  tot etnisch profileren.
  \end{quote}
\item
  \begin{quote}
  Binnenlandse veiligheidsdiensten worden beperkt in hun bevoegdheden.
  De bevoegdheden voor het verzamelen van bulk data en het hacken van
  derde personen, alsook de automatische toegang tot databases moeten
  worden teruggetrokken.
  \end{quote}
\item
  \begin{quote}
  We beëindigen alle samenwerking met de geheime diensten van
  anti-democratische landen zoals Israël en de Verenigde Staten. Om de
  veiligheid van (politieke) vluchtelingen, migranten en journalisten te
  bewaren wordt de gegevensverstrekking aan buitenlandse geheime
  diensten door Nederland sterk aan banden gelegd.
  \end{quote}
\item
  \begin{quote}
  De onafhankelijkheid van journalisten moet worden beschermd. Dat
  betekent dat we intimidatie, opsluiting en tracking van journalisten
  door OM en veiligheidsdiensten bestrijden. Ook waken we beter voor de
  veiligheid van deze onafhankelijke journalisten. Het is absurd dat
  journalisten vaker beveiliging moeten meenemen naar demonstraties,
  doordat zij worden bedreigd door extreemrechts gehits.
  \end{quote}
\item
  \begin{quote}
  Journalisten wereldwijd moeten hun werk kunnen doen zonder dat ze
  worden tegengewerkt met geweld, censuur en vervolging. In deze context
  gaat de overheid zich harder inzetten voor het vergroten van de
  veiligheid, ondersteuning en noodhulp aan journalisten.
  \end{quote}
\item
  \begin{quote}
  Klokkenluiders worden actief beschermd middels het inzetten van een
  onafhankelijke organisatie.
  \end{quote}
\item
  \begin{quote}
  Het recht op demonstratie wordt strikt nageleefd.
  \end{quote}
\end{enumerate}

\textbf{Een gezonde en directe democratie}

\begin{enumerate}
\def\labelenumi{\arabic{enumi}.}
\item
  \begin{quote}
  We richten een Constitutioneel Hof op dat wetten kan toetsen aan de
  Grondwet. Daarvoor heffen we Artikel 120 van de Grondwet op, dat het
  nu onmogelijk maakt om wetten aan de Grondwet te toetsen.
  \end{quote}
\item
  \begin{quote}
  We schaffen de monarchie, een eeuwenoud symbool van
  ongelijkwaardigheid en bovendien een grote kostenpost, af: Nederland
  wordt een Republiek met een president.
  \end{quote}
\item
  \begin{quote}
  We onteigenen het (uit koloniale tijden stammende) kapitaal van de
  Koninklijke familie en gebruiken dit geld voor rechtsherstel voor de
  voormalige koloniën en de Indische gemeenschap.
  \end{quote}
\item
  \begin{quote}
  Het stemrecht wordt hervormd. Zo wordt de stemgerechtigde leeftijd
  verlaagd naar 16 jaar en krijgen vluchtelingen en migranten sneller
  toegang tot het stemrecht. We maken het mogelijk voor gedetineerden om
  zelf te kunnen stemmen in plaats van uitsluitend bij volmacht.
  \end{quote}
\item
  \begin{quote}
  We stimuleren lokale vormen van directe democratie, zoals wijkraden en
  andere vormen van directe burgerparticipatie.
  \end{quote}
\item
  \begin{quote}
  We herzien de wet omtrent het burgerinitiatief en maken het
  burgerforum een standaard onderdeel in de vervolgstappen van een
  succesvol burgerinitiatief.
  \end{quote}
\end{enumerate}

\textbf{VEILIGHEID EN JUSTITIE}

BIJ1 streeft naar een samenleving die veilig is voor iedereen. Wij
vinden dat we als samenleving een gezamenlijke plicht hebben om misdaad
te voorkomen en om veiligheid te garanderen. Dat betekent dat we
kansenongelijkheid moeten aanpakken. En dat, wanneer mensen de fout in
gaan, we ze de mogelijkheid moeten bieden om hun fouten te herstellen.
Medemenselijkheid is voor BIJ1 de sleutel tot de preventie en aanpak van
misdaad. Daarnaast willen we ook geweld en misdaad aanpakken waarvoor op
dit moment te weinig verantwoording wordt afgelegd. Zoals het
vernietigen van de natuur, het uitbuiten van werknemers, etnisch
profileren en politiegeweld. We investeren in het uitbannen van
klassenjustitie, strijden voor een rechtvaardige samenleving, pakken
gendergerelateerd geweld aan en maken korte metten met institutioneel
racisme bij de politie.

\textbf{Herstelrecht in plaats van afstraffing}

Het huidige systeem van afstraffing en opsluiting is een racistisch
systeem dat niet werkt en niet gericht is op herstel. Etnisch profileren
is aan de orde van de dag. In de gevangenissen zitten twee keer zoveel
mensen met migratieachtergrond als mensen zonder migratieachtergrond,
terwijl mensen met migratieachtergrond slechts 24\% van de bevolking
uitmaken. Uit veel onderzoeken blijkt dat mensen van kleur stelselmatig
hogere straffen krijgen opgelegd dan witte mensen. Daarbij is er vaak
sprake van politiegeweld en repressie.

Onderzoek toont tevens aan dat afstraffing en opsluiting zelfs helemaal
geen effectieve middelen tegen criminaliteit zijn. Integendeel,
gevangenisstraffen en hardere straffen leiden vaak tot nog slechtere
herintegratie van daders, doen niets voor de slachtoffers en voorkomen
criminaliteit niet. Mee kunnen doen in de samenleving en de
verantwoordelijkheid moeten nemen om je fouten te herstellen zijn
daarentegen wél effectief. Daar moeten we dus naartoe.

\textbf{Naar een veilig, medemenselijk en rechtvaardig Nederland}

Voor een veilig, medemenselijk en rechtvaardig Nederland zet BIJ1 in op
uitbreiding van het herstelrecht als toevoeging op ons strafrecht.
Hierin is meer oog voor de onderliggende problemen bij geweld en
misdaad, ruimte voor voor genoegdoening en verwerking voor slachtoffers
en zijn er meer mogelijkheden voor daders om weer een volwaardig
onderdeel te worden van de maatschappij. Ook kiezen we ervoor niet te
investeren in het bestraffen van kleine misdaad en investeren we niet in
de politie. We willen juist wél investeren in een samenleving waarin
iedereen een tweede kans heeft en maatschappijbrede ongelijkheden worden
aangepakt.

BIJ1 stelt de volgende oplossingen voor om een samenleving te creëren
die veilig is voor iedereen.

\textbf{Van strafketen naar een eerlijke rechtsstaat}

\begin{enumerate}
\def\labelenumi{\arabic{enumi}.}
\item
  \begin{quote}
  We steken geld in goede collectieve voorzieningen en bevorderen
  gelijke kansen in de samenleving. We zorgen voor gelijke toegang tot
  onderwijs, een leefbaar inkomen en aanpak van maatschappijbreed
  racisme. We investeren niet verder in de politie. Het geld dat daarmee
  vrijkomt investeren we in goede collectieve voorzieningen.
  \end{quote}
\item
  \begin{quote}
  Mensen met onbegrepen gedrag zijn vaker slachtoffer van politiegeweld.
  Het is nodig dat we investeren in goede zorg, zodat mensen met
  onbegrepen gedrag de juiste hulp krijgen. De politie heeft hierin geen
  rol.
  \end{quote}
\item
  \begin{quote}
  We decriminaliseren niet-gewelddadige delicten, zoals drugsdelicten.
  \end{quote}
\item
  \begin{quote}
  Teelt, bezit en verkoop van softdrugs worden legaal, zodat de overheid
  meer zicht heeft op de kwaliteit van drugs. Invoering wordt
  gereguleerd. Monopolies op teelt van softdrugs door grote bedrijven
  worden bestreden.
  \end{quote}
\item
  \begin{quote}
  Harddrugs worden legaal en door de overheid gereguleerd. Zo kan er
  beter geadviseerd worden over gebruik van drugs en kunnen mensen met
  drugsproblematiek eerder worden geholpen.
  \end{quote}
\item
  \begin{quote}
  Om kleine misdaad te verminderen en buurten veiliger te maken zetten
  we pilots op voor buurt-rechtbanken. Dit zijn rechtbanken die kijken
  naar de omstandigheden die tot misdaad leiden. Hier staat niet
  vervolging, maar de mens, de leefbaarheid van de buurt en het
  herstellen van fouten centraal.
  \end{quote}
\item
  \begin{quote}
  We openen alleen nog nieuwe gevangenissen naar Fins `open' model en
  bouwen andere gevangenissen om. Zo houden daders contact met de
  samenleving, worden ze menswaardig behandeld en in staat gesteld om
  verantwoordelijkheid te nemen en gedane schade te herstellen.
  \end{quote}
\item
  \begin{quote}
  Gevangenisstraffen worden een uiterst middel. We delen voortaan eerder
  taakstraffen uit. Vervroegde vrijlating blijft daarnaast mogelijk. We
  versoepelen de voorwaarden om hiervoor in aanmerking te komen.
  \end{quote}
\end{enumerate}

\textbf{Van klassenjustitie naar bescherming van rechten}

\begin{enumerate}
\def\labelenumi{\arabic{enumi}.}
\item
  \begin{quote}
  We investeren in de sociale advocatuur en bestrijden klassenjustitie.
  We gaan sociale advocaten eerlijker betalen.
  \end{quote}
\item
  \begin{quote}
  We zorgen voor toegang tot juridische hulp bij ieder gemeentelijk
  loket en verzekeren dat deze hulp gratis is voor wie het niet kan
  betalen.
  \end{quote}
\item
  \begin{quote}
  De rechten van gevangenen worden beter verankerd in regelgeving en het
  naleven hiervan wordt beter gecontroleerd.
  \end{quote}
\item
  \begin{quote}
  Instanties die etnisch profileren worden strafrechtelijk vervolgd. Het
  nieuw op te richten Ministerie van Gelijkwaardigheid ziet erop toe dat
  gedane schade wordt hersteld.
  \end{quote}
\item
  \begin{quote}
  We handhaven strenger op onze antidiscriminatiewetgeving. Mensen die
  discrimineren krijgen een Educatieve Maatregel Discriminatie opgelegd.
  \end{quote}
\item
  \begin{quote}
  We zetten in op het opsporen en vervolgen van belastingontduiking en
  belastingfraude door grote bedrijven en de allerrijksten.
  \end{quote}
\item
  \begin{quote}
  We houden bedrijven verantwoordelijk die in Nederland, of buiten
  Nederland, de planeet en de natuur slopen. Hiervoor komt passende
  (internationale) regelgeving die gericht is op het herstellen van
  toegebrachte schade.
  \end{quote}
\end{enumerate}

\textbf{Van politiegeweld naar betere regels voor de politie}

\begin{enumerate}
\def\labelenumi{\arabic{enumi}.}
\item
  \begin{quote}
  Wie etnisch profileert wordt strafrechtelijk vervolgd. Daarnaast
  verbieden we het gebruik van etniciteit, nationaliteit en afkomst in
  risicoprofielen, introduceren we stopformulieren voor alle
  politiecontroles en worden bodycams de norm. Er komt een einde aan
  preventief fouilleren, `patsercontroles' en andere controles die
  etnisch profileren in de hand werken.
  \end{quote}
\item
  \begin{quote}
  De basisbewapening van de politie gaat terug naar wapenstok en
  pepperspray. Vuurwapens mogen alleen in uitzonderlijke gevallen worden
  gedragen. Training en protocollen omtrent vuurwapengebruik worden fors
  strenger: zo mag politie niet schieten om iemand aan te houden.
  \end{quote}
\item
  \begin{quote}
  We verbieden risicovolle arrestatietechnieken zoals de nekklem.
  \end{quote}
\item
  \begin{quote}
  Er komt een bureau dat onafhankelijk toezicht houdt op het
  functioneren van de politie en klachten over de politie in behandeling
  neemt.
  \end{quote}
\item
  \begin{quote}
  De praktische onschendbaarheid van politiefunctionarissen wordt
  ontmanteld. Wie een ander doodt, moet verantwoording afleggen.
  \end{quote}
\item
  \begin{quote}
  De politie wordt transparanter. Data over politiecontacten met burgers
  gaan ook informatie bevatten over etniciteit van burgers, aanleiding
  en uitkomst van het contact, verwondingen en overlijdensgevallen.
  Onderzoeken naar racisme en politiegeweld worden publiek gemaakt.
  \end{quote}
\item
  \begin{quote}
  Bij de politie komen er heldere regels over omgangsvormen en
  taalgebruik. Het wordt daarin makkelijker agenten te ontslaan vanwege
  racisme en geweld. Daarnaast verhinderen we dat de politie wordt
  bemenst door personen met anti-rechtsstatelijke en racistische
  meningen.
  \end{quote}
\item
  \begin{quote}
  Klokkenluiders worden in bescherming genomen en moeten alle ruimte
  krijgen om misstanden aan te kaarten.
  \end{quote}
\item
  \begin{quote}
  We werken naar afschaffing van de vreemdelingenpolitie en willen op
  termijn een einde aan alle uitzettingen. Daarbij hoort dat
  detentiecentra worden opgeheven en alle `vreemdelingen' in hechtenis
  vrijgelaten en door ervaringsdeskundige-organisaties begeleid worden.
  Ook komt er een onafhankelijk onderzoek naar mogelijke schendingen van
  mensenrechten door (mensen binnen) de vreemdelingenpolitie en
  eventuele vervolging.
  \end{quote}
\end{enumerate}

\textbf{Aanpak van gendergerelateerd geweld}

\begin{enumerate}
\def\labelenumi{\arabic{enumi}.}
\item
  \begin{quote}
  We ontwikkelen \underline{een integrale aanpak tegen gendergerelateerd
  geweld (17).} Dit betekent dat er meer onderzoek komt naar de rol die
  gender speelt in geweldsdelicten en welke vormen van onderwijs,
  opvoeding en voorlichting dit zouden kunnen verbeteren.
  \end{quote}
\item
  \begin{quote}
  Er wordt actief gewerkt aan het verhogen van de aangiftebereidheid van
  slachtoffers van gendergerelateerd geweld.
  \end{quote}
\item
  \begin{quote}
  De verkrachtingswetgeving wordt aangepast met onze definitie van
  consent als uitgangspunt.
  \end{quote}
\item
  \begin{quote}
  Er wordt structureel meer geïnvesteerd in de verschillende
  vrouwenopvangcentra en de hulpverlening voor slachtoffers van
  gendergerelateerd geweld.
  \end{quote}
\item
  \begin{quote}
  Geweldsincidenten gebaseerd op gender-identiteit, gender-expressie en
  seksualiteit worden centraal geregistreerd.
  \end{quote}
\end{enumerate}

\textbf{BES-EILANDEN EN KONINKRIJK}

Bonaire, St. Eustatius en Saba (de BES-eilanden) en Aruba, Curaçao en
St. Maarten zijn officieel sinds 1954 geen koloniën van Nederland meer
en worden sinds 1975 niet meer zo genoemd in officiële
overheidsdocumenten. Toch heeft de relatie tussen Nederland en de
overzeese eilanden van het Koninkrijk nog steeds een koloniaal karakter.
De Nederlandse staat heeft zich immers verrijkt door de koloniale
relaties, en de politiek (van links tot rechts) doet tot nu toe weinig
om deze relatie te veranderen.

BIJ1 pleit voor een einde van de ongelijkwaardige relatie tussen de
besturen en bewoners van de eilanden en die van Nederland. BIJ1 staat
voor zeggenschap en zelfbeschikkingsrecht voor inwoners: niet Nederland
beslist, maar mensen op de eilanden zelf. BIJ1 strijdt tegen de
structuren die de eilanden beperken om zelf beslissingen te nemen en
vóór een nieuwe verstandhouding tussen Nederland en de eilanden. De
nieuwe relatie moet worden gebaseerd op antikolonialisme en op het
herstel van koloniale schade. Om dit te kunnen bewerkstelligen stellen
wij de volgende beleidsveranderingen voor.

\textbf{Zelfbeschikkingsrecht en autonomie}

\begin{enumerate}
\def\labelenumi{\arabic{enumi}.}
\item
  \begin{quote}
  St. Eustatius, Bonaire en Saba worden verlost van de grip van politiek
  en ambtelijk Den Haag. De lokale bevolking beslist weer over de
  eilanden en niet de ondemocratisch gekozen ambtenaren van het
  ministerie van Binnenlandse Zaken en Koninkrijksrelaties. Er wordt
  geen beleid voor de eilanden bedacht door mensen die er niet vandaan
  komen of zelf wonen.
  \end{quote}
\item
  \begin{quote}
  De rijksministerraad, waarin zeventien Nederlandse ministers en
  slechts drie gevolmachtigde ministers van Aruba, Curaçao en St.
  Maarten plaats hebben, wordt afgeschaft. Deze oneerlijke representatie
  wordt vervangen door een nieuwe constructie voor Koninkrijk-breed
  overleg. Per Koninkrijk-breed dossier komen de belanghebbende
  ministers van alle landen bijeen voor overleg en hebben ze een gelijke
  stem.
  \end{quote}
\item
  \begin{quote}
  Er komen excuses voor koloniale exploitatie en uitbuiting, de
  slavenhandel en slavernij en de daaropvolgende periode van
  ongelijkwaardige inkomsten, bestuur en behandeling. Deze excuses zijn
  gekoppeld aan de afschaffing van de infrastructuur die het koloniale
  gedachtegoed en witte superioriteitsdenken voortzet in het beleid van
  de Nederlandse staat ten opzichte van het Koninkrijk en de rest van de
  wereld.
  \end{quote}
\item
  \begin{quote}
  Alle landen binnen het Koninkrijk behouden hun interne autonomie voor
  het invullen van hun samenlevingen. Nederland zal niet meer vanuit
  Europa dicteren hoe de eilanden bestuurd moeten worden.
  \end{quote}
\item
  \begin{quote}
  Bij onderlinge geschillen binnen het Koninkrijk zal er een
  geschillencommissie opgesteld worden waarin op gelijke schaal
  onafhankelijke expertise wordt opgenomen uit het Caribisch gebied,
  Zuid Amerika en Europa. Nederlandse ambtenaren of voormalige
  overheidsfunctionarissen worden niet langer gezien als onafhankelijk
  van de Nederlandse staat.
  \end{quote}
\item
  \begin{quote}
  De autonomie binnen de overzeese landen van het Koninkrijk wordt
  verder uitgebouwd en ontwikkeld door de parlementen het recht te geven
  om wetsvoorstellen voor rijkswetten die hen aangaan in te dienen.
  \end{quote}
\end{enumerate}

\textbf{Het herstellen van de schade}

\begin{enumerate}
\def\labelenumi{\arabic{enumi}.}
\item
  \begin{quote}
  Er komt een parlementaire enquête die in kaart brengt wat de omvang is
  van de zelfverrijking van Nederland ten koste van de eilanden tijdens
  de koloniale periode. De conclusies van dit onderzoek zullen onderdeel
  worden van lespakketten in alle lagen van het onderwijs.
  \end{quote}
\item
  \begin{quote}
  De bevolking op de eilanden bepaalt zelf de definitie van
  `herstelbetalingen' en `gerechtigheid' voor wat hen door de eeuwen
  heen is aangedaan. Dit houdt in dat `herstelbetalingen' zoals genoemd
  ook wat anders kan inhouden dan alleen geld.
  \end{quote}
\item
  \begin{quote}
  De Nederlandse staat betaalt herstelbetalingen aan de eilanden voor de
  gemiste inkomsten door financiële beleidsconstructies van vóór 1954.
  \end{quote}
\item
  \begin{quote}
  Nederland betaalt herstelbetalingen voor de ecologische afbraak van de
  gebieden door koloniale uitbuiting.
  \end{quote}
\end{enumerate}

\textbf{Gelijkwaardigheid in onderwijs}

\begin{enumerate}
\def\labelenumi{\arabic{enumi}.}
\item
  \begin{quote}
  Nederland betaalt herstelbetalingen voor de achtergestelde
  ontwikkeling van de onderwijssystemen op de eilanden. Bij het
  ontwikkelen van het onderwijs krijgen de eilanden de ruimte om ook
  Caribische en Zuid-Amerikaanse perspectieven op het onderwijs mee te
  nemen. Nederland is niet vanzelfsprekend meer het ankerpunt voor de
  onderwijssystemen op deze eilanden.
  \end{quote}
\item
  \begin{quote}
  Academische kennisuitwisseling binnen het Koninkrijk wordt
  gestimuleerd.
  \end{quote}
\item
  \begin{quote}
  Het vrije verkeer van personen binnen het Koninkrijk blijft bestaan.
  Studenten die vanaf de eilanden komen voor hun vervolgonderwijs in
  Nederland krijgen structurele ondersteuning vanuit de gemeenten waar
  zij gaan wonen. Eenmaal in Nederland hebben wij de plicht om voor elke
  student te zorgen.
  \end{quote}
\item
  \begin{quote}
  Toekomstige studenten zullen, net als Nederlandse studenten, geen
  lening af hoeven te sluiten om in Nederland of op de eilanden te
  studeren. Er zullen geen studenten aan hun loopbaan beginnen met een
  schuld.
  \end{quote}
\end{enumerate}

\textbf{DEFENSIE}

BIJ1 staat voor een defensiebeleid dat gebaseerd is op het beschermen
van onze belangrijkste waarden: de waarden van een sociale samenleving,
die is gebouwd op gelijkwaardigheid, vrijheid en rechtvaardigheid. Dat
betekent dat wij niet willen investeren in oorlog en geweld. In plaats
van militaire acties zetten wij in op hulp die gedreven wordt door
solidariteit met slachtoffers van geweld en hun behoeftes. Daarbij
streven we naar deëscalatie en ontwapening. We maken een eind aan de
huidige neokoloniale bemoeienis omwille van handelsbelangen en
geldstromen. Zelfbeschikkingsrecht van volkeren en steun voor
anti-imperialistisch verzet zijn belangrijke waarden voor ons
defensiebeleid.

\textbf{Oorlog de wereld uit}

Oorlogsvoering is een kwaadaardig machtsmiddel. Door middel van militair
ingrijpen willen Westerse landen en hun industrieën invloed krijgen in
andere landen voor hun eigen gewin. Dit gebeurde in de koloniale
Nederlandse geschiedenis, maar ook nu nog. Westerse landen voeren al
tientallen jaren oorlog om hun afzetmarkten te vergroten en hun
politieke, economische en militaire overmacht uit te breiden. Hun honger
naar grondstoffen en winst drijft hen hiertoe.

Nederland laat zich met het oog op handelsbelangen meeslepen in
oorlogen. Het Westen zegt deze oorlogen te beginnen om vrijheid,
democratie en mensenrechten te brengen in gebieden waar die nog niet
zijn. Maar deze zelfde grondrechten worden bij ons en onze bondgenoten
nog altijd bedreigd en ingedamd. Deze oorlogen gaan ten koste van vele
mensenlevens.

We willen investeren in de bescherming van onvoorwaardelijke vrijheid,
zelfbeschikking van volkeren, deëscalatie en ontwapening, om zo een
einde maken aan oorlogen en de bijbehorende industrieën.

\textbf{Vrede, zelfbeschikking en internationale rechtvaardigheid}

\begin{enumerate}
\def\labelenumi{\arabic{enumi}.}
\item
  \begin{quote}
  Nederland trekt zich terug uit alle internationale conflicten en
  steunt verzet tegen imperialistische uitbuiting en onderdrukking.
  \end{quote}
\item
  \begin{quote}
  Nederland zoekt toenadering tot landen die hun internationaal beleid
  baseren op vrede, internationale rechtvaardigheid en
  zelfbeschikkingsrecht van volkeren.
  \end{quote}
\item
  \begin{quote}
  Wie oorlog voert onder valse voorwendselen, krijgt te maken met
  politieke en economische gevolgen. We dwingen herstel af van
  humanitaire, infrastructurele en psychologische schade. Er wordt
  humanitaire hulp geboden aan bewoners van landen waar oorlog wordt
  gevoerd op onterechte gronden. Wij maken een onderscheid tussen de
  politieke regimes en de bewoners van een land.
  \end{quote}
\item
  \begin{quote}
  We krimpen onze krijgsmacht in om uiteindelijk de krijgsmacht te
  vervangen door een civiele hulporganisatie. In het geval van een
  noodsituatie bevorderen we de uitwisseling van gespecialiseerde hulp
  tussen landen.
  \end{quote}
\item
  \begin{quote}
  Nederland trekt zich terug uit de imperialistische NAVO en neemt niet
  meer deel aan NAVO-oorlogen. De door de NAVO verplicht gestelde
  uitgavengroei aan defensiedoeleinden wordt ongedaan gemaakt. De
  vrijgekomen gelden gaan sociale doeleinden dienen.
  \end{quote}
\item
  \begin{quote}
  Nederland handelt bij voorkeur in VN-verband, met wederopbouw,
  vredesbewaring en vredesbewaking als oogmerk. Deze operaties mogen
  niet dienen voor het opzetten van scheve handelsrelaties tussen het
  betreffende land en het Nederlands bedrijfsleven. Aan noodhulp worden
  geen voorwaarden verbonden.
  \end{quote}
\item
  \begin{quote}
  Nederland neemt niet deel aan een Europees leger en behoudt de
  zelfbeschikking over eigen troepen.
  \end{quote}
\item
  \begin{quote}
  Er komen ruimhartige genoegdoening en herstelbetalingen aan
  slachtoffers van Nederlandse acties overzee.
  \end{quote}
\item
  \begin{quote}
  We stellen stevige straffen in voor huurlingen die meevechten in
  buitenlandse oorlogen. We monitoren alle uitreizigers die zich
  aansluiten bij gewelddadige groepen in het buitenland, waaronder het
  Israëlisch leger.
  \end{quote}
\item
  \begin{quote}
  Strijders die zich aansluiten bij buitenlandse strijd ter verdediging
  van de waarden van vrijheid en zelfbeschikking van volkeren worden
  beschermd.
  \end{quote}
\end{enumerate}

\textbf{Ontwapening en omvorming wapenindustrie}

\begin{enumerate}
\def\labelenumi{\arabic{enumi}.}
\item
  \begin{quote}
  Kernwapens de wereld uit. Om te beginnen uit Nederland. Nederland
  sluit zich aan bij wapenbeheersings- en ontwapeningsverdragen en toont
  zich ambassadeur van deze verdragen.
  \end{quote}
\item
  \begin{quote}
  We maken een einde aan Nederland als belastingparadijs voor
  wapenproducenten en de industrie van militaire technologieën. We
  verbieden de vestiging van wapenbedrijven in Nederland en heffen hoge
  belastingen voor de bestaande wapenindustrie, zolang deze niet is
  omgevormd.
  \end{quote}
\item
  \begin{quote}
  We stimuleren de inzet van ontwikkelde technische kennis in de
  wapenindustrie voor de civiele industrie.
  \end{quote}
\item
  \begin{quote}
  We maken een eind aan (door)levering van wapens en militaire
  technologie aan landen of groeperingen die mensenrechten schenden.
  \end{quote}
\item
  \begin{quote}
  Alle niet geleverde straaljagers (JSF's) en onderzeeërs worden
  afbesteld. Wat al geleverd is wordt gedemonteerd en onderdelen worden
  gebruikt voor andere technische en civiele doeleinden.
  \end{quote}
\item
  \begin{quote}
  Er worden geen militairen ingezet tegen mensen, dus ook niet tegen
  vluchtelingen en migranten. De Marechaussee wordt onderdeel van de
  politie. De Nationale Reserve wordt omgevormd in een (binnenlandse)
  civiele rampenorganisatie.
  \end{quote}
\item
  \begin{quote}
  In het basis- en voortgezet onderwijs komt er blijvend aandacht voor
  vredeseducatie. Er zullen geen reclames meer zijn om jongeren aan te
  sporen te gaan werken bij defensie.
  \end{quote}
\end{enumerate}

\textbf{Een veilige en rechtvaardige werkomgeving}

\begin{enumerate}
\def\labelenumi{\arabic{enumi}.}
\item
  \begin{quote}
  We dragen zorg voor een goede (geestelijke) gezondheidszorg voor
  krijgsmachtpersoneel en veteranen.
  \end{quote}
\item
  \begin{quote}
  Er wordt ingezet op de bestrijding van racisme, islamofobie,
  antisemitisme, seksisme, seksueel en gendergerelateerd geweld,
  machogedrag en andere vormen van onderdrukking binnen de organisatie
  van de krijgsmacht.
  \end{quote}
\item
  \begin{quote}
  Mensen met extreemrechtse opvattingen mogen geen deel uitmaken van de
  krijgsmacht. De krijgsmacht moet doordrongen zijn van democratische
  waarden. Wie extreemrechtse opvattingen erop nahoudt, staat lijnrecht
  tegenover deze waarden.
  \end{quote}
\item
  \begin{quote}
  Wie weigert mee te vechten in overzeese oorlogen wordt niet bestraft.
  Er komt (postuum) eerherstel voor dienstweigeraars uit het verleden.
  \end{quote}
\item
  \begin{quote}
  De opgeschorte opkomstplicht wordt veranderd in een algehele opheffing
  van de dienstplicht. Er komt geen instelling van een sociale of
  maatschappelijke dienstplicht.
  \end{quote}
\end{enumerate}

\textbf{DIGITALE RECHTEN EN TECHNOLOGIE}

BIJ1 strijdt voor online en offline privacy, net als voor de vrijheid om
je door de publieke ruimte te bewegen zonder dat elke beweging
geregistreerd en opgeslagen wordt. Wij zien het internet als publieke
ruimte, die zou moeten worden ingericht om publiek belang te dienen in
plaats van commercieel belang. Digitale communicatiemiddelen moeten voor
iedereen toegankelijk en veilig zijn, waarbij onze gegevens goed worden
beschermd. En bovenal moet de digitalisering van overheidsdiensten op
een eerlijke manier gebeuren, waarbij discriminatie en etnisch
profileren verbannen wordt. Zo werken we toe naar een Nederland waar
digitale technologieën bijdragen tot gelijke kansen en sociale
inclusiviteit.

\textbf{De risico's van digitale technologie}

Digitale technologie is overal: onze broekzak, ons huis en onze straat.
Dat levert gemak en voordelen op, maar ook risico's. Ons online en
offline gedrag wordt gevolgd door bedrijven en overheidsinstanties. Onze
gegevens worden opgeslagen, verwerkt en ingezet om voorspellingen te
maken over leningen, uitkeringen en criminaliteit. Wie we zijn en wat we
doen wordt soms openbaar zichtbaar, omdat overheidsinstanties of
bedrijven onze gegevens niet goed genoeg beschermen. Dat is
onacceptabel. Daarnaast worden onze gegevens gebruikt voor commerciële
doeleinden, zonder dat we daar toestemming voor geven -- of we worden
gedwongen toestemming te geven omdat we anders geen toegang kunnen
krijgen tot internetdiensten. Bedrijven als Google en Facebook zijn
hierdoor supermachten geworden die beschikken over enorme massa's aan
data. Zij zijn bovendien niet of nauwelijks terug te fluiten als zij
privacy of mensenrechten schenden. Omdat deze technologieën zo nieuw
zijn, bestaat er nauwelijks wetgeving of beleid voor binnen Europa. Het
is hoog tijd dat de macht van technologie-giganten gebroken wordt.

\textbf{Bevooroordeelde algoritmes en discriminerend beleid}

Scholen, ziekenhuizen, de Belastingdienst, het UWV, de politie en andere
overheidsinstanties maken allemaal gebruik van digitale technologieën om
onze gegevens op te slaan, te verwerken en te analyseren door middel van
algoritmes. Dat kan heel nuttig en belangrijk zijn, Te vaak wordt met
algoritmes eerst op de kwetsbaarsten in de samenleving geëxperimenteerd.
Waarom wordt er wel grootschalig geïnvesteerd in het opsporen van
bijstandsfraude door algoritmes maar niet in het het detecteren van
witteboordencriminaliteit bij banken en grote bedrijven? Dat is
oneerlijk en werkt etnisch profileren en klassenprofilering in de hand.
Ook de technici die nieuwe technologie en algoritmes ontwikkelen zijn
zelf niet altijd vrij van vooroordelen en kunnen deze, al dan niet
bewust, in hun algoritmes verwerken, waardoor een rekenmodel zelf
racistisch of discriminerend kan zijn.

\begin{quote}
\emph{Als de politie voornamelijk (criminaliteits)gegevens over
gemarginaliseerde wijken invoert in algoritmes en veel minder uit andere
wijken, zullen die algoritmes een vertekend beeld van de werkelijkheid
voorspellen, segregatie in de hand werken en gemarginaliseerde wijken
nog meer stigmatiseren. Door de manier waarop algoritmes in elkaar
zitten zal dit proces steeds worden herhaald. Zo worden discriminatie,
vooroordelen en etnisch profileren dieper en dieper in computermodellen
verwerkt. Computermodellen waar de overheid blind op vertrouwt in het
beoordelen van uitkeringen en toeslagen, of het voorspellen van
criminaliteit.}
\end{quote}

Discriminatie en etnisch profileren in de digitalisering van
overheidsdiensten moeten daarom nu in de kiem gesmoord worden. Nederland
heeft hier nu nog niet de middelen voor. De Autoriteit Persoonsgegevens
die toeziet op de naleving van de Algemene Verordening
Gegevensbescherming heeft niet het mandaat of de middelen om (mogelijke)
discriminatie in algoritmes en gegevensverwerking te bestrijden en
bestraffen.

Om meer digitale veiligheid en rechtvaardigheid te waarborgen neemt BIJ1
de volgende maatregelen.

\textbf{Toegankelijkheid en bescherming voor iedereen}

\begin{enumerate}
\def\labelenumi{\arabic{enumi}.}
\item
  \begin{quote}
  Toegang tot het internet is een basisrecht en moet daarom voor
  iedereen toegankelijk zijn. Er komt gratis internettoegang voor
  iedereen.
  \end{quote}
\item
  \begin{quote}
  Er wordt ingezet op het verder ontwikkelen van educatieprogramma's
  voor alle Nederlandse inwoners om hen meer bewust te maken van hun
  digitale rechten en privacyrechten.
  \end{quote}
\item
  \begin{quote}
  Toegankelijkheidseisen moeten worden meegenomen in
  digitaliseringsprocessen, waarbij ervaringen van mensen met een
  beperking worden meegenomen. Alternatieve kanalen zoals telefoon of
  loket moeten open blijven voor mensen die moeite hebben met digitale
  technologieën, zoals ouderen, laaggeletterden of mensen met een
  verstandelijke beperking.
  \end{quote}
\item
  \begin{quote}
  De overheid moet de ontwikkeling van \emph{open source} en open
  standaarden in de publieke en private sector faciliteren.
  \end{quote}
\end{enumerate}

\textbf{Privacy en data van gebruikers beschermen}

\begin{enumerate}
\def\labelenumi{\arabic{enumi}.}
\item
  \begin{quote}
  We streven naar wetgeving en beleid op Nederlands en Europees niveau
  om het internet officieel te definiëren als nutsvoorziening in plaats
  van een door bedrijven gedomineerde marktplaats.
  \end{quote}
\item
  \begin{quote}
  Er komt een verbod op de ongevraagde verkoop van persoonlijke
  informatie.
  \end{quote}
\item
  \begin{quote}
  We zetten ons in op Nederlands en Europees niveau om de macht van
  technologie-giganten als Facebook en Google aan banden te leggen en de
  burger hiertegen te beschermen.
  \end{quote}
\item
  \begin{quote}
  Er wordt een verbod ingesteld op gezichtsherkenningssoftware en
  soortgelijke software die mensen herkent op basis van lichaamshouding
  of uiterlijke kenmerken.
  \end{quote}
\end{enumerate}

\begin{enumerate}
\def\labelenumi{\arabic{enumi}.}
\setcounter{enumi}{4}
\item
  \begin{quote}
  End-to-end encryptie blijft gewaarborgd. Dit is het digitale
  briefgeheim: beveiligde communicatie tussen twee personen zonder dat
  een derde partij hier toegang toe heeft. Dit is een onmisbare
  bescherming van het recht op privacy en veilige communicatie.
  \end{quote}
\item
  \begin{quote}
  Het uitbreiden van cameratoezicht in de openbare ruimte wordt
  stopgezet.
  \end{quote}
\end{enumerate}

\textbf{Het gebruik van data beveiligen en controleren}

\begin{enumerate}
\def\labelenumi{\arabic{enumi}.}
\item
  \begin{quote}
  Er komt actief overheidsbeleid tegen discriminatie en etnisch
  profileren in de digitalisering van overheidsdiensten -- van het
  ontwerp van algoritmes tot aan het evalueren van
  digitaliseringsprocessen. Dit kan alleen als de mensen die deze
  trajecten begeleiden de verschillende bevolkingsgroepen in Nederland
  vertegenwoordigen.
  \end{quote}
\item
  \begin{quote}
  Er komt een onafhankelijke toezichthouder die toeziet op de strijd
  tegen discriminatie, etnisch profileren en het schenden van
  mensenrechten in dataverzameling over burgers door de overheid. Deze
  toezichthouder heeft het mandaat om deze data op te vragen, te
  beoordelen en wettelijk bindend advies te geven aan
  overheidsinstanties.
  \end{quote}
\item
  \begin{quote}
  We stellen scherp toezicht en een mensenrechtentoets in voor de export
  van surveillance-software door Nederlandse bedrijven. In geen geval
  mag Nederlandse surveillance-software bijdragen aan
  mensenrechtenschendingen in het buitenland.
  \end{quote}
\item
  \begin{quote}
  De Wet op de inlichtingen- en veiligheidsdiensten (Wiv) moet
  mensenrechten beter gaan waarborgen. De Nederlandse inlichtingen- en
  veiligheidsdiensten moeten stoppen met het uitwisselen van
  niet-geëvalueerde bulkdata met alle buitenlandse inlichtingsdiensten.
  \end{quote}
\item
  \begin{quote}
  Er komt een digitaliseringsbeleid waarbij de rechten en behoeften van
  de burger centraal staan, en niet de digitalisering op zich.
  \end{quote}
\end{enumerate}

\textbf{WOORDENLIJST\\
~\\
Aangiftebereidheid:} De bereidheid van iemand om aangifte te doen,
meestal direct verbonden met of de persoon zich veilig en serieus
genomen voelt.

\textbf{Algoritme}: Een algoritme is een formule in de vorm van code om
een wiskundig of informaticaprobleem automatisch op te lossen.

\textbf{Anti-democratische landen:} Landen waar de democratie,
zelfbeschikking en mensenrechten van individuen en groepen (ernstig)
onder druk staan dan wel in gebrekkige vorm aanwezig zijn.

\textbf{Anti-imperialistisch:} Tegen koloniale en kapitalistische
invloeden, bezettingen en structuren.

\textbf{Associatieverdrag Israël:} Een akkoord tussen de EU en Israël,
waarmee de internationale samenwerking met Israël is vastgesteld en dat
Israël vrijstelt van invoerheffingen in de EU. Associatieovereenkomsten
die de EU met landen afsluit moeten gebaseerd zijn op respect voor de
mensenrechten en op democratische beginselen. Hun interne en
buitenlandse politiek moet hierdoor geleid worden. Dat staat ook vermeld
in artikel 2 van de overeenkomst met Israël, waar Israël zich dus niet
aan houdt.

\textbf{Biodiversiteit:} De diversiteit van flora en fauna binnen een
bepaald ecosysteem.

\textbf{Biomassacentrales:} Centrales waar (afval)hout wordt verbrand om
energie op te wekken. Helaas lang niet altijd met de garantie dat dit
hout duurzaam verkregen is.

\textbf{Blackfacing:} Het opmaken van het gezicht met zwarte/bruine
schmink en eventueel stereotyperende kenmerken om zo een racistische
karikatuur van een zwart persoon uit te beelden.

\textbf{Ceta/EU-Mercosur/TTIP:} Internationale handelsverdragen waar de
EU bij betrokken is. Deze gaan veelal gepaard met grote schade aan
natuur en klimaat, maar ook de rechten van arbeiders en dierenrechten
worden nauwelijks tot niet meegenomen in deze verdragen.

\textbf{Circulaire landbouw}: Ook wel kringlooplandbouw. Een vorm van
landbouw gebaseerd op het idee dat alles optimaal gebruikt wordt. De
resten van de ene keten zijn de grondstoffen voor een andere keten. Zo
wordt bijvoorbeeld voedsel dat wij niet meer eten als diervoeder
gebruikt. Om tot zo'n circulair landbouwsysteem te komen, hebben we een
transitie nodig waarin plantaardige en dierlijke productieketens slim
aan elkaar worden geknoopt.

\textbf{Cisgender:} Iemand waarbij de genderidentiteit overeenkomt met
het geslacht dat is aangewezen bij de geboorte.

\textbf{Claimrecht:} Het recht van bedrijven om Nederland aan te klagen
als zij winsten mislopen vanwege een besluit dat de Nederlandse overheid
heeft genomen.

\textbf{Code Diversiteit en Inclusie:} Een richtlijn voor de culturele
sector om te verbeteren op het gebied van diversiteit. Oorspronkelijk
vooral gericht op culturele diversiteit, maar inmiddels richt het zich
op een breder spectrum, zoals onder andere gender, religie,
sociaal-economische status en leeftijd.

\textbf{Consent:} Wederzijdse toestemming op basis van het recht op
zelfbeschikking en communicatieve vaardigheden om deze wederzijdse
toestemming van begin tot eind te kunnen geven. Deze toestemming kan
alleen vanuit vrijheid gegeven worden, en met de mogelijkheid om van
gedachten te kunnen veranderen.

\textbf{Cultuursensitief:} Handelen met respect voor, en uit het bewust
zijn van de normen en waarden die bij een bepaalde cultuur horen.

\textbf{Dekolonisatie:} Het proces waarin gekoloniseerde landen zich
ontdoen van de koloniale machthebbers, maar ook de processen waarin de
samenleving afstand neemt van de witte en/of westerse norm als de
universele waarheid.

Deprivatiseren: Iets uit privé-bezit halen en tot publiek bezit maken.

\textbf{Discriminatie:} Het onrechtmatig uitsluiten en niet
gelijkwaardig behandelen van en/of onderscheid maken tussen
verschillende mensen en/of groepen.

\textbf{Doelmatigheidskorting:} Een vorm van bezuinigingen in het
onderwijs. Dit is een gevolg van tekorten op de onderwijsbegroting in
een eerdere kabinetsperiode. De `doelmatigheidskorting' van het
onderwijs bedraagt per 2021 in totaal 183 miljoen euro. Het
basisonderwijs zal voor bijna 61 miljoen worden gekort, het voortgezet
onderwijs voor ruim 47,3 miljoen.

\textbf{Dublin-verordening:} Een Europese verordening die sinds januari
2014 van kracht is. Deze verordening bepaalt welk land verantwoordelijk
is voor de behandeling van een asielaanvraag. Meestal is dat het land
waar de vluchteling aankomt: in de praktijk betekent dit dat Italië en
Griekenland verreweg de meeste asielaanvragen te verwerken hebben. Deze
verdeling is oneerlijk en heeft veel schadelijke gevolgen.

\textbf{Ecosysteem:} Een systeem van levende wezens dat samen met hun
omgeving een keten vormt die het leven binnen het systeem in stand
houdt.

\textbf{Ervaringsdeskundige:} Iemand die eigen ervaringen gebruikt in
zijn werk en zo een specifieke vorm van expertise heeft.

\textbf{Etniciteit:} Een verzameling van kenmerken die onderdeel zijn
van iemands achtergrond of identiteit, zoals nationaliteit, taal,
cultuur, religie, etc.

\textbf{Etnisch profileren:} Het hanteren van (uiterlijke) eigenschappen
als huidskleur, nationaliteit, etniciteit, taal of geloof bij opsporing
of handhaving, terwijl daar geen bewezen aanleiding voor is.

\textbf{Eurocentrisme:} Het, al dan niet bewust, benadrukken van (het
perspectief van) Europa, en in het algemeen de westerse ideeën en
theorieën. Hierbij wordt geen rekening gehouden met de invloeden van
andere culturen.

\textbf{Europees Handvest:} Document waarin de grondrechten van burgers
binnen de Europese Unie worden beschreven.

\textbf{Europese Commissie:} Het overheidsorgaan van de Europese Unie,
verantwoordelijk voor onder andere de Europese begroting en het indienen
van en stemmen over Europese wetgeving.

\textbf{Farmaceuten:} De makers/verkopers van geneesmiddelen.

\textbf{Fair Practice Code:} Een gedragscode voor ondernemen en werken
binnen de kunst, cultuur en creatieve sector met als doel het
bewerkstelligen van een eerlijke en solidaire bedrijfsvoering.

\textbf{Flexibilisering van de arbeidsmarkt:} De trend waarin vaste
banen en daarmee zekerheid steeds vaker worden vervangen door `flexibele
contractvormen' zoals uitzendcontracten, contracten op oproepbasis of
tijdelijke contracten.

\textbf{FRONTEX:} \emph{European Border and Coast Guard Agency}. Frontex
is een agentschap vanuit samenwerking van Europese lidstaten en bewaakt
de Europese buitengrenzen.

\textbf{Gendergerelateerd geweld:} Geweldpleging tegen iemand vanwege
hun gender(identiteit).

\textbf{Genderexpressie:} Het uiten van het gevoel gekoppeld aan iemands
genderidentiteit.

\textbf{Genderidentiteit:} De wijze waarop iemand zich identificeert op
het gebied van diens gender.

\textbf{Gelijke behandelingsrichtlijn:} Europese richtlijn bestaande uit
de toezegging van landen om zich volledig in te zetten op het gebied van
bestrijding van alle vormen van uitsluiting en discriminatie.

\textbf{Green Deal:} Een wetsvoorstel waarmee grote stappen worden gezet
naar een duurzame en circulaire economie.

\textbf{Grootkapitaal:} De groep mensen en instanties met de grootste
financiële macht.

\textbf{Hervestiging:} Als de meest kwetsbare vluchtelingen niet kunnen
terugkeren naar hun eigen land en ze ook niet kunnen blijven in hun
opvangland, dan komen ze in aanmerking voor hervestiging in Nederland.
Dit komt bijvoorbeeld voor wanneer vluchtelingen extra gevaar lopen, of
wanneer kampen of opvanglanden niet genoeg voorzieningen hebben om te
kunnen omgaan met een complexe medische aandoening of handicap.

\textbf{IHRA-definitie:} Door de Israël-lobby wordt een poging gedaan om
de gangbare definitie van antisemitisme aan te passen door erin op te
nemen dat vormen van kritiek op de staat Israël daaronder kunnen vallen.
Deze pogingen om de IHRA-definitie aangenomen te krijgen door regeringen
en parlementen gaan niet toevallig samen met campagnes om alle kritiek
op Israël te delegitimeren.

\textbf{Inclusiviteit:} Het actief in de samenleving betrekken van
achtergestelde groepen op basis van gelijkwaardige rechten en plichten.

\textbf{Indexeren:} Het aanpassen van de hoogte van een inkomen of
uitkering aan de inflatie zodat de koopkracht hetzelfde blijft.

\textbf{Inflatievolgend:} Het aanhouden van inflatie (stijging van het
algemene prijspeil in de economie) als maximale lijn voor bijvoorbeeld
huurverhogingen.

\textbf{Infrastructurele ontwikkelingen:} De aanleg en ontwikkeling van
infrastructuur (bijvoorbeeld wegen, spoorlijnen en
communicatienetwerken)

\textbf{Inspectie SZW:} Overheidsorgaan met als kerntaak het inspecteren
op gebieden als arbeidsomstandigheden en fraude binnen werk en inkomen.

\textbf{Institutioneel:} Verweven in de cultuur en normen van
instituten, bijvoorbeeld overheidsinstellingen.

\textbf{Intersectioneel feminisme:} Bij intersectioneel feminisme gaat
het niet alleen over vrouwzijn, maar gaat het daarnaast om de combinatie
van factoren die je leven ook beïnvloeden en hoe die samenwerken. Het
gaat dan om factoren als gender, ras, klasse, seksualiteit, beperking,
enzovoorts.

\textbf{Kallemooi:} Een traditie op Schiermonnikoog waarbij een levende
haan gedurende drie dagen op 18 tot 20 meter hoogte in een mand wordt
vastgehouden.

\textbf{Kapitaalvlucht:} Het doorsluizen van kapitaal naar het
buitenland, vaak vanwege gunstige belastingtarieven.

\textbf{Kapitalisme:} Ons huidige economische systeem waarin het behalen
van winst en het opbouwen van kapitaal het belangrijkst zijn. De vrije
markt met de bijbehorende voortdurende groei moet daarvoor zorgen. Dit
gebeurt door middel van uitbuiting en gaat ten koste van onder andere
arbeidsrechten, eerlijke verdeling van macht en welvaart, dierenwelzijn
en het klimaat.

\textbf{Klassenjustitie:} Wanneer mensen met meer vermogen of hogere
opleidingen gebaseerd op dat gegeven worden bevoorrecht binnen de
rechtspraak.

\textbf{Kostendelersnorm:} Het verlagen van een bijstandsuitkering op
het moment dat meerdere volwassenen samenwonen op één adres. Dit geldt
ook als de samenwonende volwassenen een familie vormen.

\textbf{Kringlooplandbouw:} Zie Circulaire landbouw.

\textbf{Kwetsbare vluchtelingen:} Groepen vluchtelingen die door hun
achtergrond of (delen van) hun identiteit als bijzonder kwetsbaar worden
aangemerkt, bijvoorbeeld oudere asielzoekers, zwangere vluchtelingen,
alleenstaande vrouwen (met of zonder kinderen), minderjarigen en
slachtoffers van martelingen of seksueel misbruik.

\textbf{LHBTQI+:} Een afkorting voor `Lesbisch, Homoseksueel,
Biseksueel, Trans, Queer, Intersekse' en meer.

\textbf{Liberaal:} Politieke stroming waar de overheid zo min mogelijk
invloed heeft op de invulling van de maatschappij, waar het belang van
bedrijfseigenaren voorop staat, en radicale verandering vrijwel
uitgesloten is.

\textbf{Loondispensatieregeling in de Wajong:} Een financiële maatregel
die werkgevers moet stimuleren meer mensen met een beperking aan te
nemen. Dit gebeurt door mensen met een Wajong-uitkering onder het
minimumloon te betalen; de rest wordt als uitkering verstrekt door het
UWV.

\textbf{Mensenrechtenverdragen:} Gemaakte afspraken tussen landen over
het beschermen van mensenrechten. Denk bijvoorbeeld aan de Universele
Verklaring van de Rechten van de Mens. Ook binnen de Nederlandse en
Europese wetgeving worden deze in verschillende wetten en afspraken
behandeld.

\textbf{Multilateraal:} Samenwerking tussen meerdere partijen.

\textbf{Neokolonialisme:} Systemen waarin rijke (vaak westerse) landen
voormalig gekoloniseerde gebieden nog steeds uitbuiten.

\textbf{Neurodiversiteit:} De van nature aanwezige verschillen en
variaties in het menselijk brein, bijvoorbeeld op het gebied van sociale
interactie of de manier van informatie opnemen. Ook wel een verzamelterm
voor onder andere autisme en dyslexie.

\textbf{NGO:} Een `niet-gouvernementele organisatie', dus een
organisatie die onafhankelijk is van de overheid.

\textbf{Non-binair:} Een non-binair persoon is iemand die zich niet
thuisvoelt in de hokjes `man' of `vrouw'.

\textbf{Nul-op-de-meter-woning:} Een nul-op-de-meter-woning is een
energieneutrale woning: door energie te besparen, en lokaal en duurzaam
energie op te wekken wordt er net zo veel verbruikt als er opgewekt
wordt. Soms wordt er zelfs meer energie opgewekt dan er verbruikt wordt.

\textbf{OESO-richtlijnen:} Richtlijnen die aangeven wat er van
multinationals wordt verwacht op sociaal-maatschappelijk gebied.

\textbf{Onteigeningswetgeving:} De mogelijkheid om eigendom van
particulieren aan de overheid over te dragen.

\textbf{Open source:} Vrije toegang tot de gebruikte bronnen van een
eindproduct.

\textbf{Parlementaire enquête:} Een middel voor de Eerste en Tweede
Kamer om informatie over een specifiek onderwerp te kunnen verkrijgen.

\textbf{Rasrichtlijn:} Europese richtlijn aangaande gelijke behandeling
ongeacht ras of etniciteit.

\textbf{Rechtsherstel:} Compensatie voor de verschillende gevolgen die
bepaalde groepen hebben moeten doorstaan bij bezetting en/of vervolging.

\textbf{Robotisering:} De trend waarin een toenemend aantal taken kan
worden uitgevoerd door geautomatiseerde systemen.

\textbf{Schijnzelfstandigheid:} Iemand die werk uitvoert als
zelfstandige, maar daarbij eigenlijk wel in constante dienst is van een
werkgever en onder diens gezag staat.

\textbf{Segregatie:} Afzondering van bevolkingsgroepen uit de
maatschappij of het scheiden van bevolkingsgroepen. Het
tegenovergestelde van integratie.

\textbf{Sociaal domein:} Alles wat (lokale) overheden doen op het gebied
van participatie, zelfredzaamheid, werk en zorg.

\textbf{Speculatie:} Een huis (of iets anders) wordt in dit geval niet
gekocht om zelf te gebruiken, maar wordt gebruikt voor winst. Veel
huizen komen op die manier leeg te staan en zorgen voor een kunstmatig
woningtekort.

\textbf{Tobintaks:} Een (kleine) belasting over transacties met
buitenlandse valuta (Euro, Pond, Dollar, etc).

\textbf{Turkijedeal:} Een deal tussen de Europese Unie en Turkije over
de opvang van Syrische vluchtelingen in Turkije. Daar stond veel
Europees geld tegenover. Deze deal is gebaseerd op het `ruilen' van
vluchtelingen: elke teruggestuurde vluchteling leidt tot opname van één
Syrische vluchteling in Europa. De deal is in de basis immoreel en
disfunctioneel en heeft onder andere geleid tot het vastzitten van
duizenden vluchtelingen op de Griekse eilanden in mensonterende
omstandigheden.

\textbf{Validisme:} De discriminatie, stigmatisering en/of uitsluiting
van mensen met een lichamelijke en/of verstandelijke beperking en van
neurodivergente mensen.

\textbf{Veiligheidsdriehoek:} De samenwerking tussen drie
personen/instanties (bijvoorbeeld de burgemeester, politie en justitie),
meestal op regionaal niveau.

\textbf{Verhuurderheffing:} Een bedrag dat woningcorporaties en andere
verhuurders in de gereguleerde sector moeten betalen als ze meer dan 50
huurwoningen verhuren. Door deze heffing hebben woningcorporaties vaak
weinig geld om te investeren in bijvoorbeeld het opknappen van woningen
of het bouwen van meer sociale huurwoningen.

\textbf{VN-verdrag inzake rechten van personen met een handicap:} Een
sinds 2016 geldend verdrag waarmee de Nederlandse staat heeft toegezegd
de situatie van mensen met een beperking dusdanig te verbeteren dat zij
volwaardig mee mogen en kunnen doen in de samenleving.

\textbf{Vrijheid van verkeer:} De vrijheid om je te verplaatsen naar
andere landen.\textbf{\hfill\break
Warmtefonds:} Een fonds waaruit woningbezitters, scholen en verenigingen
van eigenaren kunnen lenen om de verduurzaming van hun woningen of
panden te realiseren.

\textbf{Wooncoöperatie:} Een organisatie waarin huurders zeggenschap
hebben over de woningbouw en huurwoningen.

\textbf{Woonplicht:} Het verplichten van huizenkopers om ten minste een
bepaald aantal jaren te wonen in het huis dat ze kopen. Dit is om
speculatie tegen te gaan.

\textbf{Zwijntje-tik:} Een spel waarbij mensen geblinddoekt achter
opgesloten varkens aan zitten om ze te `tikken'.

\textbf{ACHTERGRONDSTUKKEN}

\begin{enumerate}
\def\labelenumi{\arabic{enumi}.}
\item
  \begin{quote}
  \textbf{ZEGGENSCHAP OVER WERK}
  \end{quote}
\end{enumerate}

In een kapitalistische economie wordt van bovenaf gedicteerd hoe je je
werk moet doen. Ook in hoger geklasseerde banen als docent in het hoger
onderwijs ben je niet vrij van productiedwang en hoge werkdruk. BIJ1
vindt dat een goede baan betekent dat je volwaardige zeggenschap over je
werk hebt. In het klein en in het groot. Degene die het werk doet kan
het beste bepalen hoe dat werk gedaan moet worden, altijd in samenspraak
met hen die afhankelijk zijn van het werk. Werknemers moeten zo
zeggenschap hebben over de arbeidsomstandigheden, over investeringen en
over het gehele bedrijfsbeleid.

Bedrijven worden van binnen gedemocratiseerd: alle geledingen binnen het
bedrijf worden verkozen en zijn afzetbaar als zij zich niet houden aan
afspraken die met de werknemers zijn gemaakt. Werknemers kiezen een
representatieve afvaardiging om mee te beslissen en hebben zo directe
zeggenschap over het bedrijfsbeleid. Uiteraard hoort bij een
democratisch bedrijf ook dat de loonkloof tussen de managers, bazen en
medewerkers veel kleiner wordt en winst ten goede komt van de
maatschappij.

\begin{enumerate}
\def\labelenumi{\arabic{enumi}.}
\setcounter{enumi}{1}
\item
  \begin{quote}
  \textbf{GENDERRECHTVAARDIGHEID EN WERK\\
  }
  \end{quote}
\end{enumerate}

Wereldwijd heeft 42\% van de vrouwen geen betaalde baan, tegen 6\% van
de mannen. De reden is dat de tijd van vrouwen veel meer in beslag
genomen wordt door zorgtaken, kinderen en huishoudelijk werk. En het
maakt nogal uit of je thuis een lopende kraan hebt, of dat je elke dag
water moet gaan halen. Over de gehele wereld gerekend doen vrouwen
driekwart van het onbetaalde werk, en tweederde van het betaalde
zorgwerk. Werk dat in het verlengde ligt van wat vrouwen traditioneel
gezien horen te doen: schoonmaken, kinderopvang, en huishoudelijk werk
voor gezinnen die dat kunnen betalen. Werk dus, dat over het algemeen
slecht wordt betaald. Dit betekent dat vrouwen veel minder vaak dan
mannen over vermogen en bezit beschikken en vaak afhankelijk zijn van
een huwelijk omdat ze van eigen inkomsten niet kunnen leven.

\textbf{En in ons welvarende en moderne land dan?}

Wij gaan er graag prat op dat de gelijkwaardigheid tussen vrouwen en
mannen hoort bij onze normen en waarden, waar al die nieuwkomers nog wat
van kunnen leren. Qua wetgeving zijn de gelijke rechten verzekerd. Maar
in werkelijkheid is daar nog weinig sprake van. Vrouwen verdienen
gemiddeld nog steeds minder dan mannen, en zijn minder vertegenwoordigd
in de politiek en in de top van het bedrijfsleven. Kortom, overal waar
er sprake is van macht zijn er minder vrouwen.

De belangrijkste reden dat vrouwen economisch gezien nog steeds
achterblijven is het krijgen van kinderen. Vanaf dat moment gaan de
meeste vrouwen minder werken, en meer tijd besteden aan zorg. Wanneer de
kinderen groot zijn halen ze de achterstand op de arbeidsmarkt niet meer
in. Het is nog steeds zo dat maar weinig mannen minder gaan werken als
ze vader worden.

Het is dus nog steeds vooral een vrouwenprobleem hoe we een balans
vinden tussen werk en zorg. Sommige vrouwen hebben daar minder last van,
omdat ze voldoende verdienen om een deel van de zorgtaak betaald door
anderen te laten doen. Andere vrouwen hebben het extra moeilijk, met
name de half miljoen alleenstaande moeders, die zowel voor inkomen als
voor zorg verantwoordelijk zijn. Het is geen toeval dat een kwart van de
alleenstaande moeders op of onder de armoedegrens leeft.

\textbf{Welk feminisme?}

De beweging die er naar streeft om verandering te brengen in de
ongelijkheid tussen vrouwen en mannen noemen we feminisme. Maar laat dit
meteen gezegd zijn: er waren altijd al verschillende stromingen binnen
de vrouwenbeweging. Er waren feministes die er vooral mee bezig waren om
vrouwen een plaats te geven aan de tafel. Meer vrouwen aan de top, meer
vrouwen in leidinggevende posities. Niet toevallig ging het daarbij
vooral om vrouwen die redelijk hoogopgeleid waren en weinig last hadden
van andere soorten van discriminatie. Maar er waren ook feministes die
beseften dat dit geen feminisme is waarbij alle vrouwen vooruitgaan. Er
waren groepen van zwarte-, migranten- en vluchtelingenvrouwen, ook wel
zmv-vrouwen genoemd. Er waren groepen van lesbische vrouwen die ook de
heteronorm bekritiseerden. Dat was het begin van wat we nu
intersectionaliteit noemen: het weten dat in onze levens altijd klasse,
kleur, sekse en seksualiteit meespelen, zonder dat we dat van elkaar
kunnen scheiden.

Behalve dat, was er als stroming ook het socialistisch feminisme (ook
wel het soc-fem genoemd), dat verder wilde gaan dan het veroveren van
een plekje aan tafel. Het uitgangspunt is dat dit kapitalistische
systeem per definitie onrechtvaardig is, en nooit alle vrouwen een
rechtvaardige gelijkheid kan bieden. En dat lossen we niet op met meer
vrouwen in de Tweede Kamer. Klasse is dus een belangrijke en vaak
verwaarloosde factor. Als feminisme moet gaan over alle vrouwen, dan ook
de vrouwen onderaan de sociale ladder. Wij sluiten ons aan bij het
internationale `feminisme van de 99\%', dat er is voor alle vrouwen.

BIJ1 gaat ervan uit dat ons feminisme vanzelfsprekend intersectioneel
is. Dat wil zeggen dat we ook oog hebben voor klasse, kleur en
seksualiteit, en dat we ons feminisme voor de 99\% alleen verwezenlijken
kunnen wanneer we ook antikapitalistisch zijn en allianties sluiten met
systeemkritische organisaties die ook staan voor een rechtvaardigere
economie.

\textbf{Gelijke beloning, maar gelijk met wie?}

Dit is het punt. Als we het hebben over gelijke beloning, met wie moeten
de vrouwen die te weinig verdienen zich vergelijken? Wil mevrouw de
minister gelijk worden aan haar chauffeur? Of is het logischer dat de
chauffeur graag evenveel zou verdienen als mevrouw de minister? De vraag
is dus: aan welke mannen willen vrouwen gelijk worden? Wat er mist in
die eenvoudige vergelijking - vrouwen horen evenveel te verdienen als
mannen - is de factor klasse. Voor de vrouwen die in de supermarkt
achter de kassa zitten heeft het weinig zin om hun loon te vergelijken
met dat van de mannen die de vakken vullen. Ze verdienen beiden te
weinig. Als we het hebben over gelijkwaardigheid en rechtvaardigheid,
horen we het dus ook altijd te hebben over klasse, en daarmee
samenhangend ook over kleur, etniciteit en migratie.

De belangrijkste bijdrage van feminisme is het besef dat we de wereld
niet meer simpel op kunnen delen in de sector `werk', waar altijd het
betaalde werk onder verstaan wordt, en de sector `thuis', waar het gaat
om de zorg voor ons dagelijkse voortbestaan. Onder het kapitalisme is
het feitelijk één geheel. Vanuit het kapitalisme gezien is niet alleen
de productie van winst belangrijk, maar ook de `productie van mensen' -
het voortbrengen en in stand houden van het leven zelf. Maar omdat de
productie van mensen belangrijk is, maar praktisch geen winst op kan
leveren, wordt dat ons toegeschoven als een puur persoonlijke zaak. En
zo zien we dat we steeds verder worden teruggedrongen in `zoek het maar
uit': als je kinderen hebt is dat je persoonlijke keuze, en als je niet
genoeg verdient om die kinderen goed groot te brengen dan is dat je
eigen schuld.

Wat we kunnen zien, en wat we met de coronacrisis nog eens flink
ingepeperd kregen is dat de gehele samenleving voor een groot deel
afhankelijk is van al dat slecht betaalde werk in de zorg, de
dienstverlening, de verpleging, en de schoonmaak, terwijl tegelijk de
zorg voor kinderen en ouderen weer vrijwel geheel teruggeschoven werd
naar het thuisfront. Daarmee blijkt dat deze wereld vooral in stand
wordt gehouden door vrouwen, en migranten, die daar weinig erkenning en
zeker geen rechtvaardige beloning voor krijgen.

\textbf{Niet om de winst, maar om de mensen}

De belangrijke bijdrage die het feminisme biedt aan de door het marxisme
geïnspireerde systeem-kritiek, is dat het kapitalisme net zo afhankelijk
is van de mensen die de goederen en diensten produceren waar winst op te
maken valt, als van de mensen die onderbetaald of onbetaald ervoor
zorgen dat de volgende generatie mensen klaar staat om het kapitalisme
van producenten en consumenten te voorzien. Dat betekent, voor de
antikapitalistische bewegingen, dat we ons niet alleen kunnen
bezighouden met betere arbeidsomstandigheden, en een hoger loon voor
alle werkenden, maar ook met de leefomstandigheden van mensen die in de
zorg werken, of die geen werk hebben.

We streven naar een eerlijkere verdeling van zorg en werk, waarbij het
niet de vrouwen en de migranten zijn die altijd weer in de minst
voordelige hoek terecht komen. Het is op zich al een schande dat nog
steeds maar de helft van de vrouwen niet voldoende verdient om
zelfstandig van te kunnen leven. Die eerlijkere verdeling van zorg en
werk zou niet een vrouwenprobleem moeten blijven. Niet alleen verwachten
we van mannen dat ze hun deel van de zorg op zich nemen, we gaan er ook
vanuit dat er een oneerlijke verdeling is tussen vrouwen onderling. Het
zijn vooral vrouwen met minder opleiding die tegen een laag loon een
deel van de huishouding en zorg voor de kinderen overnemen van vrouwen
die het zich daardoor kunnen veroorloven om een goede baan te hebben.

Het is dus ook duidelijk dat juist het werk dat in het verlengde ligt
van huishoudelijk werk in onze samenleving het slechtst betaald wordt:
veel werk in de zorg en de dienstverlening. Daarmee blijft het
grotendeels vrouwenwerk, en daarom blijft het ook slecht betaald worden:
een vicieuze cirkel. Het betekent ook dat er op de zorg als eerste
bezuinigd wordt: de staat kan erop rekenen dat veel van de zorg voor
ouderen en kinderen toch wel gedaan zal worden, omdat je die nu eenmaal
niet aan hun lot overlaat als er geen kinderopvang is, of als de
verpleeghuizen minder ouderen opnemen, of als er te weinig thuiszorg
beschikbaar is. We hebben daar een aardige term voor bedacht: die heet
`mantelzorg'.

Het betekent ook dat we internationaal moeten denken: er is een
migratiestroom gaande waarbij vrouwen uit landen met meer armoede naar
onze welvarende landen worden gehaald om hier als huishoudelijke hulp
aan de gang te gaan, vaak om op onze kinderen te passen zodat ze geld
naar huis kunnen sturen voor hun eigen kinderen. Ook moeten we ons
realiseren dat we medeplichtig zijn aan het uitbuiten van vrouwen in
andere landen die voor onze kledingindustrie tegen een veel te laag loon
T-shirts en spijkerbroeken in elkaar zetten.

Het spreekt voor zich dat we met deze feministische visie geen simpele
tweedeling kunnen maken tussen vrouwen en mannen. Waar we te maken
hebben met migranten en vluchtelingen zijn het niet alleen vrouwen die
onze steun verdienen. Waar we het hebben over schoonmakers zijn het ook
mannen, vaak mannen van kleur of met een migrantenachtergrond, die
worden achtergesteld en te weinig verdienen.

\begin{enumerate}
\def\labelenumi{\arabic{enumi}.}
\setcounter{enumi}{2}
\item
  \begin{quote}
  \textbf{VAN ARMOEDEBESTRIJDING NAAR ARMOEDEPREVENTIE}
  \end{quote}
\end{enumerate}

We horen regelmatig dat het goed gaat met de economie. Daar wordt niet
bij verteld wie van het herstel van de economie profiteert en wie niet.
De economie is de afgelopen decennia met 40\% gegroeid. In diezelfde
periode is het aantal werkende mensen dat in armoede leeft anderhalf
keer zo groot is geworden. Sinds de crisis van 2008 blijkt er sprake te
zijn van structurele welvaartsvermindering voor een groot deel van de
bevolking. Dit gebeurt in een land waar de woningnood weer terug is op
het niveau van 1945 en daarnaast de sociale voorzieningen zo ver zijn
uitgekleed dat je nauwelijks meer van een verzorgingsstaat kunt spreken.

Dat het goed gaat met de economie zegt dus niets, zolang er niet bij
verteld wordt hoe de verworven welvaart wordt verdeeld. Het is duidelijk
dat vooral de rijke bovenlaag het meeste profiteert. In Nederland heeft
de rijkste 1\% van de bevolking meer dan een kwart van het totale
vermogen in bezit.

Eens was het minimumloon gekoppeld aan de productiviteitsgroei. Wanneer
de bedrijven meer produceerden, en dus meer winst maakten, stegen de
lonen mee. Dat is al sinds de jaren zeventig niet meer het geval.
Terwijl de arbeidsproductiviteit steeg, is de koopkracht van het
minimumloon gedaald. Dat betekent concreet dat iemand met een
minimumloon nu 20\% minder uit kan geven dan in de jaren zeventig. Dat
geldt ook voor de uitkeringsgerechtigden. Dat werd zo problematisch, dat
er een stelsel van toeslagen is geïntroduceerd, zoals huurtoeslag,
zorgtoeslag en toeslag op de kinderopvang, die de allerergste armoede
moet compenseren. Het bedrijfsleven profiteert dus van de goedkope
arbeid, ten koste van de belastingbetaler. We moeten terug, zegt de FNV,
naar een situatie waarbij iedereen van een eerlijke baan rond kan komen,
zonder toeslagen nodig te hebben. Een situatie waarbij
uitkeringsgerechtigden niet terugvallen in armoede.

Er is een definitie van de absolute armoedegrens, waarbij niet in de
basisbehoeften voorzien kan worden. Maar de meest gehanteerde definitie
heeft het over een niet-veel-maar-toereikendgrens. Dat houdt niet over,
maar het gaat, als je tenminste een betaalbare huurwoning kunt vinden,
en geen pech hebt met onverwachte onkosten, want die grens voorziet niet
in een reserve. In dat budget wordt 17 euro per maand voor `uitgaan'
berekend, en 24 euro per maand om van op vakantie te gaan. In 2016 moest
een eenoudergezin met twee kinderen van 1560 euro rondkomen.

Er is bovendien nog een grote categorie mensen die nu niet onder of op
de armoedegrens leeft, maar wel reden heeft om te vrezen dat ze daar in
terecht komen. Veel zzp'ers hebben veel moeite voldoende te verdienen,
kunnen hun verzekeringen niet betalen, hebben geen reserve en komen ook
niet toe aan een oudedagsvoorziening. Flexwerkers staan ook elke keer
weer voor de vraag of er nog werk voor hen is. (Op dit moment heeft in
Nederland 36 procent van de werkenden geen vast contract.) Eens konden
studenten erop rekenen na het afstuderen werk te krijgen en een
hypotheek. Ook die zekerheid is verdwenen. Veel afgestudeerden beginnen
met een forse schuld, komen niet aan een woning, en lang niet altijd aan
werk.

Het zijn vooral oudere mensen die het grootste risico hebben op armoede
(vanwege de gestegen zorgkosten) en alleenstaande ouders. Het Nibud
heeft uitgerekend dat een stel met twee kinderen per maand gemiddeld 217
euro tekort komt.

Kabinet en gemeenten zeggen zich met name zorgen te maken over het
aantal kinderen dat in armoede leeft. Ze hebben slechtere kansen op
school en hebben vaker problemen. Het kabinet schuift miljoenen door
naar de gemeenten om wat aan de kinderarmoede te doen. Dat gebeurt
voornamelijk in de vorm van hulp in natura. Afhankelijk van de gemeente
die zelf een eigen armoedegrens in mag stellen, kunnen schoolgaande
kinderen hulp krijgen: een fiets, een laptop. Of een bijdrage voor
sport- of muziekclub. Een dagje uit naar de Efteling. Ook is er
inmiddels een keur aan organisaties die zich specialiseren in
hulppakketten met kleren en speelgoed, die steeds opnieuw eens per jaar
mogen worden aangevraagd. Of met een pakket waarmee een kind een
verjaardagsfeestje kan geven.

Dit is het punt: de maatregelen die de gemeenten inzetten tegen
kinderarmoede doen niets aan de armoede, en doen alleen aan
symptoombestrijding. Hoeveel arme kinderen kennen we met rijke ouders?
Geen. Net zo min als we rijke kinderen kennen met arme ouders. De nadruk
op kinderarmoede verhult dat het gaat om armoede van de ouders. Die
moeten van alles doen om nog wat spullen voor hun kinderen in de wacht
te slepen. Aan de armoede van de ouders zelf, die de huur en de
elektriciteit moeten betalen, wordt echter niets gedaan. Ook is het
vernederend dat de ouders zelf niets te zeggen hebben over wat hun
kinderen nodig hebben, de kinderen zelf trouwens ook niet.

Vooral langdurige armoede heeft niet alleen financiële problemen tot
gevolg, maar ook psychische en lichamelijke problemen. Het kan een
structurele vorm van permanente stress opleveren, angst voor de dag van
morgen, niet meer over een toekomst kunnen denken, depressies, isolement
en gevoelens van waardeloosheid. Arme mensen leven minder lang, en zijn
bovendien vaker ongezond. Niet alleen armoede heeft een vergaand
psychisch effect. Het gaat ook om ongelijkheid die als onrechtvaardig
wordt ervaren. In landen waar de ongelijkheid groot is, is de fysieke en
mentale gezondheid van een groot deel van de bevolking slechter. Ook is
er sprake van een bepaalde mate van `erfelijkheid': kinderen die in
armoede opgroeien hebben meer kans om als volwassenen ook in armoede
terecht te komen. We leven in toenemende mate in een harteloos
neoliberaal klimaat, waarin `je eigen broek ophouden' de toetssteen is
geworden om mensen op te waarderen, of af te wijzen.

Werkelijke armoedebestrijding vereist structurele herverdeling van de
welvaart. Daar is in Nederland geen sprake van. Wat we armoedebeleid
noemen komt meestal neer op het opvangen van de ergste scherpe kantjes,
door aan de ene kant een nieuwe vorm van liefdadigheid op te tuigen,
voedselbank, kledingpakketten, en aan de andere kant hulpverlening,
zoals cursussen budgetteren en schuldhulpverlening. We zeggen niet dat
hulpverlening slecht is, we zeggen wel dat die geen verandering brengt
in de productie van armoede.

Om werkelijk wat aan armoede te doen moeten we onder ogen zien dat de
ongelijke verdeling van welvaart inherent is aan het kapitalistische
systeem. Het zijn de rijken die kunnen beslissen waarin ze hun geld
investeren. Wordt dat niet geïnvesteerd in productie in het eigen land,
maar, wat in toenemende mate het geval is, in meer lucratieve financiële
transacties, of in `lage lonen landen', dan is er te weinig werk en
worden de lonen laag gehouden. Ook weten we inmiddels dat de vermogende
toplaag in verhouding heel weinig belasting betaalt. Deze feiten kunnen
we weten, maar worden verdoezeld door een neoliberale ideologie waarin
elk individu zelf verantwoordelijk is voor het bestaansniveau. En waarin
mensen in toenemende mate elkaars concurrenten worden, zoals alle
zzp'ers en de mensen met flexcontracten. We zien gebeuren dat er
tegenwoordig steeds meer `werkende armen' zijn, omdat veel banen niet
meer voldoende opleveren om met een gezin van te leven. Vrouwen met
kinderen werken noodgedwongen vaker parttime, en zo bestaat het dus nog
steeds dat ongeveer de helft van de vrouwen niet in hun eigen
levensonderhoud kan voorzien. Het hebben van werk is niet langer de
garantie voor een bestaansminimum.

Bij BIJ1 zien we maatregelen op korte en op lange termijn. Op korte
termijn moet ervoor gezorgd worden dat de kloof tussen arm en rijk niet
groter wordt, dat het minimumloon wordt opgetrokken, en dat de
uitkeringen meestijgen met de kosten van het bestaan. Er moet voor
gezorgd worden dat eenverdieners, met name ook alleenstaande ouders,
voldoende inkomen hebben, en liefdadigheid niet meer nodig is. Maar ook
als dat lukt is er nog geen einde gemaakt aan de wezenlijke
onrechtvaardigheid dat werkende mensen niet alleen werken om zelf in
leven te blijven, maar zonder daar zeggenschap over te hebben zien hoe
de opbrengst van hun werk verdwijnt in de zakken van de rijken, de
aandeelhouders en de multinationals. Wij horen zeggenschap te hebben
over waar het door ons verdiende geld heen gaat.

\begin{enumerate}
\def\labelenumi{\arabic{enumi}.}
\setcounter{enumi}{3}
\item
  \begin{quote}
  \textbf{ERVARINGSDESKUNDIGHEID IN DE ZORG}
  \end{quote}
\end{enumerate}

BIJ1 heeft als basis: niets over ons, zonder ons. Vanzelfsprekend
pleiten we dan ook voor de inzet van ervaringsdeskundigheid. Het is een
beweging voor visie, verbetering en verandering van hoe wij de zorg
organiseren. Vol bruggenbouwers, kwartiermakers, kritische adviseurs en
luizen in de pels.

Ervaringsdeskundigen bezitten een waardevolle vorm van kennis:
ervaringskennis. De kennis van het `aan den lijve hebben ervaren' van
een kwetsbaarheid of problematische situatie waarin je afhankelijkheid
of onmacht hebt ervaren. Ervaringskennis (van zowel ervaringswerkers,
hulpverleners als de cliënt) moet gelijkwaardig worden aan
wetenschappelijke en professionele praktijkkennis.

\textbf{Van plofparticipatie naar eigenaarschap}

We signaleren dat er nu vaak sprake is van plofparticipatie, voor de
schijn en voor de show, zonder eigenaarschap bij de mensen zelf. We
pleiten voor een beweging van ervaringsdeskundigheid, werkend vanuit
eigenaarschap, voor en door de mensen die zorg ontvangen. Zodat het haar
eigen vernieuwende kracht kan behouden en het politiek kritische aspect
met oog voor mensenrechten en (on)macht tot zijn recht komt.

\textbf{Vrije ruimte voor de leefwereld}

Ervaringsdeskundigheid heeft vrije ruimte nodig, die nogal eens wordt
bekneld vanwege de bestaande kaders. Hierdoor krijgt de vernieuwende
kracht te weinig ruimte. In de praktijk krijgt ervaringsdeskundigheid
daardoor onvoldoende de kans om structurele problemen in zorg en welzijn
op te lossen. Ervaringsdeskundigheid is namelijk een correctie vanuit de
basis op het huidige systeem. Onze oproep is dan ook: geef
ervaringsdeskundigen de ruimte om denkwijzen uit de leefwereld door te
laten werken in de systeemwereld.

\emph{We kunnen een probleem niet oplossen met de denkwijze die het
heeft veroorzaakt - Albert Einstein}

\textbf{Op alle plekken in verschillende vormen}

Ervaringsdeskundigheid moet op verschillende manieren benut worden.
Zowel in de vorm van cliëntenparticipatie als in de vorm van een
professionele bijdrage en vernieuwing in hoe we de zorg organiseren.
Momenteel lijkt de inzet van ervaringsdeskundigheid vooral gericht op
microniveau (individuele ondersteuning), enigszins op mesoniveau
(richting of in een organisatie), maar het kan juist ook waardevol zijn
op macroniveau (gericht op regionaal en landelijk strategisch beleid).

\textbf{Gelijkwaardige positionering}

Ervaringsdeskundigheid is een emancipatiebeweging van gelijkwaardigheid.
Dat betekent dat ervaringsdeskundigen ook gelijkwaardig behandeld en
gepositioneerd moeten worden. Bijvoorbeeld door ervaringswerkers in de
cao jeugdzorg een plek te geven. Door zorg te dragen voor een
fatsoenlijk betaalde, stevige positie. En door ervaringsdeskundigheid
een vast onderdeel te maken van het curriculum op opleidingen voor
hulpverleners. Professionalisering van ervaringsdeskundigheid is
wenselijk, op voorwaarde dat deze professionaliteit de eigen kenmerkende
kritische en vernieuwende kracht van ervaringskennis behoudt.

\begin{enumerate}
\def\labelenumi{\arabic{enumi}.}
\setcounter{enumi}{4}
\item
  \begin{quote}
  \textbf{ZORG VOOR TRANS PERSONEN}
  \end{quote}
\end{enumerate}

BIJ1 is tegen het fenomeen van `gatekeeping' van transgenderzorg. De weg
naar hormonen en operaties is momenteel lang en omslachtig. Artsen
bepalen welke operaties gedaan zullen worden, en hoe lichamen en
geslachtsdelen eruit moeten zien. BIJ1 gelooft in het recht op
zelfbeschikking van transgender personen. Zij hebben recht op heldere en
uitgebreide informatie van artsen, en moeten vervolgens het recht
krijgen om zelf keuzes te kunnen maken.\\
~\\
Ook moet de psychologische beoordeling van trans personen stoppen. Nu is
er een diagnose van `genderdysforie' nodig. Daar is BIJ1 tegen: trans
personen verdienen de vrijheid om zelf te bepalen of zij transgender
zijn. Als zij daar hulp bij nodig hebben, moet er uiteraard
psychologische ondersteuning beschikbaar zijn, maar dat zou meer een
informatieve dan een diagnostische functie moeten hebben. Als
transgender personen een hormoonbehandeling willen, moet het mogelijk
zijn deze te krijgen via de huisarts of (in ingewikkelde casussen) via
de endocrinoloog. Dit zou niet meer via een psycholoog geregeld moeten
worden. Ditzelfde geldt voor operaties: een chirurg zou (in samenspraak
met de huisarts) die beslissing moeten maken, zonder tussenkomst van een
psycholoog. Door middel van deze veranderingen zal de weg naar gepaste
zorg sneller en gemakkelijker zijn.

Transgenderzorg moet op deze manier net als andere vormen van zorg
worden gegeven: zonder tussenkomst van een psychologische beoordeling.
Ook moet de transgenderzorg worden uitgebreid zodat er een einde komt
aan de maanden- of soms zelfs jarenlange wachtlijsten.

\begin{enumerate}
\def\labelenumi{\arabic{enumi}.}
\setcounter{enumi}{5}
\item
  \begin{quote}
  \textbf{ONDERWIJSSEGREGATIE}
  \end{quote}
\end{enumerate}

De marktwerking die in de jaren '80 onder onderwijsminister Deetman (CHU
en CDA) zijn intrede deed, heeft schadelijke effecten. Tijdens het
beleid van minister Deetman werd de \emph{lumpsum} ingevoerd, moesten
scholen elkaar gaan beconcurreren en ging de overheid toetsen op de
resultaten. Zo kwam de verantwoordelijkheid voor de kwaliteit van het
onderwijs bij de schoolbesturen te liggen. Ook concludeert het Nationaal
Regieorgaan Onderwijsonderzoek dat er weinig empirisch bewijs is
gevonden voor het huidige beleid dat veronderstelt dat goed bestuur
leidt tot goed onderwijs.

De toenemende marktwerking in het onderwijs zorgt ervoor dat de
kansenongelijkheid tussen leerlingen en de werkdruk voor leraren telkens
toeneemt. Hoewel het onderwijs voor het grootste gedeelte niet
geprivatiseerd is in Nederland, krijgt inmiddels één op de drie
leerlingen bijles, huiswerkbegeleiding of examentraining via een
particulier instituut. Dit zorgt ervoor dat leerlingen, afhankelijk van
het geld dat beschikbaar is in het gezin, onderwijs van verschillende
kwaliteit krijgen. Extra ondersteuning hangt dus vooral af van hoeveel
geld je hiervoor beschikbaar hebt. In plaats daarvan zou voor iedereen
het onderwijs toereikend moeten zijn op het gebied van kwaliteit. Ook
leraren in het primair en voortgezet onderwijs geven steeds vaker les
als zzp'er of laten zich detacheren op een school. Dit zorgt ervoor dat
leraren tegen elkaar uit worden gespeeld. Deze ontwikkelingen zouden
onderwijssegregatie tegen moeten gaan, maar door de toenemende autonomie
van scholen en betrokken marktpartijen is segregatie juist toegenomen.

Toenemende marktwerking is slechts een van de vele oorzaken van
segregatie in het onderwijs. Ook onderadvisering speelt een grote rol.
Het schoolsysteem is nu zo vormgegeven dat het schooladvies aan het
einde van de basisschool enkel wordt heroverwogen bij een hoger
toetsadvies, gebaseerd op de basis van de uitslag van de eindtoets.
Daarnaast worden scholen niet verplicht om het schooladvies na deze
uitslag daadwerkelijk te verhogen. Dat betekent dat het advies voor de
leerling nu voornamelijk afhangt van het schooladvies, en daarnaast van
het advies van de docent. Hierdoor is er nog altijd sprake van
onderadvisering, een fenomeen dat niet iedereen gelijk treft. Leerlingen
met een niet-westerse migratieachtergrond krijgen vaker onterecht een te
laag schooladvies dan andere leerlingen. Meisjes krijgen ook nog altijd
vaker onterecht een lager schooladvies in vergelijking met jongens. De
verschillen tussen kinderen van hoog- en laagopgeleide ouders in
advisering groeien ook nog steeds. Structurele vormen van discriminatie
zoals klassisme, racisme en seksisme zorgen er dus voor dat bepaalde
bevolkingsgroepen vaker een te laag schooladvies krijgen dan andere
bevolkingsgroepen.

Daarnaast werkt vroege selectie sociale ongelijkheid in de hand. Het
moment van selectie in het Nederlandse onderwijsstelsel, op twaalfjarige
leeftijd, wordt steeds bepalender. De brugklassen versmallen, de
brugperiode wordt verkort, en er zijn steeds minder brede
scholengemeenschappen, vooral in de kleine steden. Segregatie tussen
scholen neemt toe, zo laat de Onderwijsinspectie zien. Het feit dat
leerlingen op zo'n vroege leeftijd worden gescheiden zorgt ervoor dat de
kansen van een leerling veel te vroeg worden bepaald. Eenmaal terecht
gekomen op een middelbare school, is opstromen naar een hoger niveau
vaak erg moeilijk.

Door de hiervoor genoemde marktwerking gaan scholen steeds meer als
marktpartijen concurreren voor de `beste' leerlingen, met grotere
ongelijkheden en veel meer segregatie als gevolg. Er moet dus een einde
komen aan ons economisch onderwijsmodel, en de overheid moet meer
manieren faciliteren om het selectiemoment te verlaten, bijvoorbeeld
door het stimuleren van brede brugklassen.

In het PO en VO is ook nog sprake van een andere vorm van segregatie:
die tussen het speciaal en het regulier onderwijs. Leerlingen met een
beperking komen vrijwel niet in aanraking met leerlingen zonder een
beperking, omdat zij al vanaf vroege leeftijd naar het speciaal
onderwijs gaan. BIJ1 gelooft dat de samenleving zo inclusief mogelijk
moet zijn. Dit houdt in dat de samenleving aangepast zou moeten worden
zodat mensen met een beperking volledig kunnen deelnemen.

Het belang van de school als ontmoetingsplaats voor leerlingen met
verschillende achtergronden en capaciteiten is onmiskenbaar en wordt in
het huidige systeem onvoldoende erkend. Uit onderzoekt blijkt dat het
goed is voor de ontwikkeling van álle kinderen, als kinderen met en
zonder beperking samen naar school gaan. Daarnaast is samen leren niet
alleen van elkaar leren, maar elkaar ook beter begrijpen, nu en in de
toekomst. Dit kan ook de discriminatie van mensen met een beperking op
latere leeftijd verminderen. Tegelijkertijd biedt een rijke, diverse
schoolomgeving kinderen niet alleen meer kansen, maar ook gelijke
kansen. Het huidige stelsel scheidt de leerlingen van jongs af aan en
houdt hierdoor segregatie in stand. Wil het onderwijs kunnen voldoen aan
zijn maatschappelijke opdracht van zowel inclusief als passend
onderwijs, dan vraagt dat juist om méér differentiatie binnen de school
en de klas.

Om onderwijssegregatie tegen te gaan, stellen wij dan ook de volgende
beleidsaanbevelingen voor: allereerst is een uitstel van het
selectiemoment bij de overgang van het PO naar het VO belangrijk.
Mogelijkheden hiertoe moeten onderzocht worden. Brede brugklassen moeten
door het hele land op scholengemeenschappen aanwezig zijn. Alle
schoolbesturen moeten zich in gaan zetten om onderadvisering tegen te
gaan. Samen naar School-klassen en andere vormen van inclusief onderwijs
voor kinderen met een beperking moeten een optie worden in het hele
land, zodat iedere leerling en diens ouders een keuze hebben over de
vorm van onderwijs die zij willen krijgen.

\begin{enumerate}
\def\labelenumi{\arabic{enumi}.}
\setcounter{enumi}{6}
\item
  \begin{quote}
  \textbf{DE ONDERWIJSVISIE VAN BIJ1}
  \end{quote}
\end{enumerate}

Hoofdwerk heeft meer waardering in onze maatschappij dan handwerk. Dit
wordt op allerlei manieren duidelijk in onderwijs en samenleving:
technische en uitvoerende beroepen worden minder gewaardeerd. Zowel in
het primair onderwijs als in het voortgezet onderwijs is er een enorme
focus op cognitieve ontwikkeling, met veel aandacht voor vakken zoals
taal, rekenen, begrijpend lezen en spelling. Er wordt weinig aandacht
besteed aan vaardigheden als creativiteit, organisatievermogen,
inlevingsvermogen, planningsvaardigheden, het nemen van initiatief,
samenwerkingsvermogen en het goed om kunnen gaan met teleurstellingen en
emoties. Als je kijkt naar wat kinderen nodig hebben op latere leeftijd,
is de tweede groep vaardigheden minstens zo belangrijk als de eerste.
Ons onderwijs moet kinderen meer bieden dan alleen ontwikkeling van
cognitieve vaardigheden. Scholen zouden ook de plek kunnen zijn waar
kinderen bijvoorbeeld een democratische opvoeding krijgen, goed leren
omgaan met elkaar, leren om met elkaar een gemeenschap te vormen en
leren om met verantwoordelijkheden om te gaan.

Op het primair onderwijs is het bijbrengen van kennis over taal,
rekenen, spelling en begrijpend lezen het doel van het werk geworden
voor de docenten. BIJ1 is van mening dat deze vakken een middel zouden
moeten zijn om andere, belangrijkere vaardigheden te leren. Zowel
leraren, die te kampen hebben met hoge leerdoelen, hoge werkdruk en
weinig voorbereidingstijd, als leerlingen, die snel verveeld raken en
creatief onvoldoende gestimuleerd worden, hebben veel last van het
huidige systeem.

Daarnaast laat het huidige onderwijssysteem te weinig ruimte voor
variatie in eigenschappen en vaardigheden van de leerlingen. Leerlingen
worden langs dezelfde meetlat gelegd. Als je op de middelbare school
slecht bent in talen, maar heel goed in wiskunde, kan het zomaar zijn
dat je niet op vwo-niveau eindexamen kan doen, omdat je door je cijfers
op taal al bent afgestroomd naar de havo. Dat betekent dat je dus ook
niet meer voldoende uitgedaagd wordt in wiskunde. Zo moeten kinderen dus
vooral werken aan het wegwerken van een achterstand van iets waar ze
niet goed in zijn, terwijl zij nauwelijks worden uitgedaagd in de vakken
waar ze heel goed in zijn. Daarbovenop komt nog de toetscultuur, die een
grote rol speelt in de ontwikkeling van een laag zelfbeeld voor
leerlingen die moeite hebben met de manier waarop en welke vaardigheden
worden getoetst.

Kortom: er zou meer focus moeten komen op het aanleren van vaardigheden
en executieve functies. Executieve functies horen bij het denkvermogen.
Het zijn denkprocessen die nodig zijn om activiteiten te plannen en aan
te sturen. Je kunt ze zien als een 'dirigent'. Ze helpen bij alle
soorten taken. Hier moeten meer leerlijnen voor komen. Er moet een einde
komen aan de toetscultuur. Het curriculum moet uitgebreid worden met
meer handvakken. Er moet onderzoek gedaan worden naar de mogelijkheden
voor een flexibeler curriculum, dat meer toegespitst is op de
individuele talenten van de leerlingen. Er moet meer focus komen op
sociale vaardigheden. Kinderen moeten meer kunnen bewegen tijdens
schooltijd en meer in contact kunnen komen met de natuur, in plaats van
altijd binnen te zitten. School zou een plek moeten zijn waar kinderen
zich meer, in plaats van minder gestimuleerd voelen.

Tot slot is het belangrijk dat het curriculum op het primair en
voortgezet onderwijs aandacht gaat besteden aan het koloniale en
slavernijverleden van Nederland. Dit is een belangrijk onderdeel van
onze geschiedenis dat vaak onderbelicht wordt, en vanaf jonge leeftijd
zou moeten worden meegegeven. De consequenties van ons verleden spelen
nog steeds door in onze huidige samenleving, waarin racisme een
structureel en geïnstitutionaliseerd probleem is. Het curriculum op
scholen zou meer ruimte moeten laten voor een diverser en inclusiever
perspectief op onze geschiedenis, ook om ons heden beter te kunnen
begrijpen.

\begin{enumerate}
\def\labelenumi{\arabic{enumi}.}
\setcounter{enumi}{7}
\item
  \begin{quote}
  \textbf{DIGITALISERING IN HET ONDERWIJS}
  \end{quote}
\end{enumerate}

Digitalisering zou een middel moeten zijn om kansengelijkheid en
onderwijskwaliteit te bevorderen. Dat is het nu vaak niet. De klassen
worden te groot waardoor leraren te weinig tijd en aandacht hebben om
leerlingen adequaat bij te kunnen staan. De leerling-volgsystemen zijn
een consequentie van dit tijd- en geldgebrek, maar menselijk contact
raakt hierdoor verloren. BIJ1 gelooft dat menselijk contact een
essentieel onderdeel van onderwijs is dat absoluut niet verloren mag
gaan.

Ook komt er steeds meer surveillance in het onderwijs door soortgelijke
systemen. BIJ1 gelooft dat het belangrijk is dat leerlingen al op jonge
leeftijd leren dat ze een fout mogen maken, en dat ze in alle vrijheid
en met privacy hun intellectuele ontwikkeling moeten kunnen doormaken.
Bepaalde leerling-volgsystemen permitteren ouders om de prestaties van
hun kind, soms al bij oefenopgaven, te vergelijken met andere kinderen
in de klas. Daarom is het belangrijk om kritisch te kijken naar de rol
van surveillance in de digitalisering van het onderwijs en de mogelijk
schadelijke effecten hiervan. Daarnaast zou de overheid scholen moeten
ontmoedigen gebruik te maken van platforms van bedrijven die
surveillance als bedrijfsmodel hebben.

Digitalisering speelt een grote rol in onze huidige samenleving. BIJ1
gelooft dat het niet alleen belangrijk is dat leerlingen digitale
vaardigheden worden bijgebracht, maar dat zij ook een kritische houding
leren aan te nemen ten opzichte van technologie. Dit zou bijvoorbeeld
kunnen door programmeerlessen aan te bieden op scholen, zodat leerlingen
beter leren begrijpen hoe computers en technologie in elkaar zitten.
Door meer te begrijpen over hoe computers, algoritmes en technologie in
elkaar zitten, is het ook makkelijker je tegen bepaalde risico's als
privacyschending te wapenen.

Digitale oplossingen kunnen ook laagdrempeligheid waarborgen en
toegankelijkheid bevorderen. Opleidingsinstituten moeten daaraan
meewerken, zodat het onderwijs toegankelijker wordt voor bijvoorbeeld
leerlingen en studenten met een beperking. Daarbij zouden scholen altijd
laptops en wifi-hotspots ter beschikking moeten stellen voor leerlingen
uit minima-gezinnen.

\begin{enumerate}
\def\labelenumi{\arabic{enumi}.}
\setcounter{enumi}{8}
\item
  \begin{quote}
  \textbf{HET MBO}
  \end{quote}
\end{enumerate}

BIJ1 staat voor inclusief en gelijkwaardig onderwijs. Op dit moment is
er in Nederland nog steeds sprake van een onderwaardering en
onderinvestering in het mbo-onderwijs. Daarom zouden wij de volgende
verbeterpunten willen aandragen: de doorstroom van het mbo naar het hbo
moet verbeterd worden. Dit zou bijvoorbeeld beter verlopen als er een
contactpersoon wordt ingesteld op elke hbo-instelling die zich
bezighoudt met de voormalige mbo-studenten. Ook moeten er betere
schakelprogramma's komen, zodat de drempel tot doorstroom zo laag
mogelijk wordt.

Ook is het nu zo dat het mbo verplicht burgerschapsonderwijs als vak
moet geven. Voor andere onderwijsinstellingen geldt dit niet. Deze
uitzonderingspositie is volgens BIJ1 nergens voor nodig.
Burgerschapsonderwijs zou op elke onderwijsinstelling als vak gegeven
moeten worden. Ook zou de werkdruk op het mbo verlaagd moeten worden
voor docenten. Nu is het vaak zo dat zij te weinig tijd hebben om hun
lessen goed en adequaat voor te bereiden. Daarnaast hebben zij erg
weinig doorgroeimogelijkheden in hun beroep.

Kortom: het beroepsonderwijs zou versterkt moeten worden. Vmbo- en
mbo-instellingen zouden beter moeten samenwerken voor een soepele
doorstroom. Mbo-docenten moeten meer ruimte krijgen om door te groeien.
En het allerbelangrijkste: de waardering voor het beroepsonderwijs moet
versterkt worden, niet alleen op ideologisch niveau, maar ook op
praktisch niveau. Dit betekent dat er meer financiële middelen moeten
komen voor de docenten en de leerlingen; zodat elke leerling de
benodigde aandacht krijgt.

\begin{enumerate}
\def\labelenumi{\arabic{enumi}.}
\setcounter{enumi}{9}
\item
  \begin{quote}
  \textbf{HET WO}
  \end{quote}
\end{enumerate}

In het wetenschappelijk onderwijs (wo) is het niet veel beter dan in
andere onderwijssectoren. Sinds 2000 zijn er 68\% meer studenten in het
wo, en is de rijksbijdrage per student met maar liefst 25\% geslonken.
De Nederlandse uitgaven aan het wo blijven ver achter in vergelijking
met andere landen.

Universitair personeel werkt structureel en massaal over. De werkdruk is
enorm. Docenten worden gedwongen hun onderzoekstijd te gebruiken voor
onderwijs, omdat daar veel te veel werk is. Daardoor moeten zij hun
onderzoek vaak in hun vrije tijd doen, wat zorgt voor een nog hogere
ervaring van werkdruk en massale burn-outs. Uit onderzoeken van FNV en
VAWO blijkt dat 78\% van het personeel aan universiteiten in de
weekenden en avonden moet werken om hun werk bij te te houden, maar dat
73\% zelfs dan hun werk nog niet af krijgt. Bijna de helft van de
docenten werkt gewoon door als ze ziek zijn. Ook krijgt ruim driekwart
het overwerk niet gecompenseerd. Meer dan de helft geeft aan
lichamelijke of psychische klachten te hebben als gevolg van de hoge
werkbelasting. De toenemende studentenaantallen, een ontoereikende
urenvergoeding en het personeelstekort zorgt ervoor dat de werkdruk
alleen maar toeneemt. Hierdoor ontstaat er een lagere betrokkenheid,
minder bevlogenheid en een lage tevredenheid onder docenten.

Als gevolg van de structurele afname in financiering en de toenemende
onzekerheid in financiering van afgestudeerde PhD-studenten is het
aantal flexibele contracten voor gepromoveerden aan de universiteit
ernstig toegenomen. Deze toenemende flexibilisering van de arbeidsmarkt
zorgt voor een grote prestatiedruk, veel onzekerheid en werkstress. Het
is belangrijk dat promovendi een vast contract krijgen in plaats van dat
ze met beurzen worden beloond, om hun rechten en financiering te
waarborgen. Daarnaast moet de NWO garanderen dat jonge onderzoekers
genoeg kansen hebben om gesubsidieerd te worden.

Deze verarming van het wetenschappelijk onderwijs zorgt ervoor dat de
onderwijskwaliteit achterblijft. De voorbereidingstijd voor colleges
neemt af, het aantal studenten per werkgroep neemt toe. Het aantal
scripties dat docenten moeten begeleiden is zo groot dat de kwaliteit
van de begeleiding hieronder lijdt. Om al deze schadelijke
ontwikkelingen tegen te gaan, stellen wij het volgende voor: de
doelmatigheidskorting wordt afgeschaft, om verdere bezuinigingen in het
wo te voorkomen. Daarnaast moet de overheidsbijdrage aan het wo hersteld
worden naar het niveau van 2000. Dit komt neer op een extra investering
van 1,15 miljard euro per jaar. Zo keert de rijksbijdrage per student
terug naar een niveau waarop het onderwijs een betere kwaliteit zal
hebben op alle vlakken.

Ook moet er meer democratisering en decentralisatie komen op het hoger
onderwijs. Dit zou bijvoorbeeld kunnen door het herintroduceren van de
Universiteitsraad zoals die tot halverwege de jaren `90 bestond, waarin
studenten en medewerkers gezamenlijk fungeren als een soort parlement,
en instemmingsrecht, budgetrecht en initiatiefrecht hebben. Colleges van
Bestuur en Raden van Toezicht worden afgeschaft. Decanen moeten gekozen
kunnen worden. Er moet kritisch gekeken worden naar de
bachelor-masterstructuur (en het promotietraject) van het
Bologna-akkoord, waarmee een hele reeks aan centralisering,
uniformisering en marktwerking in gang is gezet. De accreditatie van
opleidingen zou minder op eindtermen en leerdoelen gebaseerd moeten
zijn. Onderwijsinstellingen en vooral democratische en decentrale
organen in die instellingen zouden hier een sterke stem in moeten
hebben.

Marktwerking in het hoger onderwijs moet tegengegaan worden. De
arbeidsmarkt dient geweerd te worden van (semi-)openbare instellingen in
zoverre zij onderwijs en onderzoek op de universiteit en hogeschool
ondermijnt. Leren draait niet om duurzame inzetbaarheid voor de
arbeidsmarkt of om het hiertoe verkrijgen van 'competences', maar om
nieuwsgierig verkennen en bestuderen, om falen, om het cultiveren van
een houding tegenover de wereld. Universiteiten moeten hun geld verdelen
onder faculteiten en opleidingen op basis van input, niet output. Nu is
het zo dat faculteiten en opleidingen budget krijgen op basis van de
hoeveelheid studenten, behaalde studiepunten en behaalde diploma's. In
plaats daarvan zou er gekeken moeten worden naar wat een vak en
opleiding daadwerkelijk aan budget nodig heeft om goed onderwijs te
kunnen bieden.

Tot slot moeten medewerkers een beroep kunnen doen op onafhankelijke
ombudsbureaus, die op hun beurt de macht en middelen hebben om een
onderzoek te beginnen naar misstanden.

\begin{enumerate}
\def\labelenumi{\arabic{enumi}.}
\setcounter{enumi}{10}
\item
  \begin{quote}
  \textbf{HET LERARENTEKORT}
  \end{quote}
\end{enumerate}

Om het huidige lerarentekort tegen te gaan en ervoor te zorgen dat deze
niet meer terugkomt, is het nodig om aan systeemverandering in plaats
van symptoombestrijding te doen. Dit komt neer op twee zaken: het
verlagen van de werkdruk en het verhogen van het salaris.

Er moet flink geïnvesteerd worden in het primair onderwijs. De werkdruk
is daar op het moment veel te hoog en het salaris veel te laag. Dat
zorgt ervoor dat er veel uitval is van leerkrachten, er weinig instroom
bij de lerarenopleidingen is, en een hoog ziekteverzuim. Deze
ontwikkelingen werken het lerarentekort in de hand, en zorgen ervoor dat
de onderwijskwaliteit op het PO in Nederland alsmaar daalt. Het aantal
burn-outs is nergens zo hoog als in het PO. Om deze werkdruk te
verminderen is een aantal zaken nodig, zoals het faciliteren van
onderwijsondersteunend personeel, zodat er meer hulp is voor de
leerkracht. Daarnaast moet er meer tijd komen om de lessen voor te
bereiden, bijvoorbeeld door het verminderen van de administratieve
lasten. Tot slot is het nodig dat de klassen kleiner worden. Dit
vermindert niet alleen de werkdruk voor de leraar, maar zorgt ook voor
een betere onderwijskwaliteit voor de leerling. Leraren moeten hun
creativiteit in het vak weer kwijt kunnen.

Ook in het voortgezet onderwijs loopt de werkdruk op. Leraren worden
weinig betrokken bij de besluitvorming, terwijl de prestatiedruk steeds
verder groeit. Onderwijsvernieuwingen mislukken hierdoor vaak. Bovendien
komen investeringen in het onderwijs zelden terecht bij de leraren en de
klassen zelf. Eén op de vier docenten in het VO kampt inmiddels met
burn-outklachten. Het lerarentekort neemt steeds verder toe, onder
andere door de verminderde toestroom, die deels weer te maken heeft met
de toenemende werkdruk en het lage salaris.

Om kwalitatief goed onderwijs te verzorgen is het van belang dat leraren
op het VO minder lessen per week gaan geven. In Europa geven leraren
gemiddeld 20 lessen per persoon per week. In de best presterende landen
ligt dat aantal lager. Maar in Nederland geven docenten gemiddeld 25
lessen per week. Hierdoor is er minder tijd voor kwaliteit. Het is hoog
tijd om te zorgen dat het aantal lessen per week voor leraren in
Nederland omlaag gaat, zodat deze meer tijd kunnen besteden aan de
kwaliteit van hun onderwijs.

Het lerarentekort in het VO zal drastisch blijven toelopen als er niet
nu ingegrepen wordt. Als het zo doorgaat, zullen bepaalde vakken
helemaal niet meer onderwezen worden over een aantal jaar. Om dit te
voorkomen is het belangrijk dat leraren weer meer zeggenschap krijgen
over de inrichting van hun onderwijs en dat zij betrokken worden bij
besluitvorming daarover op alle niveaus. Ook is structurele toename van
de financiering in het VO hard nodig.

Tot slot is het belangrijk dat de overheid manieren gaat onderzoeken om
het uitzenden van leerkrachten via uitzendbureaus tegen te gaan. Zo
worden leraren namelijk tegen elkaar uitgespeeld. Zo worden leraren
namelijk tegen elkaar uitgespeeld. Pabo's zouden ook meer aandacht
moeten besteden aan hoe om te gaan met de werkdruk in het onderwijsveld,
die vanuit zowel de samenleving, scholen als de ouders komt.

\begin{enumerate}
\def\labelenumi{\arabic{enumi}.}
\setcounter{enumi}{11}
\item
  \begin{quote}
  \textbf{TOEGANKELIJKHEID}
  \end{quote}
\end{enumerate}

Om de samenleving voor iedereen mensen toegankelijk en inclusief te
maken, is het belangrijk om een onderscheid te maken tussen het medische
model van beperking en het sociale model van beperking. Deze termen
maken onderscheid tussen wat daadwerkelijk beperkend is: je medische
conditie of de maatschappij die daar geen rekening mee houdt? Het
medische model stelt dat mensen met een beperking beperkt worden door
hun beperking. Deze beperking wordt gezien als een afwijking en middels
hulpmiddelen en individuele aanpassingen wordt geprobeerd deze afwijking
van de norm zo klein mogelijk te maken. Het sociale model stelt
daarentegen dat het de samenleving is die de beperking van mensen
daadwerkelijk beperkend maakt. Een concreet voorbeeld hiervan is dat het
niet de rolstoel (en dus de beperking) is die ervoor zorgt dat iemand
niet een bibliotheek in kan, maar de drempel bij de ingang, dus de
samenleving die niet voldoende is aangepast op de beperking.

Voor een duidelijke uitleg over het sociale model, zie het volgende
filmpje: https://nietsoveronszonderons.nl/sociaalmodel/

Het huidige systeem volgt het medische model. Hierdoor wordt er
voornamelijk vanuit liefdadigheid gehandeld naar mensen met een
beperking: mensen met een beperking worden vaak zielig gevonden en
gezien als hulpbehoevend. Tegelijkertijd zijn er vanuit de overheid
vrijwel geen sancties of verplichtingen om de samenleving daadwerkelijk
toegankelijk te maken. We gebruiken nu liefdadigheid om mensen met een
beperking zo dicht mogelijk naar de norm of het gemiddelde te brengen,
in plaats van structurele aanpassingen te doen die onze maatschappij
voor iedereen toegankelijk maken. Er wordt te veel vanuit liefdadigheid
gedacht en gehandeld, en niet vanuit mensenrechten.

In 2016 is in Nederland het VN-verdrag inzake rechten van personen met
een handicap\footnote{\href{https://iederin.nl/wp-content/uploads/2019/12/Schaduwrapport-VN-verdrag-Handicap.pdf}{\underline{https://iederin.nl/wp-content/uploads/2019/12/Schaduwrapport-VN-verdrag-Handicap.pdf}}
  Een toegankelijke en makkelijk begrijpelijke versie hiervan is hier te
  vinden:
  \href{https://iederin.nl/wp-content/uploads/2019/12/Schaduwrapport-eenvoudige-versie.pdf}{\underline{https://iederin.nl/wp-content/uploads/2019/12/Schaduwrapport-eenvoudige-versie.pdf}}}
geratificeerd. Dit verdrag is al in 2006 opgesteld en geschreven vanuit
het sociaal model:

\begin{quote}
``Erkennend dat het begrip handicap aan verandering onderhevig is en
voortvloeit uit de wisselwerking tussen personen met functiebeperkingen
en sociale en fysieke drempels die hen belet ten volle, effectief en op
voet van gelijkheid met anderen te participeren in de samenleving''.
\end{quote}

Het verdrag spreekt over handicaps volgens het sociaal model en benoemt
expliciet dat handicaps kunnen veranderen. Dat wil zeggen dat wie
vandaag gehandicapt is, dat morgen wellicht niet meer is. Niet omdat hun
beperking plotseling verdwijnt, maar omdat de maatschappij
toegankelijker is en hen niet langer gehandicapt maakt. BIJ1 is van
mening dat het de taak van de overheid is om deze maatschappij zo
toegankelijk mogelijk te maken. Wij vinden dat het VN-verdrag inzake de
rechten van personen met een handicap moet worden nageleefd, inclusief
sancties vanuit de overheid om ervoor te zorgen dat dit ook
daadwerkelijk gebeurt.

\begin{enumerate}
\def\labelenumi{\arabic{enumi}.}
\setcounter{enumi}{12}
\item
  \begin{quote}
  \textbf{HET RECHT OP ZELFBESCHIKKING VAN MOSLIMS}
  \end{quote}
\end{enumerate}

Er heerst in Nederland, net als in veel andere delen van de wereld, een
klimaat van islamofobie. Een jarenlange hetze tegen moslims door
politiek (extreem-)rechts heeft moslimhaat inmiddels genormaliseerd. De
Nederlandse overheid heeft de plicht om alle Nederlanders te beschermen
en gelijk te behandelen. Op dit moment is dat niet het geval.

Ook juridisch zijn er stappen te zetten in het tegengaan van moslimhaat
en het beter beschermen van het recht op zelfbeschikking voor moslims,
en in het bijzonder voor moslimvrouwen. BIJ1 wil dat het gedeeltelijk
verbod gezichtsbedekkende kleding (ook wel het niqab-verbod) per direct
wordt opgeheven. Ook moet het discriminatoire hoofddoekverbod voor
rechters en griffiers worden afgeschaft. Hier spreekt ook het College
voor de Rechten van de Mens zich hard voor uit.

Tot slot moet er hard voor gestreden worden dat werkgevers niet meer
zelf mogen bepalen of ze wel of niet mensen met een hoofddoek, keppeltje
of andere religieuze attributen in dienst nemen. Dit is discriminatie,
en daar moet de overheid zich hard tegen uitspreken.

\begin{enumerate}
\def\labelenumi{\arabic{enumi}.}
\setcounter{enumi}{13}
\item
  \begin{quote}
  \textbf{ABORTUS}
  \end{quote}
\end{enumerate}

Wereldwijd staan vrijheden en rechten zoals abortus onder druk. Ook in
Nederland zijn er signalen dat er nog altijd huisartsen zijn die
moraliserend omgaan met abortus, door hun morele oordeel mee te nemen in
het wel of niet doorverwijzen van een ongewenst zwanger persoon naar een
abortuskliniek. Er is geen transparantie in de Wet Afbreking
Zwangerschap, wat betekent dat er geen zicht is op hoe huisartsen de
beslissing (zouden moeten) maken over het wel of niet doorverwijzen naar
een abortuskliniek.

BIJ1 spreekt zich hier hard tegen uit, en vindt dat de overheid
hetzelfde moet doen. Er is nog altijd een taboe rondom abortus. Dit valt
niet alleen te zien aan de groeiende protesten rondom abortusklinieken,
maar ook aan de uitzonderingspositie die aan abortuszorg is toegekend
ten opzichte van reguliere zorg.

Abortus staat nog altijd in het Wetboek van Strafrecht. Abortus wordt
nog altijd behandeld als een criminele kwestie in plaats van een
medische zaak -- onterecht, wat ons betreft. Abortus moet dus uit het
Wetboek van Strafrecht. Om het taboe rondom abortus tegen te gaan zou
het ministerie van VWS een campagne moeten starten om abortus
bespreekbaar te maken. Het is belangrijk dat ongewenst zwangere personen
het gevoel hebben dat ze hierover mogen en kunnen praten in alle
vrijheid.

Abortusklinieken zijn in Nederland van hun financiering afhankelijk van
subsidies die zij jaarlijks moeten aanvragen. Dit geeft abortuszorg een
uitzonderingspositie. Ook is er maar een beperkt aantal abortusklinieken
in Nederland, waardoor er mensen zijn die soms ver moeten reizen
(bijvoorbeeld van Harlingen naar Groningen). Ook wordt niet voor alle
ongewenst zwangere personen abortus nu vergoed, en klinieken mogen zelf
bepalen of en hoeveel korting zij geven aan mensen die niet verzekerd
zijn onder de WLZ (Wet Langdurige Zorg), zoals ongedocumenteerde mensen.

Tegen al deze beperkende factoren en regelgeving spreekt BIJ1 zich uit.
Abortuszorg is een urgente zorg die voor iedereen volledig toegankelijk
moet zijn. BIJ1 is ook van mening dat abortuszorg onder de reguliere
zorg zou moeten vallen. Dit betekent ook dat er geen aparte
abortusklinieken meer zijn, maar abortusafdelingen in alle
verloskundigenpraktijken en andere medische centra. Dit bevordert niet
alleen de bereikbaarheid voor de cliënt, maar ook de bereikbaarheid voor
abortusartsen zelf.~

~

BIJ1 is van mening dat iemand die ongewenstzwanger is altijd het recht
heeft om in alle vrijheid een keuze te kunnen maken. De overheid heeft
de plicht om deze vrijheid te waarborgen. Er moet dus een einde komen
aan de verplichte 5 dagen bedenktijd, die er soms voor zorgt dat
ongewenst zwangere personen langer moeten wachten dan zij willen. Dit
kan leiden tot gevoelens van machteloosheid en stress. Ook moeten mensen
die hun zwangerschap willen beëindigen kunnen rekenen op onafhankelijke
en onpartijdige hulp bij deze moeilijke beslissing. Deze zorgverlener
zou, net als bij andere zorg, keuzehulpgesprekken aan de ongewenst
zwangere persoon moeten aanbieden, zonder een moreel oordeel aan deze
gesprekken te verbinden. Het is belangrijk dat de overheid geen
subsidies meer verstrekt aan organisaties die deze onafhankelijke hulp
niet bieden.

~

Ook moet de abortuspil toegankelijker zijn. Niet alle huisartsen
schrijven deze voor. Dit zou wel genormaliseerd moeten worden, en het is
belangrijk dat huisartsen hierin gaan samenwerken met abortusklinieken
en verloskundigenpraktijken.

Tot slot is het belangrijk dat er meer ervaringsdeskundigheid komt in de
abortuszorg. Het stigma rondom abortus is op het moment zo groot, dat
maar weinig mensen hun zorgervaring delen. Om de abortuszorg te kunnen
verbeteren, is het belangrijk dat mensen die een abortus hebben gehad
hierover mee kunnen praten.

\begin{enumerate}
\def\labelenumi{\arabic{enumi}.}
\setcounter{enumi}{14}
\item
  \begin{quote}
  \textbf{KLIMAATRACISME}
  \end{quote}
\end{enumerate}

De klimaatcrisis heeft op verschillende plekken wereldwijd al enkele
jaren schadelijke effecten op lokale ecosystemen, de beschikbaarheid van
water en vruchtbare grond, en daarmee het welzijn van de bevolking. In
Nederland en de rest van wat onder `het westen' wordt geschaard, zitten
we in een positie van privilege, want deze crisis zal op de korte
termijn onze levens nog niet op hetzelfde niveau bedreigen. In het kort:
de klimaatcrisis is veroorzaakt door westerse landen en multinationals,
maar treft vooral landen in Azië, Afrika en Zuid-Amerika.

Deze scheve verhoudingen gelden als pijnlijke illustratie van hoe
kolonialisme en imperialisme de huidige ongelijkheid zo ver uiteen heeft
gezet. Het zijn vooral landen die zelf een veel kleinere bijdrage
leveren aan de wereldwijde uitstoot die het meeste gevaar lopen. Denk
hierbij aan overstromingen of juist extreme droogte, die als gevolg
hebben dat ecosystemen en voedselvoorziening verdwijnen.

Deze gecreëerde ongelijkheid noemen we `klimaatracisme': witte
instituties hebben jaren geprofiteerd van bijvoorbeeld de verschillende
grondstoffen uit deze regio's en hebben daarbij gezorgd voor het
ontstaan van de klimaatcrisis, allemaal terwijl ze daarmee vooral de
lokale inwoners willens en weten blootstellen aan deze gevaren en hen
daarin aan hun lot overlaten.

Ook economische uitbuiting speelt hierin een grote rol: naast het feit
dat deze landen veel harder getroffen worden, hebben ze door eeuwen aan
westerse uitbuiting vaak ook niet de middelen om zowel preventief als
reactief om te gaan met de situaties die ontstaan door
klimaatverandering. Dit omdat de huidige verhoudingen veelal voor
armoede zorgen, wat de kwetsbaarheid van een land alleen maar verder
vergroot.

De oorzaak van de klimaatcrisis, namelijk de dorst naar constante
economische groei en de schadelijke maatregelen die daarvoor nodig zijn,
is niet alleen indirect verantwoordelijk voor leed: oorspronkelijke
bewoners wereldwijd krijgen te maken met directe aanvallen met als doel
hen te verjagen van hun grondgebied. Bijvoorbeeld om weilanden aan te
leggen of een mijn te bouwen voor aanwezige grondstoffen.

Ondanks dit alles zien we dat ook de overheersende stemmen in de
klimaatbeweging wit zijn. Ook hier vallen gemarginaliseerde groepen
veelal buiten de boot en krijgen hun stemmen nog te vaak niet de ruimte
die zij binnen deze beweging zouden moeten hebben. Dit terwijl
indigenous* groepen wereldwijd verreweg het grootste aandeel leveren als
het aankomt op de bescherming van natuurgebieden.

Daarom moet een rechtvaardige aanpak van de klimaatcrisis altijd
anti-kolonialistisch zijn en niet alleen opereren vanuit de
westerse/witte normen.

\emph{*Oorspronkelijke bewoners van een land en/of gebied}

\begin{enumerate}
\def\labelenumi{\arabic{enumi}.}
\setcounter{enumi}{15}
\item
  \begin{quote}
  \textbf{PALESTINA}
  \end{quote}
\end{enumerate}

Vanuit de twee grondwaarden die we bij BIJ1 hanteren, radicale
gelijkwaardigheid en economische rechtvaardigheid, is het voor ons
vanzelfsprekend dat we ons ook inzetten voor de rechten van Palestijnen.
Daarmee hoort het ook tot onze taak om duidelijk stelling te nemen tegen
de politiek van de staat Israël die deze rechten met voeten treedt.

Ons uitgangspunt daarbij is dat protest tegen de politiek van de staat
Israël meer dan gerechtvaardigd is. Onze stellingname daarbij blijft
geheel binnen de grenzen van de wet, binnen de normen van wat we onder
democratie verstaan - zie artikel 1 van onze Grondwet - en binnen de
internationaal afgesproken verdragen.

\textbf{Waarom het zo moeilijk blijkt om over de kwestie
Palestina/Israël te spreken}

Het is niet eenvoudig om een zinnige discussie te voeren over de kwestie
Palestina/Israël zonder dat de emoties hoog oplopen. De reden daarvoor
is dat er onderhuids paradigma's meespelen, waardoor we langs elkaar
heen spreken. In grote lijnen zijn er drie paradigma's te onderscheiden:

\begin{itemize}
\item
  \begin{quote}
  Paradigma 1: Israël is een klein land, een toevluchtsoord voor de
  joden, de overlevenden van de Shoah, een land dat zich staande moet
  houden in een vijandige, Arabische omgeving.
  \end{quote}
\item
  \begin{quote}
  Paradigma 2: Joodse Israëli's en Palestijnen vechten om hetzelfde
  stukje land, en zijn beiden niet bereid tot compromissen. Met andere
  woorden: waar er twee vechten, hebben er twee schuld.
  \end{quote}
\item
  \begin{quote}
  Paradigma 3: Israël is een settler-koloniale en bezettende mogendheid,
  die de inheemse Palestijnse bewoners van het land bij de stichting van
  de staat heeft onteigend en verdreven, en nog steeds doorgaat met die
  etnische zuivering.
  \end{quote}
\end{itemize}

Deze tegenover elkaar staande paradigma's gaan dus over de vraag wie de
hoofdverantwoordelijken zijn voor de ontstane situatie in het land dat
vroeger Palestina heette en nu Israël wordt genoemd.

Het is duidelijk dat wij ons terugvinden in paradigma 3. Dat is geen
ontkenning van de grote rol die jodenvervolging heeft gespeeld in het
ontstaan van de staat Israël, en ook geen ontkenning van het
uitgangspunt dat joden recht hebben op bescherming en veiligheid. Maar
we kunnen ook niet ontkennen dat de stichting van de staat Israël
rampzalige gevolgen heeft gehad voor een ander volk dat geen enkele
schuld treft voor het leed dat de joden in Europa is aangedaan. We
geloven daarom ook niet dat Israël op deze wijze door kan gaan, en we
zijn er ook van overtuigd dat er betere methoden zijn om antisemitisme
tegen te gaan. We sluiten ons aan bij de hoogleraar in Joodse studies
Steven Beller, die stelt dat het een betere strategie is om mee te
werken aan een maatschappij waarin verschil mag bestaan, en kleine
minderheden zoals de joden beschermd worden door een consensus waarbij
de meerderheid er ook voor zorgt dat de rechten en belangen van
minderheden worden gerespecteerd. Die opvatting delen we ook met de
joodse organisaties zoals in de VS the Jewish Voice for Peace, die vindt
dat Israël op de verkeerde weg is en niet namens hen spreekt. En met
soortgelijke groepen in alle westerse landen, zoals in Nederland Een
Ander Joods Geluid.

Het zionisme was eens onder joden in Europa een niet erg populaire
politieke stroming. Dat het uiteindelijk een enorme overwinning heeft
behaald en met behulp van de westerse mogendheden een staat kon vestigen
in het land dat Palestina heette, is zonder twijfel het gevolg van de
enorme misdaad die de joden (en andere groepen als Roma en Sinti) onder
het Duitse nazisme werd aangedaan. Het is begrijpelijk dat zionisten
zochten naar een oplossing die voor eens en altijd een einde zou maken
aan het antisemitisme. Maar wat in paradigma 1 wordt weggelaten, is dat
de `oplossing' ten koste is gegaan, en nog gaat, van een ander volk.
Israëlisch historicus Ilan Pappé beschrijft het zo: zionisme ontstond
als een streven naar een veilige vluchthaven voor de joden die het
slachtoffer werden van het Europese antisemitisme. Maar aangezien de
plek die daartoe werd gekozen al bewoond werd, werd het zionistische
streven een project van `settler colonialism'. In vroegere dergelijke
projecten zoals in Amerika en Australië leidde het settler colonialism
tot genocide op de inheemse bevolking. In Palestina leidde het tot een
eindeloos proces van etnische zuivering, dat tot op de dag van vandaag
doorgaat. Dat wil zeggen dat de Palestijnen die tussen 1947 en 1949 en
in 1967 zijn gevlucht niet meer terug mochten komen, en dat Palestijnen
op een steeds kleiner terrein worden teruggedrongen, hun grond en hun
huizen hun worden afgenomen, zoals op de Westoever, of dat ze zijn
opgesloten in de grootste openluchtgevangenis ter wereld: Gaza, waar
twee miljoen Palestijnen moeten leven op een stukje land ter grootte van
twee keer Texel.

Het is duidelijk waar wij als BIJ1 staan. We gaan ervan uit dat de
kwestie Palestina/Israël het best te framen is als een koloniaal
project, met desastreuze gevolgen, omdat dit project nog steeds gedoogd
en ondersteund wordt door de grootmachten van de westerse wereld. We
citeren The Rights Forum:

``De Palestijnen zijn niet bij machte hun rechten af te dwingen.
Daarvoor zijn zij aangewezen op de steun van de derde belangrijke partij
in de kwestie: de internationale gemeenschap. Die beperkt zich echter
tot het op papier eindeloos herbevestigen van de Palestijnse rechten.
Voor het daadwerkelijk beschermen daarvan ontbreekt, ook bij Nederland,
tot op de dag van vandaag de politieke wil. Daarmee is de internationale
gemeenschap medeverantwoordelijk voor het voortbestaan van de kwestie.
Anders dan een wijdverbreide voorstelling van zaken wil, is er in de
kwestie-Palestina/Israël geen sprake van twee partijen die in
vergelijkbare mate inbreuk maken op elkaars rechten. De kern van de
kwestie is de Israëlische bezetting en kolonisering van Palestijns
gebied, die gepaard gaat met schendingen van een scala aan rechten van
de overheerste Palestijnse bevolking. Dat ook Palestijnen zich schuldig
maken aan schendingen van het recht is evident, maar staat in geen
verhouding tot de Israëlische onderdrukking van miljoenen Palestijnen op
de Westelijke Jordaanoever, in Oost-Jeruzalem en in Gaza.''

\textbf{De politiek van de staat Israël}

Israël heeft geen grondwet die de gelijkheid voor de wet van alle
inwoners, ongeacht etniciteit of religie garandeert. Integendeel. Israël
is gedefinieerd als een joodse staat, waarbij joden nadrukkelijk meer
burgerrechten hebben dan niet-joden. Zo kan elke jood waar ook ter
wereld, ook die nog nooit in Israël is geweest, eenvoudig
staatsburgerschap verwerven in het kader van `het recht op terugkeer'.
Terwijl diezelfde wetten ervoor zorgen dat Palestijnen, ook zij die er
generaties hebben gewoond, niet terug mogen keren als ze eens gevlucht
zijn. Ook zijn er wetten die er specifiek op gericht zijn om zoveel
mogelijk land te onttrekken aan de Palestijnse eigenaren en tot
staatsgrond te verklaren die alleen door joden gebruikt mag worden.
Palestijnen met Israëlisch staatsburgerschap krijgen vaak geen
vergunning om een huis te bouwen op de grond die officieel nog van hen
is. Er zijn wetten die maken dat Palestijnse staatsburgers niet kunnen
wonen in alleen voor joden bestemde delen van oude en nieuwe steden.

Palestijnen zijn in Israël zelf om `veiligheidsredenen' uitgesloten van
een reeks van beroepen. De werkloosheid onder Palestijnse burgers is dan
ook veel hoger dan onder joden. Joodse steden hebben aanzienlijk meer
gemeentepersoneel. Palestijnen krijgen een veel kleiner deel van het
gezondheidsbudget. De zuigelingensterfte is onder Palestijnen vier keer
zo hoog als onder joden. Palestijnse dorpen en steden hebben door de
onteigening van grondgebied veel moeite om uit te kunnen breiden.
Daarentegen worden er nog steeds nieuwe joodse gemeenschappen
bijgebouwd, onder andere als nederzettingen op de Westoever. Experts op
het gebied van internationaal recht én VN-functionarissen noemen het
beleid van Israël apartheid --- volgens internationaal recht een misdaad
tegen de mensheid.

Dit gaat alleen over de miljoen Palestijnen die in Israël zelf wonen. Op
de Westoever worden de nederzettingen steeds verder uitgebreid, en de
Palestijnse bevolking op steeds kleinere enclaves bijeengedreven. Het is
geen geheim meer dat het de bedoeling is om de nederzettingen te
annexeren, wat feitelijk in de praktijk al is gebeurd. Jeruzalem wordt
gestadig `gejudaïseerd'. Dat wil zeggen dat Palestijnen in
Oost-Jeruzalem uit hun huizen worden gezet, die vervolgens joodse
inwoners krijgen. Het wordt voor Palestijnen die nog in Jeruzalem wonen
steeds moeilijker gemaakt, omdat ze onder andere hun residentierechten
kunnen verliezen als ze werk zoeken buiten de stadsgrenzen. En in de
Gazastrook zitten twee miljoen Palestijnen opgesloten, die een
vergunning nodig hebben om de strook te verlaten of weer binnen te
komen, die maar in zeldzame gevallen wordt verstrekt. De beperking van
de invoer van goederen als grondstoffen en het verbod op export, maakt
dat er nauwelijks meer van eigen productie sprake is. De Palestijnse
economie is verwoest, meer dan de helft van de bevolking is werkloos,
wat betekent dat een groot deel van de bevolking afhankelijk is van
externe voedselhulp, voornamelijk van de UNRWA, de VN. Door een gebrek
aan elektriciteit werken de pompen van de riolering niet voldoende,
waardoor het water vervuilt en binnenkort ondrinkbaar is. De
landbouwgrond die er in het overbevolkte Gaza nog over is, wordt voor
een groot deel ontoegankelijk gemaakt als het dichtbij de grens met
Israël ligt. Vissers mogen nog maar op een kleine strook water de zee
in. Dan hebben we het nog niet gehad over de massieve, verwoestende
aanvallen op Gaza, waarbij zoveel van wat met internationale hulp was
opgebouwd weer werd vernietigd, en duizenden mensen het leven lieten.

Die situatie binnen het land dat onder het gezag van Israël staat, dus
inclusief Oost-Jeruzalem, de Westoever en de Gazastrook, is onmenselijk
en valt op geen enkele manier te verdedigen. We vinden het zeer
gerechtvaardigd om tegen de bezettingspolitiek van Israël stelling te
nemen, en vinden dat iedereen die democratische waarden, mensenrechten
en gelijkwaardigheid hoog in het vaandel heeft dat ook zou moeten doen.

Is dat hetzelfde als `Israël willen vernietigen' zoals de aanhangers van
paradigma 1 zo vaak beweren? Dat is het niet. We gaan ervan uit dat
Israël een werkelijk democratische staat kan worden, met gelijke
burgerrechten voor alle inwoners ongeacht etniciteit of religie, dat de
bezetting moet worden opgeheven en het internationaal recht moet worden
geëerbiedigd. Het is niet te verdedigen dat er twee rechtssystemen zijn,
een voor joden en een voor Palestijnen. We wijzen het ook volledig af
dat onze kritiek op de staat Israël neer zou komen op een nieuwe vorm
van antisemitisme. De Israël-lobby in Nederland probeert al jaren
eenieder die bijvoorbeeld achter de BDS-beweging staat weg te zetten als
antisemieten. De pijlen zijn ook al enkele keren op BIJ1 gericht.

\textbf{De Internationale BDS-beweging}

In 2005 hebben 171 maatschappelijke Palestijnse organisaties opgeroepen
tot vreedzaam verzet tegen Israël. Het is een oproep tot boycot,
desinvesteren en sancties (BDS) tegen Israël totdat het haar
verplichtingen ten aanzien van de Palestijnen, conform het
internationaal recht, nakomt. (Zie BDS Nederland.)

Dit zijn de doelstellingen van de BDS-beweging:

1. Het beëindigen van de bezetting (en het koloniseren) van het Arabisch
gebied dat in juni 1967 bezet werd (VN-Resolutie 242) en het afbreken
van de (illegale) muur.

2. Het erkennen van de fundamentele rechten en de volledige
gelijkwaardigheid van de Arabisch-Palestijnse inwoners van Israël
(VN-Verdrag tegen Apartheid).

3. Het respecteren, beschermen en promoten van de rechten van
Palestijnse vluchtelingen om terug te keren naar hun huizen en
bezittingen (VN-Resolutie 194).

Let wel: al deze eisen vallen binnen de normen van het internationaal
recht. Deze vreedzame vorm van verzet, het boycotten van bedrijven en
instellingen (zowel Israëlisch, als niet Israëlisch) die bijdragen aan
of profiteren van het voortduren van de rechteloosheid, past binnen de
democratische rechtsorde en wordt beschermd door de vrijheid van
meningsuiting en de vrijheid van vergadering. Dit werd onlangs bevestigd
door het Europese Hof voor de Mensenrechten. Om het nog anders te
zeggen: de weigering om Israëlische aardappels te kopen bij Albert Heijn
is legitiem. Die aardappelen zijn niet joods, ze zijn Israëlisch. Gezien
het feit dat de Israëlische economie en die van de nederzettingen
onlosmakelijk met elkaar verbonden zijn en het bovendien voor
Israëlische bedrijven bij wet verboden is om de illegale nederzettingen
uit te sluiten van zaken, maakt het niet uit of een aardappel afkomstig
is uit Israël of uit een illegale nederzetting. Alle Israëlische
landbouwbedrijven zijn betrokken bij mensenrechtenschendingen.

Kortom, we spreken ons dus nadrukkelijk uit om waakzaam te zijn tegen
opnieuw de kop opstekend antisemitisme, en vooral om de
rechtsextremistische groepen te blijven monitoren. Waar joodse
instellingen zoals synagogen bedreigd worden, staan wij geheel achter
betere beveiliging. We komen ook op voor het recht van joden om hun
godsdienstige gebruiken te eerbiedigen, in het openbaar als joden
zichtbaar te mogen zijn, hun eigen instellingen zoals scholen te mogen
blijven handhaven, en door de overheid ondersteund moeten worden in het
openlijk herdenken van de Shoah.

Tegelijk gaan we ervan uit dat we niet alleen het recht maar ook de
plicht hebben om op te komen voor de rechten van Palestijnen, en er bij
de staat Israël, maar er ook bij onze eigen regering, op aan moeten
dringen dat het internationaal recht er is om geëerbiedigd te worden.
Ook door Israël.

De reden dat we ons druk maken over de koloniale politiek van Israël,
zoals we dat in Nederland ook eens gedaan hebben met de Zuid-Afrikaanse
strijd tegen de apartheid, is tweeledig. We voelen ons bij Israël
betrokken vanwege een gedeelde geschiedenis. Als Europees land zijn we
medeplichtig aan de Shoah, of op zijn minst heeft Nederland te weinig
gedaan om de jodenvervolging tegen te gaan. Dat mag geen reden zijn om
nu een ander volk tot slachtoffer te maken.

Daarnaast is het ook Israël zelf dat zich tracht te profileren als een
Europees land (zie het Songfestival en het EK voetbal) en een beroep
doet op onze vanzelfsprekende solidariteit. We zijn medeplichtig gemaakt
aan de politiek van Israël, en dat maakt ons extra verantwoordelijk. En
ja, we zijn zeker van mening dat Israël niet het enige land is waar het
internationaal recht wordt geschonden, en andere landen horen daar net
zo goed op aangesproken te worden. Wij horen ook in ons eigen land de
hand in eigen boezem te steken waar het gaat om het handhaven van
internationaal recht en mensenrechten.

\begin{enumerate}
\def\labelenumi{\arabic{enumi}.}
\setcounter{enumi}{16}
\item
  \begin{quote}
  \textbf{GENDER-GERELATEERD GEWELD}
  \end{quote}
\end{enumerate}

We leven in een verkrachtingscultuur: dat wil zeggen dat verkrachting is
genormaliseerd en zelfs wordt verexcuseerd in de media en in onze
cultuur. Voorbeelden van deze verkrachtingscultuur zijn dat vaak wordt
gedacht dat het slachtoffer iets heeft gedaan om het uit te lokken: `Ze
vroeg erom', of: `Wat had ze aan?'. Of dat seksueel overschrijdend en
agressief gedrag wordt gezien als iets dat nou eenmaal hoort bij het
`man-zijn', en dat mannen naar hun instinct handelen.

Zeker zijn smakeloze grapjes nog geen verkrachting, maar het feit dat
vrouwen geacht worden dat maar te accepteren plaveit de weg naar
volgende stappen. Vrouwen worden er vervolgens vaak van verdacht dat ze
overdrijven, en dat ze liegen. Zo wordt nog steeds niet serieus genomen
hoeveel vrouwen er dagelijks worden verkracht. De nadruk ligt op de
vrouwen die moeten leren om `weerbaar' te zijn, en er is te weinig
aandacht voor de vraag waarom mannen leren dat ze `recht' hebben op
seks. Verkrachtingscultuur zit in onze systemen, in onze media, maar ook
in onze taal. Zo komt er schuld en schaamte bij het slachtoffer terecht,
en niet bij de dader. Dit zorgt ook voor een hoge drempel van
aangiftebereidheid.

We leven in een samenleving die mannelijk geweld bagatelliseert,
legitimeert, normaliseert en zelfs uitlokt. Een halve eeuw nadat de
eerste feministen interpersoonlijk en seksueel geweld tot een thema
maakten, is er nog steeds geen teken dat dat geweld afneemt. Nog steeds
moeten we zorg dragen voor de slachtoffers/overlevers. De vrouwenopvang
en hulpverlening komt altijd geld, capaciteit en personeel tekort. Het
is dan ook dringend nodig om daarin structureel veel meer te investeren.
Maar er is veel meer nodig: we moeten erkennen dat emancipatie ook over
jongens en mannen gaat.

BIJ1 is van mening dat we ons meer op de daders moeten richten. Dit zou
ook tot veel effectiever beleid leiden. Om daadwerkelijk iets aan
gender-based violence te doen, is het belangrijk om het hele plaatje te
bekijken. De context is altijd relevant: het gaat niet slechts over het
incident, maar het gaat over economische ongelijkheid (vrouwen die niet
weg kunnen omdat ze financieel afhankelijk zijn), opvattingen over
gender, en de manier waarop onze samenleving en de publieke ruimte is
ingericht.

De Nederlandse overheid heeft een verantwoordelijkheid om geweld tegen
meisjes, vrouwen, en LHBQTI+-mensen tegen te gaan. Gender-gerelateerd
geweld is eigenlijk een pleonasme; geweld is namelijk per definitie
gender-gerelateerd in deze wereld. Het gaat niet alleen om geweld tegen
vrouwen, meisjes en LHBTQI+-mensen, maar ook over criminaliteit,
hooliganism, en andere vormen van geweld waarin mannelijkheid een
belangrijke rol speelt. Toch zullen wij de term gender-gerelateerd
geweld aanhouden omdat het verschillende vormen van geweld behelst. En,
door onze visie op het tegengaan van gender-gerelateerd geweld, zullen
ook andere vormen van geweld tegengegaan worden.

Bijna al het geld dat er nu wordt geïnvesteerd in justitie en veiligheid
wordt geïnvesteerd in bestraffen. De vormen van preventie zijn bij lange
na niet structureel genoeg. BIJ1 staat voor een samenleving waarin
iedereen zich veilig kan voelen om zich uit te drukken, te kleden, te
gedragen en te voelen zoals diegene wil. Dat betekent ook dat wij toe
willen naar een wereld waarin mannen veel meer ruimte ervaren in hun
gender uitdrukking en ervaring van mens-zijn. Waarin mannelijkheid niet
gelijk staat aan stoer zijn, hard zijn of gewelddadig zijn. Waarin er
ruimte is voor gevoeligheid, empathie en kwetsbaarheid. Waar mannen zich
niet meer misdragen om indruk te maken en geaccepteerd te worden door
andere mannen.

Te veel van het Nederlandse beleid is op het moment gebaseerd op een
gender-neutrale aanpak waarbij wordt weggelaten dat de meerderheid van
geweldplegers mannen zijn. Op die manier komt niet ter sprake hoe
mannelijkheid kan worden getransformeerd bij de jongens en mannen die
neigen tot stoer en grensoverschrijdend gedrag. Gedrag waarmee ze
anderen, maar uiteindelijk ook zichzelf schaden.

BIJ1 is voor een gender-transformatieve aanpak bij het tegengaan van
gender-gerelateerd geweld. Dit houdt in dat we jongens en mannen moeten
aanspreken op het feit dat ze bepaald gedrag vertonen, maar ze daarmee
ook andere mogelijkheden meegeven. Hun - en onze - opvattingen over
mannelijkheid kunnen zo getransformeerd worden dat ze meer opties
hebben, en dat daadkracht en empathie niet elkaars tegendelen hoeven te
zijn. Ook is er bij een gender-transformatieve aanpak aandacht voor het
versterken van de positie van vrouwen, het vergroten van de rol van
mannen, en het blootleggen en transformeren van ongelijke
machtsstructuren. Ook wordt hierin een non-binaire benadering van gender
nagestreefd, en worden gender en seksuele diversiteit erkend.

Deze gender-transformatieve aanpak is belangrijk bij preventie, maar ook
bij straffen. Mannen die zich vernederd voelen zullen na een straf niet
minder geneigd zijn tot geweld. Ook het onderwijs moet hier een grote
rol in gaan spelen. We moeten werken naar een integrale aanpak tegen
gender-gerelateerd geweld: op interpersoonlijk niveau, op
gemeenschapsniveau, op internationaal niveau en op institutioneel
niveau. Ook is er meer onderzoek nodig om in kaart te brengen welke
maatschappelijke problemen op welke manier mede veroorzaakt worden door
schadelijke opvattingen over mannelijkheid. Er moeten commissies worden
ingesteld, pilots worden opgezet, en uitgebreid onderzoek worden gedaan
naar mogelijkheden om gender-transformatief beleid op te stellen met
betrekking tot geweld.

BIJ1 ziet dat er steeds meer mannen zijn die deel van de oplossing
willen zijn. De overheid heeft hierin ook een faciliterende taak:
bijvoorbeeld door het stimuleren van meer betrokken vaderschap - dat
hand in hand gaat met het bevorderen van zorgzame mannelijkheid en
daarmee met het voorkomen van gewelddadige mannelijkheid - door
bijvoorbeeld het geboorteverlof uit te breiden. Ook is het belangrijk
dat de overheid jongens stimuleert om te werken binnen het onderwijs en
de zorg, beroepen die traditioneel als vrouwelijk worden gezien.

Alleen door de problemen rondom gender-gerelateerd geweld bij de wortel
aan te pakken, kunnen we er daadwerkelijk iets tegen doen.
